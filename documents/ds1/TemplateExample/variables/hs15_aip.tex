%EVERY VARIABLE HAS A OWN PAGE
	
	%	
	%	
	%
	
	\subsection{hs15\_aip}
	\label{subSection:hs15_aip}

	%TABLE FOR THE VARIABLE DETAILS
	\noindent\textbf{Variable:}\\
		\begin{tabular}{lL{11.3cm}}
			\label{tableVariable:hs15_aip}
			Name & hs15\_aip \\
			Label & 15. Tätigkeit: Name der HS/Akademie \\
			Beschreibung & - \\
			Skalenniveau & nominal \\
			Zugangswege &
				not-accessible,
 \\
			Panelvariablen & -
			 \\
			 \\
 \\
					Generierungsbeschreibung & Diese Variable enthält die offen erfassten Hochschulangaben in codierter Form. Als Codierliste für deutsche Hochschulen wurde das Destatis-Schlüsselverzeichnis für die Studenten- und Prüfungsstatistik (WiSe 2007/2008) verwendet (vgl. cl-destatis-hochschule-2008). Die Codes einiger ehemaliger Hochschulen wurden aus älteren Verzeichnissen ergänzt, wenn die Befragten diese angaben. Für die Hochschulen  im Ausland wurde eine projekteigene Codierliste verwendet (vgl. cl-dzhw-9). 
				 \\	
			 \\
		\end{tabular}






	
	\newpage
