%EVERY VARIABLE HAS A OWN PAGE
	
	% 	
	%
	
	\subsection{bact05\_g1r}
	\label{subSection:bact05_g1r}

	%TABLE FOR THE VARIABLE DETAILS
	\noindent\textbf{Variable:}\\
		\begin{tabular}{lL{11.3cm}}
			\label{tableVariable:bact05_g1r}
			Name & bact05\_g1r \\
			Label & Beginn nachschulische Tätigkeit: Jahr \\
			Beschreibung & - \\
			Skalenniveau & kontinuierlich \\
			Zugangswege &
				remote-desktop-suf,
				onsite-suf,
 \\
			Panelvariablen & -
			 \\
			 \\
 \\
					Generierungsbeschreibung & Das Jahr wurde aus der gemeinsam erfassten Angabe von Jahr und Monat generiert. 
				 \\	
			 \\
		\end{tabular}

		%TABLE FOR QUESTION DETAILS
		\vspace*{1 cm}
		\noindent\textbf{Frage:}\\
		\begin{tabular}{lL{11.3cm}}
			\label{tableQuestion:bact05_g1r}
			Nummer & 19 \\
			Einleitung der Frage & - \\
			Fragetext & Wann wollen Sie mit diesem Studium/dieser Berufsausbildung bzw. Tätigkeit beginnen? (Monat) \\
			Ausfüllanweisung & - \\
		\end{tabular}




		%TABLE FOR THE METRIC STATISTICS
		\vspace*{1 cm}
		\noindent\begin{minipage}[l]{.4\linewidth}
		\noindent\textbf{Statistische Daten:}\\
			\begin{tabular}{ll}
				\label{tableStatistics:bact05_g1r}
					Arithmetisches Mittel & 2009 \\
					Standardabweichung & 3 \\
					Minimum & 2009 \\
					Unteres Quartil & 2009 \\
					Median & 2009 \\
					Oberes Quartil & 2009 \\
					Maximum & 2011 \\
					Schiefe & 182 \\
					Wölbung & 142 \\
			\end{tabular}
		\end{minipage}%				
			%BOXPLOT, IMPORTANT: No space between end / start minimpage. if not: no side by side orientation of table and figure		
			\begin{minipage}[l]{.55\linewidth}
			\label{boxPlot:bact05_g1r}
			\center
			\begin{tikzpicture}
			\begin{axis}
	    		[
    				/pgf/number format/.cd,
		    	    use comma,
	    	    		1000 sep={},
    				ytick = {2009,2009,2009,2009,2009},
    				boxplot/draw direction=y,
  				boxplot/draw direction=y,
  				boxplot/every box/.style={fill=dzhwblue,draw=dzhwblue},
  				boxplot/every whisker/.style={dzhwblue},
  				boxplot/every median/.style={white},
	    			width=0.35\textwidth,
		      	height=0.6\textwidth,
    				hide x axis,
				axis y line*=left,
				ymin=2008,
				ymax=2010
    			]
    			\addplot+[
    				boxplot prepared={
	      		median=2009,
    		  		upper quartile=2009,
      			lower quartile=2009,
    	  			upper whisker=2009,
	    		  	lower whisker=2009
    				},
    			] coordinates {};
	  		\end{axis}
			\end{tikzpicture}
			\captionof{figure}{bact05\_g1r Verteilungen}
			\end{minipage}


			%TABLE FOR THE CONTINOUS VALUES
			\vspace*{1 cm}
			\noindent\textbf{Werte:}\\
			\begin{table}[!ht]
			\label{tableValues:bact05_g1r}
				\centering
				\begin{tabular}{C{1.7cm}L{6cm}rr}
					\toprule
					\textbf{Code} & \textbf{Wertelabel} & \textbf{Häufigkeiten} & \textbf{Prozent} \\
					\midrule
					
					\multicolumn{4}{l}{\textbf{Gültige Werte}}\\
							2009 &  & 1630 & 5.78 \\
							2010 &  & 54 & 0.19 \\
							2011 &  & 6 & 0.02 \\
						
					\midrule
					\multicolumn{4}{l}{\textbf{Fehlende Werte}}\\	
							-999 & weiß nicht & 7 & 0.02  \\
							-998 & keine Angabe & 59 & 0.21  \\
							-995 & keine Teilnahme (Panel) & 22249 & 78.95  \\
							-989 & filterbedingt fehlend & 4177 & 14.82  \\
					\midrule
					\multicolumn{2}{l}{\textbf{Summe (gesamt)}} & \textbf{28182} & \textbf{100} \\
				\bottomrule					
				\end{tabular}
				\caption{Werte der Variable bact05\_g1r}
			\end{table}
	
	\newpage
