%EVERY VARIABLE HAS A OWN PAGE
	
	% 	
	%
	
	\subsection{asys01\_r}
	\label{subSection:asys01_r}

	%TABLE FOR THE VARIABLE DETAILS
	\noindent\textbf{Variable:}\\
		\begin{tabular}{lL{11.3cm}}
			\label{tableVariable:asys01_r}
			Name & asys01\_r \\
			Label & Bundesland der Hochschulreife \\
			Beschreibung & - \\
			Skalenniveau & nominal \\
			Zugangswege &
				remote-desktop-suf,
				onsite-suf,
 \\
			Panelvariablen & -
			 \\
			 \\
 \\
					Generierungsbeschreibung & Diese Variable wurde zum Zweck der Pseudonymisierung generiert. Dabei wurde die Zuordnung des Destatis-Bundeslandschlüssel (vgl. cl-destatis-bundesland-1990) verwendet. Diese Codierliste wurde am den Wert "18 - Ausland" projekteigen erweitert.
				 \\	
			 \\
		\end{tabular}






			%TABLE FOR THE NOMINAL / ORDINAL VALUES
			\vspace*{1 cm}
			\noindent\textbf{Werte:}\\
			\begin{table}[!ht]
				\label{tableValues:asys01_r}
				\centering
				\begin{tabular}{C{1cm}L{6cm}rrR{1.5cm}}
					\toprule
					\textbf{Code} & \textbf{Wertelabel} & \textbf{Häufigkeiten} & \textbf{Prozent} & \textbf{gültige Prozent} \\
					\midrule
					\multicolumn{5}{l}{\textbf{Gültige Werte}}\\										
						
								1 & Schleswig-Holstein & 599 & 2.13 & 2.13 \\
								2 & Hamburg & 686 & 2.43 & 2.43 \\
								3 & Niedersachsen & 1828 & 6.49 & 6.49 \\
								4 & Bremen & 540 & 1.92 & 1.92 \\
								5 & Nordrhein-Westfalen & 6081 & 21.58 & 21.58 \\
								6 & Hessen & 2065 & 7.33 & 7.33 \\
								7 & Rheinland-Pfalz & 1377 & 4.89 & 4.89 \\
								8 & Baden-Württemberg & 3399 & 12.06 & 12.06 \\
								9 & Bayern & 3154 & 11.19 & 11.19 \\
								10 & Saarland & 386 & 1.37 & 1.37 \\
								11 & Berlin & 1050 & 3.73 & 3.73 \\
								12 & Brandenburg & 1251 & 4.44 & 4.44 \\
								13 & Mecklenburg-Vorpommern & 1616 & 5.73 & 5.73 \\
								14 & Sachsen & 1177 & 4.18 & 4.18 \\
								15 & Sachsen-Anhalt & 1887 & 6.7 & 6.7 \\
								16 & Thüringen & 1086 & 3.85 & 3.85 \\

					\midrule
					\multicolumn{5}{l}{\textbf{Fehlende Werte}}\\
						& & 0 & 0 & 0 \\										
					
					\midrule
						\multicolumn{2}{l}{\textbf{Summe (gültig)}} & \textbf{28182} & \textbf{100} & \textbf{100}\\
					\multicolumn{2}{l}{\textbf{Summe (gesamt)}} & \textbf{28182} & \textbf{100} & \textbf{-} \\			
					\bottomrule		
				\end{tabular}
				\caption{Werte der Variable asys01\_r}
			\end{table}

	
	\newpage
