%EVERY VARIABLE HAS A OWN PAGE
	
	% 	
	%
	
	\subsection{bstu20b\_g1o}
	\label{subSection:bstu20b_g1o}

	%TABLE FOR THE VARIABLE DETAILS
	\noindent\textbf{Variable:}\\
		\begin{tabular}{lL{11.3cm}}
			\label{tableVariable:bstu20b_g1o}
			Name & bstu20b\_g1o \\
			Label & Alternative Qualifizierung: Studienfach \\
			Beschreibung & - \\
			Skalenniveau & nominal \\
			Zugangswege &
				onsite-suf,
 \\
			Panelvariablen & -
			 \\
			Eingangsfilter & if inrange(bact02,4,6) | inrange(bact08,1,3) \\
 \\
					Generierungsbeschreibung & Diese Variable enthält die offen erfasste Angabe zu einem alternativ geplanten Studium in codierter Form. Als Codierliste für die Studienfächer wurde das Destatis-Schlüsselverzeichnis für die Studenten- und Prüfungsstatistik (WiSe 2007/08) verwendet. Einige darin nicht enthaltenen Fächercodes (ab 452 Techn.Informatik) wurden vom DZHW vergeben.
				 \\	
			 \\
		\end{tabular}

		%TABLE FOR QUESTION DETAILS
		\vspace*{1 cm}
		\noindent\textbf{Frage:}\\
		\begin{tabular}{lL{11.3cm}}
			\label{tableQuestion:bstu20b_g1o}
			Nummer & 30 \\
			Einleitung der Frage & - \\
			Fragetext & Was werden/wollen Sie statt dessen tun?
ein (anderes) Studium absolvieren, und zwar: \\
			Ausfüllanweisung & Bitte nur eine Antwort. \\
		\end{tabular}





			%TABLE FOR THE NOMINAL / ORDINAL VALUES
			\vspace*{1 cm}
			\noindent\textbf{Werte:}\\
			\begin{table}[!ht]
				\label{tableValues:bstu20b_g1o}
				\centering
				\begin{tabular}{C{1cm}L{6cm}rrR{1.5cm}}
					\toprule
					\textbf{Code} & \textbf{Wertelabel} & \textbf{Häufigkeiten} & \textbf{Prozent} & \textbf{gültige Prozent} \\
					\midrule
					\multicolumn{5}{l}{\textbf{Gültige Werte}}\\										
						
								4 & Interdisziplinäre Studien (Schwerpunkt Sprach- und Kulturwissenschaften) & 1 & 0 & 1.09 \\
								8 & Anglistik/Englisch & 1 & 0 & 1.09 \\
								12 & Archäologie & 1 & 0 & 1.09 \\
								13 & Architektur & 2 & 0.01 & 2.17 \\
								21 & Betriebswirtschaftslehre & 6 & 0.02 & 6.52 \\
								23 & Bildende Kunst/Graphik & 1 & 0 & 1.09 \\
								26 & Biologie & 1 & 0 & 1.09 \\
								29 & Sportwissenschaft & 1 & 0 & 1.09 \\
								32 & Chemie & 1 & 0 & 1.09 \\
							... & ... & ... & ... & ... \\
								226 & Verfahrenstechnik & 1 & 0 & 1.09 \\
								232 & Gesundheitswissenschaft/-management & 1 & 0 & 1.09 \\
								252 & Journalistik & 1 & 0 & 1.09 \\
								260 & Bundeswehrverwaltung & 1 & 0 & 1.09 \\
								274 & Tourismuswirtschaft & 2 & 0.01 & 2.17 \\
								320 & Ernährungswissenschaft & 2 & 0.01 & 2.17 \\
								458 & Umweltschutz & 1 & 0 & 1.09 \\
								553 & 553. Kommunale Verwaltung & 1 & 0 & 1.09 \\
								579 & 579. Gesundheitswesen & 1 & 0 & 1.09 \\

					\midrule
					\multicolumn{5}{l}{\textbf{Fehlende Werte}}\\
							-999 & weiß nicht & 1 & 0 & - \\						
							-998 & keine Angabe & 5 & 0.02 & - \\						
							-995 & keine Teilnahme (Panel) & 22249 & 78.95 & - \\						
							-989 & filterbedingt fehlend & 5835 & 20.7 & - \\						
					
					\midrule
						\multicolumn{2}{l}{\textbf{Summe (gültig)}} & \textbf{92} & \textbf{0.33} & \textbf{100}\\
					\multicolumn{2}{l}{\textbf{Summe (gesamt)}} & \textbf{28182} & \textbf{100} & \textbf{-} \\			
					\bottomrule		
				\end{tabular}
				\caption{Werte der Variable bstu20b\_g1o}
			\end{table}

	
	\newpage
