%EVERY VARIABLE HAS A OWN PAGE
	
	% 	
	%
	
	\subsection{cstu217c\_r}
	\label{subSection:cstu217c_r}

	%TABLE FOR THE VARIABLE DETAILS
	\noindent\textbf{Variable:}\\
		\begin{tabular}{lL{11.3cm}}
			\label{tableVariable:cstu217c_r}
			Name & cstu217c\_r \\
			Label & 7. Tätigkeit: angestrebter Abschluss \\
			Beschreibung & - \\
			Skalenniveau & nominal \\
			Zugangswege &
				remote-desktop-suf,
				onsite-suf,
 \\
			Panelvariablen & -
			 \\
			 \\
 \\
		\end{tabular}

		%TABLE FOR QUESTION DETAILS
		\vspace*{1 cm}
		\noindent\textbf{Frage:}\\
		\begin{tabular}{lL{11.3cm}}
			\label{tableQuestion:cstu217c_r}
			Nummer & 2.1 \\
			Einleitung der Frage & - \\
			Fragetext & Wir bitten Sie nun, uns in dem folgenden Schema einen Überblick Ihres Werdegangs von Juli 2008 bis Dezember 2012 zu geben.
Studium
Nennen Sie bitte Ihre angestrebte Abschlussprüfung 
(siehe Liste der Studienabschlüsse, rechte Seite oben) \\
			Ausfüllanweisung & Geben Sie bitte alle bisherigen Tätigkeiten – z. B. Studium, Berufsausbildung, Erwerbstätigkeit, aber auch Praktikum, Haushaltstätigkeit,
Erziehungszeit, Arbeitslosigkeit – mit ihren jeweiligen Anfangs- und Endterminen an. Tragen Sie die Tätigkeiten in die entsprechenden Spalten ein bzw. machen Sie die entsprechenden „Kreuze“. Verwenden Sie immer dann eine neue Zeile, wenn sich eine Änderung der Tätigkeit – beispielsweise auch Wechsel des Studienfachs oder der Hochschule – ergeben hat. Wichtig ist für uns, dass im zeitlichen Ablauf keine Lücken entstehen. Wenn sich wesentliche Tätigkeiten zeitlich überschneiden, geben Sie jede in einer eigenen Zeile an. \\
		\end{tabular}





			%TABLE FOR THE NOMINAL / ORDINAL VALUES
			\vspace*{1 cm}
			\noindent\textbf{Werte:}\\
			\begin{table}[!ht]
				\label{tableValues:cstu217c_r}
				\centering
				\begin{tabular}{C{1cm}L{6cm}rrR{1.5cm}}
					\toprule
					\textbf{Code} & \textbf{Wertelabel} & \textbf{Häufigkeiten} & \textbf{Prozent} & \textbf{gültige Prozent} \\
					\midrule
					\multicolumn{5}{l}{\textbf{Gültige Werte}}\\										
						
								2 & B.A/Diplom an Verwaltungsfachhochschule & 1 & 0 & 0.25 \\
								3 & B.A. an Fachhochschule & 68 & 0.24 & 17.22 \\
								4 & B.A. an Universität (außer LA) & 103 & 0.37 & 26.08 \\
								5 & M.A. an Fachhochschule & 22 & 0.08 & 5.57 \\
								6 & M.A. an Universität (außer LA) & 73 & 0.26 & 18.48 \\
								7 & Diplom an Fachhochschule & 4 & 0.01 & 1.01 \\
								8 & Diplom an Universität & 12 & 0.04 & 3.04 \\
								9 & LA B.A. GHR & 4 & 0.01 & 1.01 \\
								10 & LA B.A. Gymnasium & 12 & 0.04 & 3.04 \\
								11 & LA B.A. berufsbildende Schulen & 2 & 0.01 & 0.51 \\
								13 & LA Staatsexamen GHR & 8 & 0.03 & 2.03 \\
								14 & LA Staatsexamen Gymnasium & 16 & 0.06 & 4.05 \\
								17 & Staatsexamen (außer LA) & 46 & 0.16 & 11.65 \\
								18 & Magister & 4 & 0.01 & 1.01 \\
								19 & kirchlicher Abschluss & 3 & 0.01 & 0.76 \\
								21 & Abschluss im Ausland & 15 & 0.05 & 3.8 \\
								23 & deutscher \& ausländischer Abschluss & 2 & 0.01 & 0.51 \\

					\midrule
					\multicolumn{5}{l}{\textbf{Fehlende Werte}}\\
							-998 & keine Angabe & 3276 & 11.62 & - \\						
							-995 & keine Teilnahme (Panel) & 24511 & 86.97 & - \\						
					
					\midrule
						\multicolumn{2}{l}{\textbf{Summe (gültig)}} & \textbf{395} & \textbf{1.4} & \textbf{100}\\
					\multicolumn{2}{l}{\textbf{Summe (gesamt)}} & \textbf{28182} & \textbf{100} & \textbf{-} \\			
					\bottomrule		
				\end{tabular}
				\caption{Werte der Variable cstu217c\_r}
			\end{table}

	
	\newpage
