%EVERY VARIABLE HAS A OWN PAGE
	
	% 	
	%
	
	\subsection{cfin05e\_r}
	\label{subSection:cfin05e_r}

	%TABLE FOR THE VARIABLE DETAILS
	\noindent\textbf{Variable:}\\
		\begin{tabular}{lL{11.3cm}}
			\label{tableVariable:cfin05e_r}
			Name & cfin05e\_r \\
			Label & Finanzierung letze 2 Semester (in \%): Kredit \\
			Beschreibung & - \\
			Skalenniveau & kontinuierlich \\
			Zugangswege &
				remote-desktop-suf,
				onsite-suf,
 \\
			Panelvariablen & -
			 \\
			Eingangsfilter & if cstu22==2 \\
 \\
		\end{tabular}

		%TABLE FOR QUESTION DETAILS
		\vspace*{1 cm}
		\noindent\textbf{Frage:}\\
		\begin{tabular}{lL{11.3cm}}
			\label{tableQuestion:cfin05e_r}
			Nummer & 4.12 \\
			Einleitung der Frage & - \\
			Fragetext & Auf welche Weise haben Sie in den letzten beiden Studiensemestern Ihren Lebensunterhalt finanziert?
Kredit von einer Bank \\
			Ausfüllanweisung & Bitte geben Sie prozentual an, welcher Anteil auf welche Finanzierungsquelle entfällt \\
		\end{tabular}




		%TABLE FOR THE METRIC STATISTICS
		\vspace*{1 cm}
		\noindent\begin{minipage}[l]{.4\linewidth}
		\noindent\textbf{Statistische Daten:}\\
			\begin{tabular}{ll}
				\label{tableStatistics:cfin05e_r}
					Arithmetisches Mittel & 37.92 \\
					Standardabweichung & 100 \\
					Minimum & 5 \\
					Unteres Quartil & 15 \\
					Median & 40 \\
					Oberes Quartil & 50 \\
					Maximum & 100 \\
					Schiefe & 141 \\
					Wölbung & 87 \\
			\end{tabular}
		\end{minipage}%				
			%BOXPLOT, IMPORTANT: No space between end / start minimpage. if not: no side by side orientation of table and figure		
			\begin{minipage}[l]{.55\linewidth}
			\label{boxPlot:cfin05e_r}
			\center
			\begin{tikzpicture}
			\begin{axis}
	    		[
    				/pgf/number format/.cd,
		    	    use comma,
	    	    		1000 sep={},
    				ytick = {5,15,40,50,100},
    				boxplot/draw direction=y,
  				boxplot/draw direction=y,
  				boxplot/every box/.style={fill=dzhwblue,draw=dzhwblue},
  				boxplot/every whisker/.style={dzhwblue},
  				boxplot/every median/.style={white},
	    			width=0.35\textwidth,
		      	height=0.6\textwidth,
    				hide x axis,
				axis y line*=left,
				ymin=4,
				ymax=101
    			]
    			\addplot+[
    				boxplot prepared={
	      		median=40,
    		  		upper quartile=50,
      			lower quartile=15,
    	  			upper whisker=100,
	    		  	lower whisker=5
    				},
    			] coordinates {};
	  		\end{axis}
			\end{tikzpicture}
			\captionof{figure}{cfin05e\_r Verteilungen}
			\end{minipage}


			%TABLE FOR THE CONTINOUS VALUES
			\vspace*{1 cm}
			\noindent\textbf{Werte:}\\
			\begin{table}[!ht]
			\label{tableValues:cfin05e_r}
				\centering
				\begin{tabular}{C{1.7cm}L{6cm}rr}
					\toprule
					\textbf{Code} & \textbf{Wertelabel} & \textbf{Häufigkeiten} & \textbf{Prozent} \\
					\midrule
					
					\multicolumn{4}{l}{\textbf{Gültige Werte}}\\
						& & 112 & 0.4 \\	
						
					\midrule
					\multicolumn{4}{l}{\textbf{Fehlende Werte}}\\	
							-998 & keine Angabe & 2971 & 10.54  \\
							-995 & keine Teilnahme (Panel) & 24511 & 86.97  \\
							-989 & filterbedingt fehlend & 588 & 2.09  \\
					\midrule
					\multicolumn{2}{l}{\textbf{Summe (gesamt)}} & \textbf{28182} & \textbf{100} \\
				\bottomrule					
				\end{tabular}
				\caption{Werte der Variable cfin05e\_r}
			\end{table}
	
			%HISTOGRAM FOR CONTINOUS DATA, IF THERE ARE MORE THAN 10 BINs
			
			\begin{figure}[!ht]
			\label{histogram:cfin05e_r}
			\center
			\begin{tikzpicture}
			\begin{axis}[
			    /pgf/number format/.cd,
		    	    use comma,		    	    
	    	    		1000 sep={},
	    	    		ylabel={Absolute Häufigkeit},
			    xlabel={Werte},
			    ybar,
			    ymin=0,
			    axis y line*=left,
			    axis x line*=bottom
			]
			\addplot +[
				fill=dzhwblue, 
				draw=white,
			    hist={
			        bins=21,
			        data min=4.5,
			        data max=100.5
			    }   
			] table[row sep=\\,y index=0] {
				data\\
						5\\ 
						5\\ 
						7\\ 
						10\\ 
						10\\ 
						10\\ 
						10\\ 
						10\\ 
						10\\ 
						10\\ 
						10\\ 
						10\\ 
						10\\ 
						10\\ 
						10\\ 
						10\\ 
						10\\ 
						10\\ 
						10\\ 
						10\\ 
						10\\ 
						12\\ 
						15\\ 
						15\\ 
						15\\ 
						15\\ 
						15\\ 
						15\\ 
						15\\ 
						20\\ 
						20\\ 
						20\\ 
						20\\ 
						20\\ 
						20\\ 
						25\\ 
						25\\ 
						25\\ 
						25\\ 
						25\\ 
						25\\ 
						30\\ 
						30\\ 
						30\\ 
						30\\ 
						30\\ 
						30\\ 
						30\\ 
						30\\ 
						30\\ 
						30\\ 
						30\\ 
						30\\ 
						33\\ 
						35\\ 
						40\\ 
						40\\ 
						40\\ 
						40\\ 
						40\\ 
						40\\ 
						40\\ 
						40\\ 
						40\\ 
						40\\ 
						40\\ 
						40\\ 
						40\\ 
						40\\ 
						40\\ 
						40\\ 
						40\\ 
						41\\ 
						44\\ 
						50\\ 
						50\\ 
						50\\ 
						50\\ 
						50\\ 
						50\\ 
						50\\ 
						50\\ 
						50\\ 
						50\\ 
						50\\ 
						50\\ 
						50\\ 
						50\\ 
						50\\ 
						50\\ 
						55\\ 
						60\\ 
						60\\ 
						60\\ 
						60\\ 
						60\\ 
						70\\ 
						70\\ 
						75\\ 
						75\\ 
						75\\ 
						80\\ 
						80\\ 
						80\\ 
						80\\ 
						80\\ 
						80\\ 
						90\\ 
						90\\ 
						90\\ 
						100\\ 
						100\\ 
			};
			\end{axis}
			\end{tikzpicture}	
			\caption{Histogramm der Variable cfin05e\_r}	
			\end{figure}	
	\newpage
