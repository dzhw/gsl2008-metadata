%EVERY VARIABLE HAS A OWN PAGE
	
	% 	
	%
	
	\subsection{cstu216d\_g3r}
	\label{subSection:cstu216d_g3r}

	%TABLE FOR THE VARIABLE DETAILS
	\noindent\textbf{Variable:}\\
		\begin{tabular}{lL{11.3cm}}
			\label{tableVariable:cstu216d_g3r}
			Name & cstu216d\_g3r \\
			Label & 6. Tätigkeit: Staat einer ausl. Hochschule \\
			Beschreibung & - \\
			Skalenniveau & nominal \\
			Zugangswege &
				remote-desktop-suf,
				onsite-suf,
 \\
			Panelvariablen & -
			 \\
			 \\
 \\
					Generierungsbeschreibung & Die Staaten im Ausland wurden aus den codierten Hochschulangaben generiert. Die Generierung enthält ausschließlich die Länder von ausländische Hochschulen, deutschen Hochschulen wurden auf trifft nicht zu gesetzt. Für die Länder wurde eine projekteigene Codierliste verwendet (vgl. cl-dzhw-9). Die Variable enthält nur die fünf häufigsten Staaten, alle anderen Staaten wurden auf sonstige Staaten gesetzt. 
				 \\	
			 \\
		\end{tabular}

		%TABLE FOR QUESTION DETAILS
		\vspace*{1 cm}
		\noindent\textbf{Frage:}\\
		\begin{tabular}{lL{11.3cm}}
			\label{tableQuestion:cstu216d_g3r}
			Nummer & 2.1 \\
			Einleitung der Frage & - \\
			Fragetext & Wir bitten Sie nun, uns in dem folgenden Schema einen Überblick Ihres Werdegangs von Juli 2008 bis Dezember 2012 zu geben.
Studium
Name und Ort der Hochschule/Berufsakademie
(z. B. "Uni Köln", "FH Merseburg" oder "BA Mosbach") \\
			Ausfüllanweisung & Geben Sie bitte alle bisherigen Tätigkeiten – z. B. Studium, Berufsausbildung, Erwerbstätigkeit, aber auch Praktikum, Haushaltstätigkeit,
Erziehungszeit, Arbeitslosigkeit – mit ihren jeweiligen Anfangs- und Endterminen an. Tragen Sie die Tätigkeiten in die entsprechenden Spalten ein bzw. machen Sie die entsprechenden „Kreuze“. Verwenden Sie immer dann eine neue Zeile, wenn sich eine Änderung der Tätigkeit – beispielsweise auch Wechsel des Studienfachs oder der Hochschule – ergeben hat. Wichtig ist für uns, dass im zeitlichen Ablauf keine Lücken entstehen. Wenn sich wesentliche Tätigkeiten zeitlich überschneiden, geben Sie jede in einer eigenen Zeile an. \\
		\end{tabular}





			%TABLE FOR THE NOMINAL / ORDINAL VALUES
			\vspace*{1 cm}
			\noindent\textbf{Werte:}\\
			\begin{table}[!ht]
				\label{tableValues:cstu216d_g3r}
				\centering
				\begin{tabular}{C{1cm}L{6cm}rrR{1.5cm}}
					\toprule
					\textbf{Code} & \textbf{Wertelabel} & \textbf{Häufigkeiten} & \textbf{Prozent} & \textbf{gültige Prozent} \\
					\midrule
					\multicolumn{5}{l}{\textbf{Gültige Werte}}\\										
						
								9027 & HS in Brasilien & 2 & 0.01 & 3.08 \\
								9033 & HS in China, Republik (Taiwan) & 1 & 0 & 1.54 \\
								9034 & HS in China, Volksrepublik (China) & 1 & 0 & 1.54 \\
								9037 & HS in Dänemark ( Färöer \& Grönland) & 2 & 0.01 & 3.08 \\
								9048 & HS in Finnland & 1 & 0 & 1.54 \\
								9049 & HS in Frankreich & 7 & 0.02 & 10.77 \\
								9056 & HS in Großbritannien (Vereinigtes Königreich) & 10 & 0.04 & 15.38 \\
								9067 & HS in Irland & 2 & 0.01 & 3.08 \\
								9070 & HS in Italien & 3 & 0.01 & 4.62 \\
							... & ... & ... & ... & ... \\
								9147 & HS in Rumänien & 1 & 0 & 1.54 \\
								9148 & HS in Russland & 1 & 0 & 1.54 \\
								9155 & HS in Schweden & 7 & 0.02 & 10.77 \\
								9156 & HS in Schweiz & 6 & 0.02 & 9.23 \\
								9167 & HS in Spanien & 3 & 0.01 & 4.62 \\
								9172 & HS in Südafrika & 1 & 0 & 1.54 \\
								9189 & HS in Türkei & 1 & 0 & 1.54 \\
								9194 & HS in Ungarn & 2 & 0.01 & 3.08 \\
								9195 & HS in Uruguay & 1 & 0 & 1.54 \\

					\midrule
					\multicolumn{5}{l}{\textbf{Fehlende Werte}}\\
							-998 & keine Angabe & 3030 & 10.75 & - \\						
							-995 & keine Teilnahme (Panel) & 24511 & 86.97 & - \\						
							-988 & trifft nicht zu & 576 & 2.04 & - \\						
					
					\midrule
						\multicolumn{2}{l}{\textbf{Summe (gültig)}} & \textbf{65} & \textbf{0.23} & \textbf{100}\\
					\multicolumn{2}{l}{\textbf{Summe (gesamt)}} & \textbf{28182} & \textbf{100} & \textbf{-} \\			
					\bottomrule		
				\end{tabular}
				\caption{Werte der Variable cstu216d\_g3r}
			\end{table}

	
	\newpage
