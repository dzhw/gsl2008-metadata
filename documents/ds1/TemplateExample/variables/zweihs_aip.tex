%EVERY VARIABLE HAS A OWN PAGE
	
	%	
	%	
	%
	
	\subsection{zweihs\_aip}
	\label{subSection:zweihs_aip}

	%TABLE FOR THE VARIABLE DETAILS
	\noindent\textbf{Variable:}\\
		\begin{tabular}{lL{11.3cm}}
			\label{tableVariable:zweihs_aip}
			Name & zweihs\_aip \\
			Label & (2.) Qualifizierung: Hochschule \\
			Beschreibung & - \\
			Skalenniveau & nominal \\
			Zugangswege &
				not-accessible,
 \\
			Panelvariablen & -
			 \\
			Eingangsfilter & if inrange(bact02,4,6) | bstu09\_g4\textless{}0 | bstu10a\textless{}0| bestu10b\textless{}0 | inrange(bstu11,1,2) | inrange(bstu12,4,5) \\
 \\
					Generierungsbeschreibung & Diese Variable enthält die offen erfassten Hochschulangaben in codierter Form. Als Codierliste für deutsche Hochschulen wurde das Destatis-Schlüsselverzeichnis für die Studenten- und Prüfungsstatistik (WiSe 2007/2008) verwendet (vgl. cl-destatis-hochschule-2008). Für ausländische Hochschulen wurde eine projekteigene Codierliste verwendet. Die Codes einiger ehemaliger Hochschulen wurden aus älteren Verzeichnissen ergänzt, wenn die Befragten diese angaben.
				 \\	
			 \\
		\end{tabular}

		%TABLE FOR QUESTION DETAILS
		\vspace*{1 cm}
		\noindent\textbf{Frage:}\\
		\begin{tabular}{lL{11.3cm}}
			\label{tableQuestion:zweihs_aip}
			Nummer & 24 \\
			Einleitung der Frage & - \\
			Fragetext & Angaben zum möglichen/beabsichtigten Studium:
Name und Ort der Hochschule \\
			Ausfüllanweisung & - \\
		\end{tabular}





	
	\newpage
