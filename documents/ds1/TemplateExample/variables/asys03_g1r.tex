%EVERY VARIABLE HAS A OWN PAGE
	
	% 	
	%
	
	\subsection{asys03\_g1r}
	\label{subSection:asys03_g1r}

	%TABLE FOR THE VARIABLE DETAILS
	\noindent\textbf{Variable:}\\
		\begin{tabular}{lL{11.3cm}}
			\label{tableVariable:asys03_g1r}
			Name & asys03\_g1r \\
			Label & Schulform (allgemeinb. vs. berufl. Schulen) \\
			Beschreibung & - \\
			Skalenniveau & nominal \\
			Zugangswege &
				remote-desktop-suf,
				onsite-suf,
 \\
			Panelvariablen & -
			 \\
			 \\
 \\
					Generierungsbeschreibung & Diese Variable enthält einen vom Projekt vergebenen Code. Der Code gibt an, welcher Schulart eine Stichprobenschule zuzuordnen ist, über die die Befragten den Fragebogen erhalten haben. Dabei wurde eine projekteigene Codierliste verwendet (vgl. cl-dzhw-6).
				 \\	
			 \\
		\end{tabular}






			%TABLE FOR THE NOMINAL / ORDINAL VALUES
			\vspace*{1 cm}
			\noindent\textbf{Werte:}\\
			\begin{table}[!ht]
				\label{tableValues:asys03_g1r}
				\centering
				\begin{tabular}{C{1cm}L{6cm}rrR{1.5cm}}
					\toprule
					\textbf{Code} & \textbf{Wertelabel} & \textbf{Häufigkeiten} & \textbf{Prozent} & \textbf{gültige Prozent} \\
					\midrule
					\multicolumn{5}{l}{\textbf{Gültige Werte}}\\										
						
								10 & allgemeinbildene Schulen & 19782 & 70.19 & 70.19 \\
								11 & berufliche Schulen & 8400 & 29.81 & 29.81 \\

					\midrule
					\multicolumn{5}{l}{\textbf{Fehlende Werte}}\\
						& & 0 & 0 & 0 \\										
					
					\midrule
						\multicolumn{2}{l}{\textbf{Summe (gültig)}} & \textbf{28182} & \textbf{100} & \textbf{100}\\
					\multicolumn{2}{l}{\textbf{Summe (gesamt)}} & \textbf{28182} & \textbf{100} & \textbf{-} \\			
					\bottomrule		
				\end{tabular}
				\caption{Werte der Variable asys03\_g1r}
			\end{table}

	
	\newpage
