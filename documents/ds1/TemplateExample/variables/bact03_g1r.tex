%EVERY VARIABLE HAS A OWN PAGE
	
	% 	
	%
	
	\subsection{bact03\_g1r}
	\label{subSection:bact03_g1r}

	%TABLE FOR THE VARIABLE DETAILS
	\noindent\textbf{Variable:}\\
		\begin{tabular}{lL{11.3cm}}
			\label{tableVariable:bact03_g1r}
			Name & bact03\_g1r \\
			Label & anderer nachschulischer Werdegang (Nennung) \\
			Beschreibung & - \\
			Skalenniveau & nominal \\
			Zugangswege &
				remote-desktop-suf,
				onsite-suf,
 \\
			Panelvariablen & -
			 \\
			Eingangsfilter & if inrange(bact01,1,15) \\
 \\
					Generierungsbeschreibung & Diese Variable enthält die offen erfassten Nennungen anderer nachschulischer Werdegänge in codierter Form. Dazu wurde anhand der offenen Angaben (oder basierend auf bereits angelegten Listen auf Basis der Nennungen bisheriger Kohorten) eine Codierliste erstellt.
				 \\	
			 \\
		\end{tabular}

		%TABLE FOR QUESTION DETAILS
		\vspace*{1 cm}
		\noindent\textbf{Frage:}\\
		\begin{tabular}{lL{11.3cm}}
			\label{tableQuestion:bact03_g1r}
			Nummer & 18 \\
			Einleitung der Frage & - \\
			Fragetext & Für welchen nächsten Schritt Ihres nachschulischen Werdegangs haben Sie sich entschieden?
weder ein Studium noch berufliche Ausbildung noch Berufstätigkeit, sondern: \\
			Ausfüllanweisung & Bitte nur eine Antwort ankreuzen. \\
		\end{tabular}





			%TABLE FOR THE NOMINAL / ORDINAL VALUES
			\vspace*{1 cm}
			\noindent\textbf{Werte:}\\
			\begin{table}[!ht]
				\label{tableValues:bact03_g1r}
				\centering
				\begin{tabular}{C{1cm}L{6cm}rrR{1.5cm}}
					\toprule
					\textbf{Code} & \textbf{Wertelabel} & \textbf{Häufigkeiten} & \textbf{Prozent} & \textbf{gültige Prozent} \\
					\midrule
					\multicolumn{5}{l}{\textbf{Gültige Werte}}\\										
						
								1 & Haushaltstätigkeit & 4 & 0.01 & 100 \\

					\midrule
					\multicolumn{5}{l}{\textbf{Fehlende Werte}}\\
							-995 & keine Teilnahme (Panel) & 22249 & 78.95 & - \\						
							-989 & filterbedingt fehlend & 5929 & 21.04 & - \\						
					
					\midrule
						\multicolumn{2}{l}{\textbf{Summe (gültig)}} & \textbf{4} & \textbf{0.01} & \textbf{100}\\
					\multicolumn{2}{l}{\textbf{Summe (gesamt)}} & \textbf{28182} & \textbf{100} & \textbf{-} \\			
					\bottomrule		
				\end{tabular}
				\caption{Werte der Variable bact03\_g1r}
			\end{table}

	
	\newpage
