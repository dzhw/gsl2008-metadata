%EVERY VARIABLE HAS A OWN PAGE
	
	% 	
	%
	
	\subsection{bvoc05a\_g1r}
	\label{subSection:bvoc05a_g1r}

	%TABLE FOR THE VARIABLE DETAILS
	\noindent\textbf{Variable:}\\
		\begin{tabular}{lL{11.3cm}}
			\label{tableVariable:bvoc05a_g1r}
			Name & bvoc05a\_g1r \\
			Label & 1. Nennung Idee: Ausbildung/Berufstätigkeit \\
			Beschreibung & - \\
			Skalenniveau & nominal \\
			Zugangswege &
				remote-desktop-suf,
				onsite-suf,
 \\
			Panelvariablen & -
			 \\
			Eingangsfilter & if inrange(bact01,1,15) \\
 \\
					Generierungsbeschreibung & Diese Variable enthält die offen erfasste Angabe zum nächsten Schritt des nachschulischen Werdegangs (bact02) in codierter Form. Als Codierliste für die Angaben zum (Ausbildungs-)Beruf wurde die Destatis-Klassifikation der Berufe (KldB-92) (cl-destatis-kldb-1992) in der Ausgabe 1992 verwendet. 
				 \\	
			 \\
		\end{tabular}

		%TABLE FOR QUESTION DETAILS
		\vspace*{1 cm}
		\noindent\textbf{Frage:}\\
		\begin{tabular}{lL{11.3cm}}
			\label{tableQuestion:bvoc05a_g1r}
			Nummer & 18 \\
			Einleitung der Frage & - \\
			Fragetext & Für welchen nächsten Schritt Ihres nachschulischen Werdegangs haben Sie sich entschieden?
ich habe mich noch nicht endgültig entschieden, werde aber wahrscheinlich … \\
			Ausfüllanweisung & Bitte nur eine Antwort ankreuzen. \\
		\end{tabular}





			%TABLE FOR THE NOMINAL / ORDINAL VALUES
			\vspace*{1 cm}
			\noindent\textbf{Werte:}\\
			\begin{table}[!ht]
				\label{tableValues:bvoc05a_g1r}
				\centering
				\begin{tabular}{C{1cm}L{6cm}rrR{1.5cm}}
					\toprule
					\textbf{Code} & \textbf{Wertelabel} & \textbf{Häufigkeiten} & \textbf{Prozent} & \textbf{gültige Prozent} \\
					\midrule
					\multicolumn{5}{l}{\textbf{Gültige Werte}}\\										
						
								411 & Köche/Köchinnen & 1 & 0 & 4 \\
								441 & Maurer, Feuerungs- und Schornsteinbauer & 1 & 0 & 4 \\
								670 & Kaufleute o.n.A., Händler/Händlerinnen, a.n.g. & 1 & 0 & 4 \\
								691 & Bankfachleute & 1 & 0 & 4 \\
								703 & Werbefachleute & 1 & 0 & 4 \\
								726 & Luftverkehrsberufe & 2 & 0.01 & 8 \\
								774 & Datenverarbeitungsfachleute, Informatiker/Informatikerinnen o.n.A. & 1 & 0 & 4 \\
								775 & Softwareentwickler/Softwareentwicklerinnen & 1 & 0 & 4 \\
								785 & Industriekaufleute, Technische Kaufleute, Betriebswirte/Betriebswirtinnen (ohne Diplom), a.n.g. & 2 & 0.01 & 8 \\
								787 & Verwaltungsfachleute (mittlerer Dienst), a.n.g. & 2 & 0.01 & 8 \\
								801 & Soldaten, Grenzschutz-, Polizeibedienstete & 1 & 0 & 4 \\
								834 & Bildende Künstler/Künstlerinnen (angewandte Kunst) & 1 & 0 & 4 \\
								837 & Fotografen/Fotografinnen, Kameraleute & 1 & 0 & 4 \\
								852 & Masseure/Masseurinnen, Medizinische Bademeister/Bademeisterinnen und Krankengymnasten/Krankengymnastinnen & 1 & 0 & 4 \\
								853 & Krankenschwestern/-pfleger, Hebammen/Entbindungspfleger & 3 & 0.01 & 12 \\
								859 & Therapeutische Berufe, a.n.g. & 3 & 0.01 & 12 \\
								876 & Sportlehrer/Sportlehrerinnen & 1 & 0 & 4 \\
								902 & Kosmetiker/Kosmetikerinnen & 1 & 0 & 4 \\

					\midrule
					\multicolumn{5}{l}{\textbf{Fehlende Werte}}\\
							-998 & keine Angabe & 8 & 0.03 & - \\						
							-995 & keine Teilnahme (Panel) & 22249 & 78.95 & - \\						
							-989 & filterbedingt fehlend & 5821 & 20.66 & - \\						
							-969 & unbekannter fehlender Wert & 79 & 0.28 & - \\						
					
					\midrule
						\multicolumn{2}{l}{\textbf{Summe (gültig)}} & \textbf{25} & \textbf{0.09} & \textbf{100}\\
					\multicolumn{2}{l}{\textbf{Summe (gesamt)}} & \textbf{28182} & \textbf{100} & \textbf{-} \\			
					\bottomrule		
				\end{tabular}
				\caption{Werte der Variable bvoc05a\_g1r}
			\end{table}

	
	\newpage
