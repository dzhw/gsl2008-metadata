%EVERY VARIABLE HAS A OWN PAGE
	
	% 	
	%
	
	\subsection{cdec03h\_v1r}
	\label{subSection:cdec03h_v1r}

	%TABLE FOR THE VARIABLE DETAILS
	\noindent\textbf{Variable:}\\
		\begin{tabular}{lL{11.3cm}}
			\label{tableVariable:cdec03h_v1r}
			Name & cdec03h\_v1r \\
			Label & Grund Studienverzicht: Belastung BAföG-Darlehen \\
			Beschreibung & - \\
			Skalenniveau & nominal \\
			Zugangswege &
				remote-desktop-suf,
				onsite-suf,
 \\
			Panelvariablen & -
			 \\
			 \\
 \\
		\end{tabular}

		%TABLE FOR QUESTION DETAILS
		\vspace*{1 cm}
		\noindent\textbf{Frage:}\\
		\begin{tabular}{lL{11.3cm}}
			\label{tableQuestion:cdec03h_v1r}
			Nummer & 6.1 \\
			Einleitung der Frage & - \\
			Fragetext & Wenn Sie noch kein Studium an einer Universität oder einer Fachhochschule aufgenommen haben und dies auch zukünftig nicht tun wollen, nennen Sie uns bitte die ausschlaggebenden Gründe hierfür.
Ich fürchte die Belastung durch das BAföG-Darlehen. \\
			Ausfüllanweisung & Bitte machen Sie auch Angaben, wenn Sie ein Studium an einer Berufsakademie, Verwaltungsfachhochschule oder Hochschule der Bundeswehr aufnehmen bzw. aufgenommen haben. Mehrfachnennung möglich. \\
		\end{tabular}





			%TABLE FOR THE NOMINAL / ORDINAL VALUES
			\vspace*{1 cm}
			\noindent\textbf{Werte:}\\
			\begin{table}[!ht]
				\label{tableValues:cdec03h_v1r}
				\centering
				\begin{tabular}{C{1cm}L{6cm}rrR{1.5cm}}
					\toprule
					\textbf{Code} & \textbf{Wertelabel} & \textbf{Häufigkeiten} & \textbf{Prozent} & \textbf{gültige Prozent} \\
					\midrule
					\multicolumn{5}{l}{\textbf{Gültige Werte}}\\										
						
								0 & nicht genannt & 1363 & 4.84 & 94.19 \\
								1 & genannt & 84 & 0.3 & 5.81 \\

					\midrule
					\multicolumn{5}{l}{\textbf{Fehlende Werte}}\\
							-998 & keine Angabe & 48 & 0.17 & - \\						
							-995 & keine Teilnahme (Panel) & 24511 & 86.97 & - \\						
							-988 & trifft nicht zu & 2176 & 7.72 & - \\						
					
					\midrule
						\multicolumn{2}{l}{\textbf{Summe (gültig)}} & \textbf{1447} & \textbf{5.13} & \textbf{100}\\
					\multicolumn{2}{l}{\textbf{Summe (gesamt)}} & \textbf{28182} & \textbf{100} & \textbf{-} \\			
					\bottomrule		
				\end{tabular}
				\caption{Werte der Variable cdec03h\_v1r}
			\end{table}

	
	\newpage
