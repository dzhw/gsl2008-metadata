%EVERY VARIABLE HAS A OWN PAGE
	
	% 	
	%
	
	\subsection{bvoc07b\_g1o}
	\label{subSection:bvoc07b_g1o}

	%TABLE FOR THE VARIABLE DETAILS
	\noindent\textbf{Variable:}\\
		\begin{tabular}{lL{11.3cm}}
			\label{tableVariable:bvoc07b_g1o}
			Name & bvoc07b\_g1o \\
			Label & Alternative Qualifizierung: Beruf der Ausbildung \\
			Beschreibung & - \\
			Skalenniveau & nominal \\
			Zugangswege &
				onsite-suf,
 \\
			Panelvariablen & -
			 \\
			Eingangsfilter & if inrange(bact02,4,6) | inrange(bact08,1,3) \\
 \\
					Generierungsbeschreibung & Diese Variable enthält die offen erfasste Angabe zu einem alternativ geplanten beruflichen Ausbildungsberuf in codierter Form. Als Codierliste wurde die Destatis-Klassifikation der Berufe (KldB-92) (cl-destatis-kldb-1992) in der Ausgabe 1992 verwendet. 
				 \\	
			 \\
		\end{tabular}

		%TABLE FOR QUESTION DETAILS
		\vspace*{1 cm}
		\noindent\textbf{Frage:}\\
		\begin{tabular}{lL{11.3cm}}
			\label{tableQuestion:bvoc07b_g1o}
			Nummer & 30 \\
			Einleitung der Frage & - \\
			Fragetext & Was werden/wollen Sie statt dessen tun?
eine (andere) Berufsausbildung absolvieren, und zwar: \\
			Ausfüllanweisung & Bitte nur eine Antwort. \\
		\end{tabular}





			%TABLE FOR THE NOMINAL / ORDINAL VALUES
			\vspace*{1 cm}
			\noindent\textbf{Werte:}\\
			\begin{table}[!ht]
				\label{tableValues:bvoc07b_g1o}
				\centering
				\begin{tabular}{C{1cm}L{6cm}rrR{1.5cm}}
					\toprule
					\textbf{Code} & \textbf{Wertelabel} & \textbf{Häufigkeiten} & \textbf{Prozent} & \textbf{gültige Prozent} \\
					\midrule
					\multicolumn{5}{l}{\textbf{Gültige Werte}}\\										
						
								23 & Tier-, Pferde-, Fischwirte und -wirtinnen & 1 & 0 & 8.33 \\
								622 & Elektrotechniker/Elektrotechnikerinnen & 1 & 0 & 8.33 \\
								691 & Bankfachleute & 1 & 0 & 8.33 \\
								704 & Handelsmakler/Handelsmaklerinnen, Immobilienkaufleute & 1 & 0 & 8.33 \\
								754 & Fachgehilfen/Fachgehilfinnen in steuer- und wirtschaftsberatenden Berufen, Steuerfachleute, a.n.g. & 1 & 0 & 8.33 \\
								774 & Datenverarbeitungsfachleute, Informatiker/Informatikerinnen o.n.A. & 1 & 0 & 8.33 \\
								785 & Industriekaufleute, Technische Kaufleute, Betriebswirte/Betriebswirtinnen (ohne Diplom), a.n.g. & 2 & 0.01 & 16.67 \\
								834 & Bildende Künstler/Künstlerinnen (angewandte Kunst) & 1 & 0 & 8.33 \\
								853 & Krankenschwestern/-pfleger, Hebammen/Entbindungspfleger & 2 & 0.01 & 16.67 \\
								859 & Therapeutische Berufe, a.n.g. & 1 & 0 & 8.33 \\

					\midrule
					\multicolumn{5}{l}{\textbf{Fehlende Werte}}\\
							-995 & keine Teilnahme (Panel) & 22249 & 78.95 & - \\						
							-989 & filterbedingt fehlend & 5921 & 21.01 & - \\						
					
					\midrule
						\multicolumn{2}{l}{\textbf{Summe (gültig)}} & \textbf{12} & \textbf{0.04} & \textbf{100}\\
					\multicolumn{2}{l}{\textbf{Summe (gesamt)}} & \textbf{28182} & \textbf{100} & \textbf{-} \\			
					\bottomrule		
				\end{tabular}
				\caption{Werte der Variable bvoc07b\_g1o}
			\end{table}

	
	\newpage
