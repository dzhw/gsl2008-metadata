%EVERY VARIABLE HAS A OWN PAGE
	
	%	
	%	
	%
	
	\subsection{cjob0529a\_g1o}
	\label{subSection:cjob0529a_g1o}

	%TABLE FOR THE VARIABLE DETAILS
	\noindent\textbf{Variable:}\\
		\begin{tabular}{lL{11.3cm}}
			\label{tableVariable:cjob0529a_g1o}
			Name & cjob0529a\_g1o \\
			Label & 9. Job: Arbeitsort (PLZ) \\
			Beschreibung & - \\
			Skalenniveau & nominal \\
			Zugangswege &
				onsite-suf,
 \\
			Panelvariablen & -
			 \\
			 \\
 \\
					Generierungsbeschreibung & Diese Variable enthält die offen erfasste Angaben zum Ort einer Arbeit im Studium in codierter Form. Dabei wurden die Postleitzzahlen Als Codierliste verwendet. Angaben zu einem Ort im Ausland wurden in einer gesonderten Variable erfasst. 
				 \\	
			 \\
		\end{tabular}






			%TABLE FOR THE NOMINAL / ORDINAL VALUES
			\vspace*{1 cm}
			\noindent\textbf{Werte:}\\
			\begin{table}[!ht]
				\label{tableValues:cjob0529a_g1o}
				\centering
				\begin{tabular}{C{1cm}L{6cm}rrR{1.5cm}}
					\toprule
					\textbf{Code} & \textbf{Wertelabel} & \textbf{Häufigkeiten} & \textbf{Prozent} & \textbf{gültige Prozent} \\
					\midrule
					\multicolumn{5}{l}{\textbf{Gültige Werte}}\\										
						& & 0 & 0 & 0 \\

					\midrule
					\multicolumn{5}{l}{\textbf{Fehlende Werte}}\\
							-998 & keine Angabe & 3078 & 10.92 & - \\						
							-995 & keine Teilnahme (Panel) & 24511 & 86.97 & - \\						
							-989 & filterbedingt fehlend & 593 & 2.1 & - \\						
					
					\midrule
					\multicolumn{2}{l}{\textbf{Summe (gesamt)}} & \textbf{28182} & \textbf{100} & \textbf{-} \\			
					\bottomrule		
				\end{tabular}
				\caption{Werte der Variable cjob0529a\_g1o}
			\end{table}

	
	\newpage
