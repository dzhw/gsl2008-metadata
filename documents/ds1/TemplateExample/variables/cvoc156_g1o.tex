%EVERY VARIABLE HAS A OWN PAGE
	
	% 	
	%
	
	\subsection{cvoc156\_g1o}
	\label{subSection:cvoc156_g1o}

	%TABLE FOR THE VARIABLE DETAILS
	\noindent\textbf{Variable:}\\
		\begin{tabular}{lL{11.3cm}}
			\label{tableVariable:cvoc156_g1o}
			Name & cvoc156\_g1o \\
			Label & 6. Tätigkeit: Ausbildungsberuf \\
			Beschreibung & - \\
			Skalenniveau & nominal \\
			Zugangswege &
				onsite-suf,
 \\
			Panelvariablen & -
			 \\
			 \\
 \\
					Generierungsbeschreibung & Diese Variable enthält die offen erfassten Angaben zum Ausbildungsberuf in codierter Form. Als Codierliste wurde die Destatis-Klassifikation der Berufe (KldB-92) (cl-destatis-kldb-1992) in der Ausgabe 1992 verwendet. 
				 \\	
			 \\
		\end{tabular}

		%TABLE FOR QUESTION DETAILS
		\vspace*{1 cm}
		\noindent\textbf{Frage:}\\
		\begin{tabular}{lL{11.3cm}}
			\label{tableQuestion:cvoc156_g1o}
			Nummer & 2.1 \\
			Einleitung der Frage & - \\
			Fragetext & Wir bitten Sie nun, uns in dem folgenden Schema einen Überblick Ihres Werdegangs von Juli 2008 bis Dezember 2012 zu geben.
Berufliche Ausbildung
Art der Ausbildung und Ausbildungsberuf
(z. B. betriebliche Ausbildung Bürokauffrau/mann) \\
			Ausfüllanweisung & Geben Sie bitte alle bisherigen Tätigkeiten – z. B. Studium, Berufsausbildung, Erwerbstätigkeit, aber auch Praktikum, Haushaltstätigkeit,
Erziehungszeit, Arbeitslosigkeit – mit ihren jeweiligen Anfangs- und Endterminen an. Tragen Sie die Tätigkeiten in die entsprechenden Spalten ein bzw. machen Sie die entsprechenden „Kreuze“. Verwenden Sie immer dann eine neue Zeile, wenn sich eine Änderung der Tätigkeit – beispielsweise auch Wechsel des Studienfachs oder der Hochschule – ergeben hat. Wichtig ist für uns, dass im zeitlichen Ablauf keine Lücken entstehen. Wenn sich wesentliche Tätigkeiten zeitlich überschneiden, geben Sie jede in einer eigenen Zeile an. \\
		\end{tabular}





			%TABLE FOR THE NOMINAL / ORDINAL VALUES
			\vspace*{1 cm}
			\noindent\textbf{Werte:}\\
			\begin{table}[!ht]
				\label{tableValues:cvoc156_g1o}
				\centering
				\begin{tabular}{C{1cm}L{6cm}rrR{1.5cm}}
					\toprule
					\textbf{Code} & \textbf{Wertelabel} & \textbf{Häufigkeiten} & \textbf{Prozent} & \textbf{gültige Prozent} \\
					\midrule
					\multicolumn{5}{l}{\textbf{Gültige Werte}}\\										
						
								1210 & Keramiker/innen, Töpfer/innen, allgemein & 1 & 0 & 2.22 \\
								2700 & Industriemechaniker/innen o.n.F., Mechaniker/innen o.n.A. & 1 & 0 & 2.22 \\
								2813 & Zweiradmechaniker/innen & 1 & 0 & 2.22 \\
								2900 & Werkzeugmechaniker/innen, Werkzeugmacher/innen o.n.F. & 1 & 0 & 2.22 \\
								3112 & Energieelektroniker/innen (Betriebstechnik) & 1 & 0 & 2.22 \\
								4110 & Köch(e/innen), allgemein & 1 & 0 & 2.22 \\
								6203 & Planungs-, Projekttechniker/innen o.n.F. & 1 & 0 & 2.22 \\
								6700 & Kaufleute o.n.A. & 1 & 0 & 2.22 \\
								6711 & Groß- und Außenhandelskaufleute & 1 & 0 & 2.22 \\
							... & ... & ... & ... & ... \\
								8532 & Säuglings-, Kinderkrankenschwestern/-pfleger & 1 & 0 & 2.22 \\
								8571 & Medizinisch-technische Laboratoriumsassistent(en/innen), Medizinische Laboranten(en/innen) & 1 & 0 & 2.22 \\
								8594 & Beschäftigungs- und Arbeitstherapeut(en/innen) & 1 & 0 & 2.22 \\
								8620 & Heilpädagog(en/innen) & 2 & 0.01 & 4.44 \\
								8630 & Erzieher/innen o.n.A. & 2 & 0.01 & 4.44 \\
								8713 & Wissenschaftliche und künstlerische Mitarbeiter/innen (Hochschule) & 1 & 0 & 2.22 \\
								9020 & Kosmetiker/innen, allgemein & 1 & 0 & 2.22 \\
								9125 & Flugbegleiter/innen & 1 & 0 & 2.22 \\
								9212 & Hauswirtschafter/innen & 1 & 0 & 2.22 \\

					\midrule
					\multicolumn{5}{l}{\textbf{Fehlende Werte}}\\
							-998 & keine Angabe & 3625 & 12.86 & - \\						
							-995 & keine Teilnahme (Panel) & 24511 & 86.97 & - \\						
							-969 & unbekannter fehlender Wert & 1 & 0 & - \\						
					
					\midrule
						\multicolumn{2}{l}{\textbf{Summe (gültig)}} & \textbf{45} & \textbf{0.16} & \textbf{100}\\
					\multicolumn{2}{l}{\textbf{Summe (gesamt)}} & \textbf{28182} & \textbf{100} & \textbf{-} \\			
					\bottomrule		
				\end{tabular}
				\caption{Werte der Variable cvoc156\_g1o}
			\end{table}

	
	\newpage
