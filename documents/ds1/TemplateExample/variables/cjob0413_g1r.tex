%EVERY VARIABLE HAS A OWN PAGE
	
	% 	
	%
	
	\subsection{cjob0413\_g1r}
	\label{subSection:cjob0413_g1r}

	%TABLE FOR THE VARIABLE DETAILS
	\noindent\textbf{Variable:}\\
		\begin{tabular}{lL{11.3cm}}
			\label{tableVariable:cjob0413_g1r}
			Name & cjob0413\_g1r \\
			Label & 13. Tätigkeit: Beruf Erwerbstätigkeit \\
			Beschreibung & - \\
			Skalenniveau & nominal \\
			Zugangswege &
				remote-desktop-suf,
				onsite-suf,
 \\
			Panelvariablen & -
			 \\
			 \\
 \\
					Generierungsbeschreibung & Diese Variable enthält die offen erfassten Berufsangaben einer Erwerbstätigkeit in codierter Form. Als Codierliste wurde die Destatis-Klassifikation der Berufe (KldB-92) (cl-destatis-kldb-1992) in der Ausgabe 1992 verwendet. 
				 \\	
			 \\
		\end{tabular}






			%TABLE FOR THE NOMINAL / ORDINAL VALUES
			\vspace*{1 cm}
			\noindent\textbf{Werte:}\\
			\begin{table}[!ht]
				\label{tableValues:cjob0413_g1r}
				\centering
				\begin{tabular}{C{1cm}L{6cm}rrR{1.5cm}}
					\toprule
					\textbf{Code} & \textbf{Wertelabel} & \textbf{Häufigkeiten} & \textbf{Prozent} & \textbf{gültige Prozent} \\
					\midrule
					\multicolumn{5}{l}{\textbf{Gültige Werte}}\\										
						
								1 & Studium an Fachhochschule & 1 & 0 & 11.11 \\
								2 & Studium an Universität & 8 & 0.03 & 88.89 \\

					\midrule
					\multicolumn{5}{l}{\textbf{Fehlende Werte}}\\
							-998 & keine Angabe & 3662 & 12.99 & - \\						
							-995 & keine Teilnahme (Panel) & 24511 & 86.97 & - \\						
					
					\midrule
						\multicolumn{2}{l}{\textbf{Summe (gültig)}} & \textbf{9} & \textbf{0.03} & \textbf{100}\\
					\multicolumn{2}{l}{\textbf{Summe (gesamt)}} & \textbf{28182} & \textbf{100} & \textbf{-} \\			
					\bottomrule		
				\end{tabular}
				\caption{Werte der Variable cjob0413\_g1r}
			\end{table}

	
	\newpage
