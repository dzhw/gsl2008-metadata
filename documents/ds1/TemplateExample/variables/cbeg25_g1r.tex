%EVERY VARIABLE HAS A OWN PAGE
	
	% 	
	%
	
	\subsection{cbeg25\_g1r}
	\label{subSection:cbeg25_g1r}

	%TABLE FOR THE VARIABLE DETAILS
	\noindent\textbf{Variable:}\\
		\begin{tabular}{lL{11.3cm}}
			\label{tableVariable:cbeg25_g1r}
			Name & cbeg25\_g1r \\
			Label & 5. Job: Beginn (Jahr) \\
			Beschreibung & - \\
			Skalenniveau & kontinuierlich \\
			Zugangswege &
				remote-desktop-suf,
				onsite-suf,
 \\
			Panelvariablen & -
			 \\
			 \\
 \\
					Generierungsbeschreibung & Das Jahr wurde aus der gemeinsam erfassten Angabe von Jahr und Monat generiert. 
				 \\	
			 \\
		\end{tabular}

		%TABLE FOR QUESTION DETAILS
		\vspace*{1 cm}
		\noindent\textbf{Frage:}\\
		\begin{tabular}{lL{11.3cm}}
			\label{tableQuestion:cbeg25_g1r}
			Nummer & 4.14 \\
			Einleitung der Frage & - \\
			Fragetext & Geben Sie uns bitte nähere Auskünfte über die Art, Dauer und Nähe der Arbeitstätigkeit zu Ihrem bisherigen Studium.
Zeitraum
Beginn Monat/Jahr \\
			Ausfüllanweisung & Geben Sie bitte alle wesentlichen Tätigkeiten mit ihren jeweiligen Anfangs- und Endterminen an. Tragen Sie die Tätigkeiten in die entsprechenden Spalten ein bzw. machen Sie die entsprechenden „Kreuze“. \\
		\end{tabular}




		%TABLE FOR THE METRIC STATISTICS
		\vspace*{1 cm}
		\noindent\begin{minipage}[l]{.4\linewidth}
		\noindent\textbf{Statistische Daten:}\\
			\begin{tabular}{ll}
				\label{tableStatistics:cbeg25_g1r}
					Arithmetisches Mittel & 2011.7 \\
					Standardabweichung & 25 \\
					Minimum & 2008 \\
					Unteres Quartil & 2012 \\
					Median & 2012 \\
					Oberes Quartil & 2012 \\
					Maximum & 2012 \\
					Schiefe & 112 \\
					Wölbung & 57 \\
			\end{tabular}
		\end{minipage}%				
			%BOXPLOT, IMPORTANT: No space between end / start minimpage. if not: no side by side orientation of table and figure		
			\begin{minipage}[l]{.55\linewidth}
			\label{boxPlot:cbeg25_g1r}
			\center
			\begin{tikzpicture}
			\begin{axis}
	    		[
    				/pgf/number format/.cd,
		    	    use comma,
	    	    		1000 sep={},
    				ytick = {2012,2012,2012,2012,2012},
    				boxplot/draw direction=y,
  				boxplot/draw direction=y,
  				boxplot/every box/.style={fill=dzhwblue,draw=dzhwblue},
  				boxplot/every whisker/.style={dzhwblue},
  				boxplot/every median/.style={white},
	    			width=0.35\textwidth,
		      	height=0.6\textwidth,
    				hide x axis,
				axis y line*=left,
				ymin=2011,
				ymax=2013
    			]
    			\addplot+[
    				boxplot prepared={
	      		median=2012,
    		  		upper quartile=2012,
      			lower quartile=2012,
    	  			upper whisker=2012,
	    		  	lower whisker=2012
    				},
    			] coordinates {};
	  		\end{axis}
			\end{tikzpicture}
			\captionof{figure}{cbeg25\_g1r Verteilungen}
			\end{minipage}


			%TABLE FOR THE CONTINOUS VALUES
			\vspace*{1 cm}
			\noindent\textbf{Werte:}\\
			\begin{table}[!ht]
			\label{tableValues:cbeg25_g1r}
				\centering
				\begin{tabular}{C{1.7cm}L{6cm}rr}
					\toprule
					\textbf{Code} & \textbf{Wertelabel} & \textbf{Häufigkeiten} & \textbf{Prozent} \\
					\midrule
					
					\multicolumn{4}{l}{\textbf{Gültige Werte}}\\
							2008 &  & 1 & 0 \\
							2009 &  & 1 & 0 \\
							2010 &  & 1 & 0 \\
							2011 &  & 18 & 0.06 \\
							2012 &  & 66 & 0.23 \\
						
					\midrule
					\multicolumn{4}{l}{\textbf{Fehlende Werte}}\\	
							-998 & keine Angabe & 2403 & 8.53  \\
							-995 & keine Teilnahme (Panel) & 24511 & 86.97  \\
							-989 & filterbedingt fehlend & 1181 & 4.19  \\
					\midrule
					\multicolumn{2}{l}{\textbf{Summe (gesamt)}} & \textbf{28182} & \textbf{100} \\
				\bottomrule					
				\end{tabular}
				\caption{Werte der Variable cbeg25\_g1r}
			\end{table}
	
	\newpage
