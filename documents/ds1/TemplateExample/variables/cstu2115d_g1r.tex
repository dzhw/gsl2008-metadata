%EVERY VARIABLE HAS A OWN PAGE
	
	% 	
	%
	
	\subsection{cstu2115d\_g1r}
	\label{subSection:cstu2115d_g1r}

	%TABLE FOR THE VARIABLE DETAILS
	\noindent\textbf{Variable:}\\
		\begin{tabular}{lL{11.3cm}}
			\label{tableVariable:cstu2115d_g1r}
			Name & cstu2115d\_g1r \\
			Label & 15. Tätigkeit: Bundesland der HS \\
			Beschreibung & - \\
			Skalenniveau & nominal \\
			Zugangswege &
				remote-desktop-suf,
				onsite-suf,
 \\
			Panelvariablen & -
			 \\
			 \\
 \\
					Generierungsbeschreibung & Die Bundesländer wurden aus den codierten Hochschulangaben generiert. Die Generierung enthält ausschließlich Bundesländer, Länder von ausländische Hochschulen wurden auf trifft nicht zu gesetzt. Für die Zuordnung von Hochschulen zu Bundesländern wurde das Destatis Schlüsselverzeichnis für die Studenten- und Prüfungsstatistik (WiSe 2007/08) verwendet (cl-destatis-hochschule-2008).
				 \\	
			 \\
		\end{tabular}






			%TABLE FOR THE NOMINAL / ORDINAL VALUES
			\vspace*{1 cm}
			\noindent\textbf{Werte:}\\
			\begin{table}[!ht]
				\label{tableValues:cstu2115d_g1r}
				\centering
				\begin{tabular}{C{1cm}L{6cm}rrR{1.5cm}}
					\toprule
					\textbf{Code} & \textbf{Wertelabel} & \textbf{Häufigkeiten} & \textbf{Prozent} & \textbf{gültige Prozent} \\
					\midrule
					\multicolumn{5}{l}{\textbf{Gültige Werte}}\\										
						
								9 & Bayern & 1 & 0 & 100 \\

					\midrule
					\multicolumn{5}{l}{\textbf{Fehlende Werte}}\\
							-998 & keine Angabe & 3670 & 13.02 & - \\						
							-995 & keine Teilnahme (Panel) & 24511 & 86.97 & - \\						
					
					\midrule
						\multicolumn{2}{l}{\textbf{Summe (gültig)}} & \textbf{1} & \textbf{0} & \textbf{100}\\
					\multicolumn{2}{l}{\textbf{Summe (gesamt)}} & \textbf{28182} & \textbf{100} & \textbf{-} \\			
					\bottomrule		
				\end{tabular}
				\caption{Werte der Variable cstu2115d\_g1r}
			\end{table}

	
	\newpage
