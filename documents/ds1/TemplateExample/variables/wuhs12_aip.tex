%EVERY VARIABLE HAS A OWN PAGE
	
	%	
	%	
	%
	
	\subsection{wuhs12\_aip}
	\label{subSection:wuhs12_aip}

	%TABLE FOR THE VARIABLE DETAILS
	\noindent\textbf{Variable:}\\
		\begin{tabular}{lL{11.3cm}}
			\label{tableVariable:wuhs12_aip}
			Name & wuhs12\_aip \\
			Label & künftiges Studium: Hochschule \\
			Beschreibung & - \\
			Skalenniveau & nominal \\
			Zugangswege &
				not-accessible,
 \\
			Panelvariablen & -
			 \\
			Eingangsfilter & if cstu22==2 | inrange(cstu33\_g1,4,5) \\
 \\
					Generierungsbeschreibung & Diese Variable enthält die offen erfassten Hochschulangaben in codierter Form. Als Codierliste für deutsche Hochschulen wurde das Destatis-Schlüsselverzeichnis für die Studenten- und Prüfungsstatistik (WiSe 2006/07 und SoSe 2007) verwendet. Für ausländische Hochschulen wurde eine projekteigene Codierliste verwendet.
				 \\	
			 \\
		\end{tabular}

		%TABLE FOR QUESTION DETAILS
		\vspace*{1 cm}
		\noindent\textbf{Frage:}\\
		\begin{tabular}{lL{11.3cm}}
			\label{tableQuestion:wuhs12_aip}
			Nummer & 4.20 \\
			Einleitung der Frage & - \\
			Fragetext & An welcher Hochschule wollen Sie studieren? \\
			Ausfüllanweisung & Bitte Name und Ort der Hochschule angeben. \\
		\end{tabular}





	
	\newpage
