%EVERY VARIABLE HAS A OWN PAGE
	
	%	
	%	
	%
	
	\subsection{cjob0420\_g2r}
	\label{subSection:cjob0420_g2r}

	%TABLE FOR THE VARIABLE DETAILS
	\noindent\textbf{Variable:}\\
		\begin{tabular}{lL{11.3cm}}
			\label{tableVariable:cjob0420_g2r}
			Name & cjob0420\_g2r \\
			Label & 20. Tätigkeit: Berufsbereich Erwerbstätigkeit \\
			Beschreibung & - \\
			Skalenniveau & nominal \\
			Zugangswege &
				remote-desktop-suf,
				onsite-suf,
 \\
			Panelvariablen & -
			 \\
			 \\
 \\
					Generierungsbeschreibung & Die Berufsbereiche wurden aus den codierten Angaben zum Beruf generiert. Die Zuordnung ergibt sich auf Grundlage der Destatis-Klassifikation der Berufe (KldB-92) (cl-destatis-kldb-1992) in der Ausgabe 1992. 
				 \\	
			 \\
		\end{tabular}






			%TABLE FOR THE NOMINAL / ORDINAL VALUES
			\vspace*{1 cm}
			\noindent\textbf{Werte:}\\
			\begin{table}[!ht]
				\label{tableValues:cjob0420_g2r}
				\centering
				\begin{tabular}{C{1cm}L{6cm}rrR{1.5cm}}
					\toprule
					\textbf{Code} & \textbf{Wertelabel} & \textbf{Häufigkeiten} & \textbf{Prozent} & \textbf{gültige Prozent} \\
					\midrule
					\multicolumn{5}{l}{\textbf{Gültige Werte}}\\										
						& & 0 & 0 & 0 \\

					\midrule
					\multicolumn{5}{l}{\textbf{Fehlende Werte}}\\
							-998 & keine Angabe & 3671 & 13.03 & - \\						
							-995 & keine Teilnahme (Panel) & 24511 & 86.97 & - \\						
					
					\midrule
					\multicolumn{2}{l}{\textbf{Summe (gesamt)}} & \textbf{28182} & \textbf{100} & \textbf{-} \\			
					\bottomrule		
				\end{tabular}
				\caption{Werte der Variable cjob0420\_g2r}
			\end{table}

	
	\newpage
