%EVERY VARIABLE HAS A OWN PAGE
	
	% 	
	%
	
	\subsection{cact123\_g1r}
	\label{subSection:cact123_g1r}

	%TABLE FOR THE VARIABLE DETAILS
	\noindent\textbf{Variable:}\\
		\begin{tabular}{lL{11.3cm}}
			\label{tableVariable:cact123_g1r}
			Name & cact123\_g1r \\
			Label & 3. Tätigkeit: Tätigkeitsart \\
			Beschreibung & - \\
			Skalenniveau & nominal \\
			Zugangswege &
				remote-desktop-suf,
				onsite-suf,
 \\
			Panelvariablen & -
			 \\
			 \\
 \\
					Generierungsbeschreibung & Diese Variable enthält einen vom Projekt vergebenen Code. Der Code bündelt die  Informationen der Hochschulart bei einem Studium sowie der  offen erfassten Angaben zur Art der Berufsausbildung oder zu anderen Tätigkeiten. Dabei wurde eine projekteigene Codierliste verwendet.
				 \\	
			 \\
		\end{tabular}

		%TABLE FOR QUESTION DETAILS
		\vspace*{1 cm}
		\noindent\textbf{Frage:}\\
		\begin{tabular}{lL{11.3cm}}
			\label{tableQuestion:cact123_g1r}
			Nummer & 2.1 \\
			Einleitung der Frage & - \\
			Fragetext & Wir bitten Sie nun, uns in dem folgenden Schema einen Überblick Ihres Werdegangs von Juli 2008 bis Dezember 2012 zu geben.
Anderes
(z. B. Praktikum, Hausfrau/Hausmann, Jobben, au pair, Arbeitslosigkeit) \\
			Ausfüllanweisung & Geben Sie bitte alle bisherigen Tätigkeiten – z. B. Studium, Berufsausbildung, Erwerbstätigkeit, aber auch Praktikum, Haushaltstätigkeit,
Erziehungszeit, Arbeitslosigkeit – mit ihren jeweiligen Anfangs- und Endterminen an. Tragen Sie die Tätigkeiten in die entsprechenden Spalten ein bzw. machen Sie die entsprechenden „Kreuze“. Verwenden Sie immer dann eine neue Zeile, wenn sich eine Änderung der Tätigkeit – beispielsweise auch Wechsel des Studienfachs oder der Hochschule – ergeben hat. Wichtig ist für uns, dass im zeitlichen Ablauf keine Lücken entstehen. Wenn sich wesentliche Tätigkeiten zeitlich überschneiden, geben Sie jede in einer eigenen Zeile an. \\
		\end{tabular}





			%TABLE FOR THE NOMINAL / ORDINAL VALUES
			\vspace*{1 cm}
			\noindent\textbf{Werte:}\\
			\begin{table}[!ht]
				\label{tableValues:cact123_g1r}
				\centering
				\begin{tabular}{C{1cm}L{6cm}rrR{1.5cm}}
					\toprule
					\textbf{Code} & \textbf{Wertelabel} & \textbf{Häufigkeiten} & \textbf{Prozent} & \textbf{gültige Prozent} \\
					\midrule
					\multicolumn{5}{l}{\textbf{Gültige Werte}}\\										
						
								1 & Fachhochschule & 258 & 0.92 & 8.41 \\
								2 & Universität & 648 & 2.3 & 21.13 \\
								3 & Technische Hochschule & 146 & 0.52 & 4.76 \\
								5 & Pädagogische Hochschule & 14 & 0.05 & 0.46 \\
								6 & Kunst-/Musikhochschule & 9 & 0.03 & 0.29 \\
								7 & Theologische Hochschule & 1 & 0 & 0.03 \\
								8 & Berufsakademie & 11 & 0.04 & 0.36 \\
								9 & VwFH (Beamte gehobener Dienst) & 16 & 0.06 & 0.52 \\
								10 & Ausländische Hochschule & 108 & 0.38 & 3.52 \\
							... & ... & ... & ... & ... \\
								38 & Vorbereitung auf Ausbildung & 9 & 0.03 & 0.29 \\
								39 & Duales FH-Studium & 35 & 0.12 & 1.14 \\
								46 & FH / Erwerbstätigkeit & 15 & 0.05 & 0.49 \\
								47 & Uni / Erwerbstätigkeit & 9 & 0.03 & 0.29 \\
								48 & FH / Praktikum & 92 & 0.33 & 3 \\
								49 & Uni / Praktikum & 218 & 0.77 & 7.11 \\
								52 & Berufliche Fortbildung & 15 & 0.05 & 0.49 \\
								54 & Urlaubssemester & 9 & 0.03 & 0.29 \\
								56 & Studium o.n.A. (eng) & 1 & 0 & 0.03 \\

					\midrule
					\multicolumn{5}{l}{\textbf{Fehlende Werte}}\\
							-998 & keine Angabe & 604 & 2.14 & - \\						
							-995 & keine Teilnahme (Panel) & 24511 & 86.97 & - \\						
					
					\midrule
						\multicolumn{2}{l}{\textbf{Summe (gültig)}} & \textbf{3067} & \textbf{10.88} & \textbf{100}\\
					\multicolumn{2}{l}{\textbf{Summe (gesamt)}} & \textbf{28182} & \textbf{100} & \textbf{-} \\			
					\bottomrule		
				\end{tabular}
				\caption{Werte der Variable cact123\_g1r}
			\end{table}

	
	\newpage
