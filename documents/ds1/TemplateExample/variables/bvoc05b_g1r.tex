%EVERY VARIABLE HAS A OWN PAGE
	
	% 	
	%
	
	\subsection{bvoc05b\_g1r}
	\label{subSection:bvoc05b_g1r}

	%TABLE FOR THE VARIABLE DETAILS
	\noindent\textbf{Variable:}\\
		\begin{tabular}{lL{11.3cm}}
			\label{tableVariable:bvoc05b_g1r}
			Name & bvoc05b\_g1r \\
			Label & 2. Nennung Idee: Ausbildung/Berufstätigkeit \\
			Beschreibung & - \\
			Skalenniveau & nominal \\
			Zugangswege &
				remote-desktop-suf,
				onsite-suf,
 \\
			Panelvariablen & -
			 \\
			Eingangsfilter & if inrange(bact01,1,15) \\
 \\
					Generierungsbeschreibung & Diese Variable enthält die offen erfasste Angabe zum nächsten Schritt des nachschulischen Werdegangs (bact02) in codierter Form. Als Codierliste für die Angaben zum (Ausbildungs-)Beruf wurde die Destatis-Klassifikation der Berufe (KldB-92) (cl-destatis-kldb-1992) in der Ausgabe 1992 verwendet. 
				 \\	
			 \\
		\end{tabular}

		%TABLE FOR QUESTION DETAILS
		\vspace*{1 cm}
		\noindent\textbf{Frage:}\\
		\begin{tabular}{lL{11.3cm}}
			\label{tableQuestion:bvoc05b_g1r}
			Nummer & 18 \\
			Einleitung der Frage & - \\
			Fragetext & Für welchen nächsten Schritt Ihres nachschulischen Werdegangs haben Sie sich entschieden?
ich habe mich noch nicht endgültig entschieden, werde aber wahrscheinlich … \\
			Ausfüllanweisung & Bitte nur eine Antwort ankreuzen. \\
		\end{tabular}





			%TABLE FOR THE NOMINAL / ORDINAL VALUES
			\vspace*{1 cm}
			\noindent\textbf{Werte:}\\
			\begin{table}[!ht]
				\label{tableValues:bvoc05b_g1r}
				\centering
				\begin{tabular}{C{1cm}L{6cm}rrR{1.5cm}}
					\toprule
					\textbf{Code} & \textbf{Wertelabel} & \textbf{Häufigkeiten} & \textbf{Prozent} & \textbf{gültige Prozent} \\
					\midrule
					\multicolumn{5}{l}{\textbf{Gültige Werte}}\\										
						
								24 & Tierpfleger/Tierpflegerinnen und verwandte Berufe, a.n.g. & 1 & 0 & 3.13 \\
								501 & Tischler/Tischlerinnen & 1 & 0 & 3.13 \\
								674 & Buch-, Musikalienhändler und -händlerinnen & 1 & 0 & 3.13 \\
								691 & Bankfachleute & 1 & 0 & 3.13 \\
								693 & Kauffrau im Gesundheitsw & 1 & 0 & 3.13 \\
								711 & Schienenfahrzeugführer/Schienenfahrzeugführerinnen & 1 & 0 & 3.13 \\
								726 & Luftverkehrsberufe & 1 & 0 & 3.13 \\
								780 & Bürofachkräfte, Kaufmännische Angestellte o.n.A. & 4 & 0.01 & 12.5 \\
								785 & Industriekaufleute, Technische Kaufleute, Betriebswirte/Betriebswirtinnen (ohne Diplom), a.n.g. & 2 & 0.01 & 6.25 \\
							... & ... & ... & ... & ... \\
								835 & Künstlerische und zugeordnete Berufe der Bühnen-, Bild- und Tontechnik & 1 & 0 & 3.13 \\
								838 & Artisten/Artistinnen, Berufssportler/Berufssportlerinnen, künstlerische Hilfsberufe & 1 & 0 & 3.13 \\
								852 & Masseure/Masseurinnen, Medizinische Bademeister/Bademeisterinnen und Krankengymnasten/Krankengymnastinnen & 1 & 0 & 3.13 \\
								853 & Krankenschwestern/-pfleger, Hebammen/Entbindungspfleger & 2 & 0.01 & 6.25 \\
								856 & Sprechstundenhelfer/Sprechstundenhelferinnen & 1 & 0 & 3.13 \\
								857 & Medizinisch-technische Assistenten/Assistentinnen und verwandte Berufe & 1 & 0 & 3.13 \\
								859 & Therapeutische Berufe, a.n.g. & 1 & 0 & 3.13 \\
								863 & Erzieher/Erzieherinnen & 1 & 0 & 3.13 \\
								876 & Sportlehrer/Sportlehrerinnen & 1 & 0 & 3.13 \\

					\midrule
					\multicolumn{5}{l}{\textbf{Fehlende Werte}}\\
							-998 & keine Angabe & 21 & 0.07 & - \\						
							-995 & keine Teilnahme (Panel) & 22249 & 78.95 & - \\						
							-989 & filterbedingt fehlend & 5821 & 20.66 & - \\						
							-969 & unbekannter fehlender Wert & 59 & 0.21 & - \\						
					
					\midrule
						\multicolumn{2}{l}{\textbf{Summe (gültig)}} & \textbf{32} & \textbf{0.11} & \textbf{100}\\
					\multicolumn{2}{l}{\textbf{Summe (gesamt)}} & \textbf{28182} & \textbf{100} & \textbf{-} \\			
					\bottomrule		
				\end{tabular}
				\caption{Werte der Variable bvoc05b\_g1r}
			\end{table}

	
	\newpage
