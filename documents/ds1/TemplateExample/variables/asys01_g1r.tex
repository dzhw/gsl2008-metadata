%EVERY VARIABLE HAS A OWN PAGE
	
	% 	
	%
	
	\subsection{asys01\_g1r}
	\label{subSection:asys01_g1r}

	%TABLE FOR THE VARIABLE DETAILS
	\noindent\textbf{Variable:}\\
		\begin{tabular}{lL{11.3cm}}
			\label{tableVariable:asys01_g1r}
			Name & asys01\_g1r \\
			Label & alte/neue Bundesländer \\
			Beschreibung & - \\
			Skalenniveau & nominal \\
			Zugangswege &
				remote-desktop-suf,
				onsite-suf,
 \\
			Panelvariablen & -
			 \\
			 \\
					Generierungsregel & recode asys01 (1/11 = 1) (12/16 = 2) 
, gen(asys03) \\
				 \\
					Generierungsbeschreibung & Diese Variable wurde zum Zweck der Anonymisierung aus dem vom Projekt ermittelten Landes des HZB-Erwerbs. Zur Aggregation zu alten und neuen Bundesländern wurde die Zuordnung des Destatis-Bundeslandschlüssel (vgl. cl-destatis-bundesland-1990) verwendet.
				 \\	
			 \\
		\end{tabular}






			%TABLE FOR THE NOMINAL / ORDINAL VALUES
			\vspace*{1 cm}
			\noindent\textbf{Werte:}\\
			\begin{table}[!ht]
				\label{tableValues:asys01_g1r}
				\centering
				\begin{tabular}{C{1cm}L{6cm}rrR{1.5cm}}
					\toprule
					\textbf{Code} & \textbf{Wertelabel} & \textbf{Häufigkeiten} & \textbf{Prozent} & \textbf{gültige Prozent} \\
					\midrule
					\multicolumn{5}{l}{\textbf{Gültige Werte}}\\										
						
								1 & alte Länder & 20115 & 71.38 & 71.38 \\
								2 & neue Länder & 8067 & 28.62 & 28.62 \\

					\midrule
					\multicolumn{5}{l}{\textbf{Fehlende Werte}}\\
						& & 0 & 0 & 0 \\										
					
					\midrule
						\multicolumn{2}{l}{\textbf{Summe (gültig)}} & \textbf{28182} & \textbf{100} & \textbf{100}\\
					\multicolumn{2}{l}{\textbf{Summe (gesamt)}} & \textbf{28182} & \textbf{100} & \textbf{-} \\			
					\bottomrule		
				\end{tabular}
				\caption{Werte der Variable asys01\_g1r}
			\end{table}

	
	\newpage
