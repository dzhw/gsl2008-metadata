%EVERY VARIABLE HAS IT'S OWN PAGE

    \setcounter{footnote}{0}

    %omit vertical space
    \vspace*{-1.8cm}
	\section{bstu08c\_g1 (3. Nennung Idee: Art Hochschulabschluss)}
	\label{section:bstu08c_g1}



	%TABLE FOR VARIABLE DETAILS
    \vspace*{0.5cm}
    \noindent\textbf{Eigenschaften
	% '#' has to be escaped
	\footnote{Detailliertere Informationen zur Variable finden sich unter
		\url{https://metadata.fdz.dzhw.eu/\#!/de/variables/var-gsl2008-ds1-bstu08c_g1$}}}\\
	\begin{tabularx}{\hsize}{@{}lX}
	Datentyp: & numerisch \\
	Skalenniveau: & nominal \\
	Zugangswege: &
	  remote-desktop-suf, 
	  onsite-suf
 \\
    \end{tabularx}



    %TABLE FOR QUESTION DETAILS
    %This has to be tested and has to be improved
    %rausfinden, ob einer Variable mehrere Fragen zugeordnet werden
    %dann evtl. nur die erste verwenden oder etwas anderes tun (Hinweis mehrere Fragen, auflisten mit Link)
				%TABLE FOR QUESTION DETAILS
				\vspace*{0.5cm}
                \noindent\textbf{Frage
	                \footnote{Detailliertere Informationen zur Frage finden sich unter
		              \url{https://metadata.fdz.dzhw.eu/\#!/de/questions/que-gsl2008-ins2-18$}}}\\
				\begin{tabularx}{\hsize}{@{}lX}
					Fragenummer: &
					  Fragebogen des DZHW-Studienberechtigtenpanels 2008 - zweite Welle:
					  18
 \\
					%--
					Fragetext: & Für welchen nächsten Schritt Ihres nachschulischen Werdegangs haben Sie sich entschieden?\par  ich habe mich noch nicht endgültig entschieden, werde aber wahrscheinlich … \\
				\end{tabularx}





				%TABLE FOR THE NOMINAL / ORDINAL VALUES
        		\vspace*{0.5cm}
                \noindent\textbf{Häufigkeiten}

                \vspace*{-\baselineskip}
					%NUMERIC ELEMENTS NEED A HUGH SECOND COLOUMN AND A SMALL FIRST ONE
					\begin{filecontents}{\jobname-bstu08c_g1}
					\begin{longtable}{lXrrr}
					\toprule
					\textbf{Wert} & \textbf{Label} & \textbf{Häufigkeit} & \textbf{Prozent(gültig)} & \textbf{Prozent} \\
					\endhead
					\midrule
					\multicolumn{5}{l}{\textbf{Gültige Werte}}\\
						%DIFFERENT OBSERVATIONS <=20

					2 &
				% TODO try size/length gt 0; take over for other passages
					\multicolumn{1}{X}{ B.A./Diplom an Verwaltungsfachhochschule   } &


					%1 &
					  \num{1} &
					%--
					  \num[round-mode=places,round-precision=2]{7,69} &
					    \num[round-mode=places,round-precision=2]{0} \\
							%????

					3 &
				% TODO try size/length gt 0; take over for other passages
					\multicolumn{1}{X}{ B.A. an Fachhochschule   } &


					%5 &
					  \num{5} &
					%--
					  \num[round-mode=places,round-precision=2]{38,46} &
					    \num[round-mode=places,round-precision=2]{0,02} \\
							%????

					4 &
				% TODO try size/length gt 0; take over for other passages
					\multicolumn{1}{X}{ B.A. an Universität (außer LA)   } &


					%3 &
					  \num{3} &
					%--
					  \num[round-mode=places,round-precision=2]{23,08} &
					    \num[round-mode=places,round-precision=2]{0,01} \\
							%????

					5 &
				% TODO try size/length gt 0; take over for other passages
					\multicolumn{1}{X}{ LA Grund-, Haupt-, Realschule   } &


					%1 &
					  \num{1} &
					%--
					  \num[round-mode=places,round-precision=2]{7,69} &
					    \num[round-mode=places,round-precision=2]{0} \\
							%????

					9 &
				% TODO try size/length gt 0; take over for other passages
					\multicolumn{1}{X}{ Staatsexamen   } &


					%2 &
					  \num{2} &
					%--
					  \num[round-mode=places,round-precision=2]{15,38} &
					    \num[round-mode=places,round-precision=2]{0,01} \\
							%????

					15 &
				% TODO try size/length gt 0; take over for other passages
					\multicolumn{1}{X}{ Abschluss im Ausland   } &


					%1 &
					  \num{1} &
					%--
					  \num[round-mode=places,round-precision=2]{7,69} &
					    \num[round-mode=places,round-precision=2]{0} \\
							%????
						%DIFFERENT OBSERVATIONS >20
					\midrule
					\multicolumn{2}{l}{Summe (gültig)} &
					  \textbf{\num{13}} &
					\textbf{100} &
					  \textbf{\num[round-mode=places,round-precision=2]{0,05}} \\
					%--
					\multicolumn{5}{l}{\textbf{Fehlende Werte}}\\
							-999 &
							weiß nicht &
							  \num{1} &
							 - &
							  \num[round-mode=places,round-precision=2]{0} \\
							-998 &
							keine Angabe &
							  \num{2} &
							 - &
							  \num[round-mode=places,round-precision=2]{0,01} \\
							-995 &
							keine Teilnahme (Panel) &
							  \num{22249} &
							 - &
							  \num[round-mode=places,round-precision=2]{78,95} \\
							-989 &
							filterbedingt fehlend &
							  \num{5821} &
							 - &
							  \num[round-mode=places,round-precision=2]{20,66} \\
							-988 &
							trifft nicht zu &
							  \num{96} &
							 - &
							  \num[round-mode=places,round-precision=2]{0,34} \\
					\midrule
					\multicolumn{2}{l}{\textbf{Summe (gesamt)}} &
				      \textbf{\num{28182}} &
				    \textbf{-} &
				    \textbf{100} \\
					\bottomrule
					\end{longtable}
					\end{filecontents}
					\LTXtable{\textwidth}{\jobname-bstu08c_g1}
				\label{tableValues:bstu08c_g1}
				\vspace*{-\baselineskip}
                    \begin{noten}
                	    \note{} Deskritive Maßzahlen:
                	    Anzahl unterschiedlicher Beobachtungen: 6%
                	    ; 
                	      Modus ($h$): 3
                     \end{noten}


