%EVERY VARIABLE HAS IT'S OWN PAGE

    \setcounter{footnote}{0}

    %omit vertical space
    \vspace*{-1.8cm}
	\section{astu03d\_g1 (unsichere Präferenz: 2. Hauptstudienfach (Studienbereich))}
	\label{section:astu03d_g1}



	%TABLE FOR VARIABLE DETAILS
    \vspace*{0.5cm}
    \noindent\textbf{Eigenschaften
	% '#' has to be escaped
	\footnote{Detailliertere Informationen zur Variable finden sich unter
		\url{https://metadata.fdz.dzhw.eu/\#!/de/variables/var-gsl2008-ds1-astu03d_g1$}}}\\
	\begin{tabularx}{\hsize}{@{}lX}
	Datentyp: & numerisch \\
	Skalenniveau: & nominal \\
	Zugangswege: &
	  remote-desktop-suf, 
	  onsite-suf
 \\
    \end{tabularx}



    %TABLE FOR QUESTION DETAILS
    %This has to be tested and has to be improved
    %rausfinden, ob einer Variable mehrere Fragen zugeordnet werden
    %dann evtl. nur die erste verwenden oder etwas anderes tun (Hinweis mehrere Fragen, auflisten mit Link)
				%TABLE FOR QUESTION DETAILS
				\vspace*{0.5cm}
                \noindent\textbf{Frage
	                \footnote{Detailliertere Informationen zur Frage finden sich unter
		              \url{https://metadata.fdz.dzhw.eu/\#!/de/questions/que-gsl2008-ins1-14$}}}\\
				\begin{tabularx}{\hsize}{@{}lX}
					Fragenummer: &
					  Fragebogen des DZHW-Studienberechtigtenpanels 2008 - erste Welle:
					  14
 \\
					%--
					Fragetext: & Welches Studienfach wird dies voraussichtlich sein? \\
				\end{tabularx}





				%TABLE FOR THE NOMINAL / ORDINAL VALUES
        		\vspace*{0.5cm}
                \noindent\textbf{Häufigkeiten}

                \vspace*{-\baselineskip}
					%NUMERIC ELEMENTS NEED A HUGH SECOND COLOUMN AND A SMALL FIRST ONE
					\begin{filecontents}{\jobname-astu03d_g1}
					\begin{longtable}{lXrrr}
					\toprule
					\textbf{Wert} & \textbf{Label} & \textbf{Häufigkeit} & \textbf{Prozent(gültig)} & \textbf{Prozent} \\
					\endhead
					\midrule
					\multicolumn{5}{l}{\textbf{Gültige Werte}}\\
						%DIFFERENT OBSERVATIONS <=20
								1 & \multicolumn{1}{X}{Sprach- und Kulturwissenschaften allgemein} & %37 &
								  \num{37} &
								%--
								  \num[round-mode=places,round-precision=2]{0,68} &
								  \num[round-mode=places,round-precision=2]{0,13} \\
								2 & \multicolumn{1}{X}{Evang. Theologie, -Religionslehre} & %12 &
								  \num{12} &
								%--
								  \num[round-mode=places,round-precision=2]{0,22} &
								  \num[round-mode=places,round-precision=2]{0,04} \\
								3 & \multicolumn{1}{X}{Kath. Theologie, Religionslehre} & %8 &
								  \num{8} &
								%--
								  \num[round-mode=places,round-precision=2]{0,15} &
								  \num[round-mode=places,round-precision=2]{0,03} \\
								4 & \multicolumn{1}{X}{Philosophie} & %50 &
								  \num{50} &
								%--
								  \num[round-mode=places,round-precision=2]{0,91} &
								  \num[round-mode=places,round-precision=2]{0,18} \\
								5 & \multicolumn{1}{X}{Geschichte} & %86 &
								  \num{86} &
								%--
								  \num[round-mode=places,round-precision=2]{1,57} &
								  \num[round-mode=places,round-precision=2]{0,31} \\
								6 & \multicolumn{1}{X}{Bibliothekswissenschaft, Dokumentation Publizistik} & %258 &
								  \num{258} &
								%--
								  \num[round-mode=places,round-precision=2]{4,71} &
								  \num[round-mode=places,round-precision=2]{0,92} \\
								7 & \multicolumn{1}{X}{Allgemeine und vergleichende Literatur- und Sprachwissenschaft} & %40 &
								  \num{40} &
								%--
								  \num[round-mode=places,round-precision=2]{0,73} &
								  \num[round-mode=places,round-precision=2]{0,14} \\
								8 & \multicolumn{1}{X}{Altphilologie (klass. Phililologie), Neugriechisch} & %11 &
								  \num{11} &
								%--
								  \num[round-mode=places,round-precision=2]{0,2} &
								  \num[round-mode=places,round-precision=2]{0,04} \\
								9 & \multicolumn{1}{X}{Germanistik (Deutsch, germanische Sprachen ohne Anglistik)} & %78 &
								  \num{78} &
								%--
								  \num[round-mode=places,round-precision=2]{1,42} &
								  \num[round-mode=places,round-precision=2]{0,28} \\
								10 & \multicolumn{1}{X}{Anglistik, Amerikanistik} & %83 &
								  \num{83} &
								%--
								  \num[round-mode=places,round-precision=2]{1,52} &
								  \num[round-mode=places,round-precision=2]{0,29} \\
							... & ... & ... & ... & ... \\
								65 & \multicolumn{1}{X}{Verkehrstechnick, Nautik} & %103 &
								  \num{103} &
								%--
								  \num[round-mode=places,round-precision=2]{1,88} &
								  \num[round-mode=places,round-precision=2]{0,37} \\

								66 & \multicolumn{1}{X}{Architektur, Innenarchitektur} & %116 &
								  \num{116} &
								%--
								  \num[round-mode=places,round-precision=2]{2,12} &
								  \num[round-mode=places,round-precision=2]{0,41} \\

								67 & \multicolumn{1}{X}{Raumplanung} & %25 &
								  \num{25} &
								%--
								  \num[round-mode=places,round-precision=2]{0,46} &
								  \num[round-mode=places,round-precision=2]{0,09} \\

								68 & \multicolumn{1}{X}{Bauningenieurwesen} & %102 &
								  \num{102} &
								%--
								  \num[round-mode=places,round-precision=2]{1,86} &
								  \num[round-mode=places,round-precision=2]{0,36} \\

								69 & \multicolumn{1}{X}{Vermessungswesen} & %5 &
								  \num{5} &
								%--
								  \num[round-mode=places,round-precision=2]{0,09} &
								  \num[round-mode=places,round-precision=2]{0,02} \\

								74 & \multicolumn{1}{X}{Kunst, Kunstwissenschaft allgemein} & %77 &
								  \num{77} &
								%--
								  \num[round-mode=places,round-precision=2]{1,41} &
								  \num[round-mode=places,round-precision=2]{0,27} \\

								75 & \multicolumn{1}{X}{Bildende Kunst} & %51 &
								  \num{51} &
								%--
								  \num[round-mode=places,round-precision=2]{0,93} &
								  \num[round-mode=places,round-precision=2]{0,18} \\

								76 & \multicolumn{1}{X}{Gestaltung} & %165 &
								  \num{165} &
								%--
								  \num[round-mode=places,round-precision=2]{3,01} &
								  \num[round-mode=places,round-precision=2]{0,59} \\

								77 & \multicolumn{1}{X}{Darstellende Kunst, Film und Fernsehen, Theaterwissenschaft} & %86 &
								  \num{86} &
								%--
								  \num[round-mode=places,round-precision=2]{1,57} &
								  \num[round-mode=places,round-precision=2]{0,31} \\

								78 & \multicolumn{1}{X}{Musik, Musikwissenschaft} & %83 &
								  \num{83} &
								%--
								  \num[round-mode=places,round-precision=2]{1,52} &
								  \num[round-mode=places,round-precision=2]{0,29} \\

					\midrule
					\multicolumn{2}{l}{Summe (gültig)} &
					  \textbf{\num{5474}} &
					\textbf{100} &
					  \textbf{\num[round-mode=places,round-precision=2]{19,42}} \\
					%--
					\multicolumn{5}{l}{\textbf{Fehlende Werte}}\\
							-999 &
							weiß nicht &
							  \num{1} &
							 - &
							  \num[round-mode=places,round-precision=2]{0} \\
							-989 &
							filterbedingt fehlend &
							  \num{5803} &
							 - &
							  \num[round-mode=places,round-precision=2]{20,59} \\
							-969 &
							unbekannter fehlender Wert &
							  \num{16904} &
							 - &
							  \num[round-mode=places,round-precision=2]{59,98} \\
					\midrule
					\multicolumn{2}{l}{\textbf{Summe (gesamt)}} &
				      \textbf{\num{28182}} &
				    \textbf{-} &
				    \textbf{100} \\
					\bottomrule
					\end{longtable}
					\end{filecontents}
					\LTXtable{\textwidth}{\jobname-astu03d_g1}
				\label{tableValues:astu03d_g1}
				\vspace*{-\baselineskip}
                    \begin{noten}
                	    \note{} Deskritive Maßzahlen:
                	    Anzahl unterschiedlicher Beobachtungen: 58%
                	    ; 
                	      Modus ($h$): 30
                     \end{noten}


