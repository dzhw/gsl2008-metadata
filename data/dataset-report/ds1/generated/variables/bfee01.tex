%EVERY VARIABLE HAS IT'S OWN PAGE

    \setcounter{footnote}{0}

    %omit vertical space
    \vspace*{-1.8cm}
	\section{bfee01 (Studiengebühren: Auswirkungen auf Studienpläne)}
	\label{section:bfee01}



	%TABLE FOR VARIABLE DETAILS
    \vspace*{0.5cm}
    \noindent\textbf{Eigenschaften
	% '#' has to be escaped
	\footnote{Detailliertere Informationen zur Variable finden sich unter
		\url{https://metadata.fdz.dzhw.eu/\#!/de/variables/var-gsl2008-ds1-bfee01$}}}\\
	\begin{tabularx}{\hsize}{@{}lX}
	Datentyp: & numerisch \\
	Skalenniveau: & nominal \\
	Zugangswege: &
	  remote-desktop-suf, 
	  onsite-suf
 \\
    \end{tabularx}



    %TABLE FOR QUESTION DETAILS
    %This has to be tested and has to be improved
    %rausfinden, ob einer Variable mehrere Fragen zugeordnet werden
    %dann evtl. nur die erste verwenden oder etwas anderes tun (Hinweis mehrere Fragen, auflisten mit Link)
				%TABLE FOR QUESTION DETAILS
				\vspace*{0.5cm}
                \noindent\textbf{Frage
	                \footnote{Detailliertere Informationen zur Frage finden sich unter
		              \url{https://metadata.fdz.dzhw.eu/\#!/de/questions/que-gsl2008-ins2-35$}}}\\
				\begin{tabularx}{\hsize}{@{}lX}
					Fragenummer: &
					  Fragebogen des DZHW-Studienberechtigtenpanels 2008 - zweite Welle:
					  35
 \\
					%--
					Fragetext: & In einigen Bundesländern werden Studiengebühren von bis zu 500€ ab dem ersten Semester erhoben. Welche Auswirkungen hat dies auf Ihre Studienpläne? \\
				\end{tabularx}





				%TABLE FOR THE NOMINAL / ORDINAL VALUES
        		\vspace*{0.5cm}
                \noindent\textbf{Häufigkeiten}

                \vspace*{-\baselineskip}
					%NUMERIC ELEMENTS NEED A HUGH SECOND COLOUMN AND A SMALL FIRST ONE
					\begin{filecontents}{\jobname-bfee01}
					\begin{longtable}{lXrrr}
					\toprule
					\textbf{Wert} & \textbf{Label} & \textbf{Häufigkeit} & \textbf{Prozent(gültig)} & \textbf{Prozent} \\
					\endhead
					\midrule
					\multicolumn{5}{l}{\textbf{Gültige Werte}}\\
						%DIFFERENT OBSERVATIONS <=20

					1 &
				% TODO try size/length gt 0; take over for other passages
					\multicolumn{1}{X}{ nie Studienwunsch gehabt   } &


					%241 &
					  \num{241} &
					%--
					  \num[round-mode=places,round-precision=2]{4,1} &
					    \num[round-mode=places,round-precision=2]{0,86} \\
							%????

					2 &
				% TODO try size/length gt 0; take over for other passages
					\multicolumn{1}{X}{ keine Aufnahme Studium   } &


					%265 &
					  \num{265} &
					%--
					  \num[round-mode=places,round-precision=2]{4,51} &
					    \num[round-mode=places,round-precision=2]{0,94} \\
							%????

					3 &
				% TODO try size/length gt 0; take over for other passages
					\multicolumn{1}{X}{ Wechsel an HS ohne Studiengebühren   } &


					%741 &
					  \num{741} &
					%--
					  \num[round-mode=places,round-precision=2]{12,62} &
					    \num[round-mode=places,round-precision=2]{2,63} \\
							%????

					4 &
				% TODO try size/length gt 0; take over for other passages
					\multicolumn{1}{X}{ keine Studiengebühren an HS geplant   } &


					%847 &
					  \num{847} &
					%--
					  \num[round-mode=places,round-precision=2]{14,42} &
					    \num[round-mode=places,round-precision=2]{3,01} \\
							%????

					5 &
				% TODO try size/length gt 0; take over for other passages
					\multicolumn{1}{X}{ Arbeitgeber zahlt Studiengebühren   } &


					%266 &
					  \num{266} &
					%--
					  \num[round-mode=places,round-precision=2]{4,53} &
					    \num[round-mode=places,round-precision=2]{0,94} \\
							%????

					6 &
				% TODO try size/length gt 0; take over for other passages
					\multicolumn{1}{X}{ bin/werde befreit   } &


					%229 &
					  \num{229} &
					%--
					  \num[round-mode=places,round-precision=2]{3,9} &
					    \num[round-mode=places,round-precision=2]{0,81} \\
							%????

					7 &
				% TODO try size/length gt 0; take over for other passages
					\multicolumn{1}{X}{ durch Studiengebühren bessere Ausbildung   } &


					%168 &
					  \num{168} &
					%--
					  \num[round-mode=places,round-precision=2]{2,86} &
					    \num[round-mode=places,round-precision=2]{0,6} \\
							%????

					8 &
				% TODO try size/length gt 0; take over for other passages
					\multicolumn{1}{X}{ Studium unabhängig von Gebühren aufnehmen bzw. fortsetzen   } &


					%2845 &
					  \num{2845} &
					%--
					  \num[round-mode=places,round-precision=2]{48,44} &
					    \num[round-mode=places,round-precision=2]{10,1} \\
							%????

					9 &
				% TODO try size/length gt 0; take over for other passages
					\multicolumn{1}{X}{ anderes   } &


					%271 &
					  \num{271} &
					%--
					  \num[round-mode=places,round-precision=2]{4,61} &
					    \num[round-mode=places,round-precision=2]{0,96} \\
							%????
						%DIFFERENT OBSERVATIONS >20
					\midrule
					\multicolumn{2}{l}{Summe (gültig)} &
					  \textbf{\num{5873}} &
					\textbf{100} &
					  \textbf{\num[round-mode=places,round-precision=2]{20,84}} \\
					%--
					\multicolumn{5}{l}{\textbf{Fehlende Werte}}\\
							-998 &
							keine Angabe &
							  \num{60} &
							 - &
							  \num[round-mode=places,round-precision=2]{0,21} \\
							-995 &
							keine Teilnahme (Panel) &
							  \num{22249} &
							 - &
							  \num[round-mode=places,round-precision=2]{78,95} \\
					\midrule
					\multicolumn{2}{l}{\textbf{Summe (gesamt)}} &
				      \textbf{\num{28182}} &
				    \textbf{-} &
				    \textbf{100} \\
					\bottomrule
					\end{longtable}
					\end{filecontents}
					\LTXtable{\textwidth}{\jobname-bfee01}
				\label{tableValues:bfee01}
				\vspace*{-\baselineskip}
                    \begin{noten}
                	    \note{} Deskritive Maßzahlen:
                	    Anzahl unterschiedlicher Beobachtungen: 9%
                	    ; 
                	      Modus ($h$): 8
                     \end{noten}


