%EVERY VARIABLE HAS IT'S OWN PAGE

    \setcounter{footnote}{0}

    %omit vertical space
    \vspace*{-1.8cm}
	\section{bsch14a\_g2 (1. Prüfungsfach (Fächergruppe))}
	\label{section:bsch14a_g2}



	%TABLE FOR VARIABLE DETAILS
    \vspace*{0.5cm}
    \noindent\textbf{Eigenschaften
	% '#' has to be escaped
	\footnote{Detailliertere Informationen zur Variable finden sich unter
		\url{https://metadata.fdz.dzhw.eu/\#!/de/variables/var-gsl2008-ds1-bsch14a_g2$}}}\\
	\begin{tabularx}{\hsize}{@{}lX}
	Datentyp: & numerisch \\
	Skalenniveau: & nominal \\
	Zugangswege: &
	  remote-desktop-suf, 
	  onsite-suf
 \\
    \end{tabularx}



    %TABLE FOR QUESTION DETAILS
    %This has to be tested and has to be improved
    %rausfinden, ob einer Variable mehrere Fragen zugeordnet werden
    %dann evtl. nur die erste verwenden oder etwas anderes tun (Hinweis mehrere Fragen, auflisten mit Link)
				%TABLE FOR QUESTION DETAILS
				\vspace*{0.5cm}
                \noindent\textbf{Frage
	                \footnote{Detailliertere Informationen zur Frage finden sich unter
		              \url{https://metadata.fdz.dzhw.eu/\#!/de/questions/que-gsl2008-ins2-04$}}}\\
				\begin{tabularx}{\hsize}{@{}lX}
					Fragenummer: &
					  Fragebogen des DZHW-Studienberechtigtenpanels 2008 - zweite Welle:
					  04
 \\
					%--
					Fragetext: & Nennen Sie bitte Ihre Prüfungsfächer und geben Sie zusätzlich an, mit welcher wöchentlichen Stundenzahl diese in Ihrem Abschlussjahr unterrichtet wurden. \\
				\end{tabularx}





				%TABLE FOR THE NOMINAL / ORDINAL VALUES
        		\vspace*{0.5cm}
                \noindent\textbf{Häufigkeiten}

                \vspace*{-\baselineskip}
					%NUMERIC ELEMENTS NEED A HUGH SECOND COLOUMN AND A SMALL FIRST ONE
					\begin{filecontents}{\jobname-bsch14a_g2}
					\begin{longtable}{lXrrr}
					\toprule
					\textbf{Wert} & \textbf{Label} & \textbf{Häufigkeit} & \textbf{Prozent(gültig)} & \textbf{Prozent} \\
					\endhead
					\midrule
					\multicolumn{5}{l}{\textbf{Gültige Werte}}\\
						%DIFFERENT OBSERVATIONS <=20

					1 &
				% TODO try size/length gt 0; take over for other passages
					\multicolumn{1}{X}{ Mathematik/Naturwissenschaften/Technik (Allgemeinbildene Schule)   } &


					%2334 &
					  \num{2334} &
					%--
					  \num[round-mode=places,round-precision=2]{39,82} &
					    \num[round-mode=places,round-precision=2]{8,28} \\
							%????

					2 &
				% TODO try size/length gt 0; take over for other passages
					\multicolumn{1}{X}{ Kulturwissenschaften/Soziologie/Wirtschaft (Allgemeinbildene Schule)   } &


					%307 &
					  \num{307} &
					%--
					  \num[round-mode=places,round-precision=2]{5,24} &
					    \num[round-mode=places,round-precision=2]{1,09} \\
							%????

					3 &
				% TODO try size/length gt 0; take over for other passages
					\multicolumn{1}{X}{ Kunst/Musik/Sport (Allgemeinbildene Schule)   } &


					%123 &
					  \num{123} &
					%--
					  \num[round-mode=places,round-precision=2]{2,1} &
					    \num[round-mode=places,round-precision=2]{0,44} \\
							%????

					5 &
				% TODO try size/length gt 0; take over for other passages
					\multicolumn{1}{X}{ Sprachen/Sprachwissenschaften (Allgemeinbildene Schule)   } &


					%2322 &
					  \num{2322} &
					%--
					  \num[round-mode=places,round-precision=2]{39,61} &
					    \num[round-mode=places,round-precision=2]{8,24} \\
							%????

					6 &
				% TODO try size/length gt 0; take over for other passages
					\multicolumn{1}{X}{ Technik (berufsbildende Schule)   } &


					%139 &
					  \num{139} &
					%--
					  \num[round-mode=places,round-precision=2]{2,37} &
					    \num[round-mode=places,round-precision=2]{0,49} \\
							%????

					7 &
				% TODO try size/length gt 0; take over for other passages
					\multicolumn{1}{X}{ Bauwesen (berufsbildende Schule)   } &


					%12 &
					  \num{12} &
					%--
					  \num[round-mode=places,round-precision=2]{0,2} &
					    \num[round-mode=places,round-precision=2]{0,04} \\
							%????

					8 &
				% TODO try size/length gt 0; take over for other passages
					\multicolumn{1}{X}{ Informatik (berufsbildende Schule)   } &


					%53 &
					  \num{53} &
					%--
					  \num[round-mode=places,round-precision=2]{0,9} &
					    \num[round-mode=places,round-precision=2]{0,19} \\
							%????

					9 &
				% TODO try size/length gt 0; take over for other passages
					\multicolumn{1}{X}{ Wirtschaft (und Verwaltung) (berufsbildende Schule)   } &


					%322 &
					  \num{322} &
					%--
					  \num[round-mode=places,round-precision=2]{5,49} &
					    \num[round-mode=places,round-precision=2]{1,14} \\
							%????

					10 &
				% TODO try size/length gt 0; take over for other passages
					\multicolumn{1}{X}{ Landwirtschaft/ Naturwissenschaften (berufsbildende Schule)   } &


					%15 &
					  \num{15} &
					%--
					  \num[round-mode=places,round-precision=2]{0,26} &
					    \num[round-mode=places,round-precision=2]{0,05} \\
							%????

					11 &
				% TODO try size/length gt 0; take over for other passages
					\multicolumn{1}{X}{ Gestaltung (berufsbildende Schule)   } &


					%31 &
					  \num{31} &
					%--
					  \num[round-mode=places,round-precision=2]{0,53} &
					    \num[round-mode=places,round-precision=2]{0,11} \\
							%????

					12 &
				% TODO try size/length gt 0; take over for other passages
					\multicolumn{1}{X}{ Ernährung/Hauswirtschaft/Gesundheit (berufsbildende Schule)   } &


					%36 &
					  \num{36} &
					%--
					  \num[round-mode=places,round-precision=2]{0,61} &
					    \num[round-mode=places,round-precision=2]{0,13} \\
							%????

					13 &
				% TODO try size/length gt 0; take over for other passages
					\multicolumn{1}{X}{ Sozialwesen (berufsbildende Schule)   } &


					%126 &
					  \num{126} &
					%--
					  \num[round-mode=places,round-precision=2]{2,15} &
					    \num[round-mode=places,round-precision=2]{0,45} \\
							%????

					14 &
				% TODO try size/length gt 0; take over for other passages
					\multicolumn{1}{X}{ Sprachen (berufsbildende Schule)   } &


					%41 &
					  \num{41} &
					%--
					  \num[round-mode=places,round-precision=2]{0,7} &
					    \num[round-mode=places,round-precision=2]{0,15} \\
							%????

					15 &
				% TODO try size/length gt 0; take over for other passages
					\multicolumn{1}{X}{ Medien (und Kommunikation) (berufsbildende Schule)   } &


					%1 &
					  \num{1} &
					%--
					  \num[round-mode=places,round-precision=2]{0,02} &
					    \num[round-mode=places,round-precision=2]{0} \\
							%????
						%DIFFERENT OBSERVATIONS >20
					\midrule
					\multicolumn{2}{l}{Summe (gültig)} &
					  \textbf{\num{5862}} &
					\textbf{100} &
					  \textbf{\num[round-mode=places,round-precision=2]{20,8}} \\
					%--
					\multicolumn{5}{l}{\textbf{Fehlende Werte}}\\
							-998 &
							keine Angabe &
							  \num{71} &
							 - &
							  \num[round-mode=places,round-precision=2]{0,25} \\
							-995 &
							keine Teilnahme (Panel) &
							  \num{22249} &
							 - &
							  \num[round-mode=places,round-precision=2]{78,95} \\
					\midrule
					\multicolumn{2}{l}{\textbf{Summe (gesamt)}} &
				      \textbf{\num{28182}} &
				    \textbf{-} &
				    \textbf{100} \\
					\bottomrule
					\end{longtable}
					\end{filecontents}
					\LTXtable{\textwidth}{\jobname-bsch14a_g2}
				\label{tableValues:bsch14a_g2}
				\vspace*{-\baselineskip}
                    \begin{noten}
                	    \note{} Deskritive Maßzahlen:
                	    Anzahl unterschiedlicher Beobachtungen: 14%
                	    ; 
                	      Modus ($h$): 1
                     \end{noten}


