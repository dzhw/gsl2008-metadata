%EVERY VARIABLE HAS IT'S OWN PAGE

    \setcounter{footnote}{0}

    %omit vertical space
    \vspace*{-1.8cm}
	\section{cvoc14 (Einschätzung persönliche Ausbildungsleistung)}
	\label{section:cvoc14}



	%TABLE FOR VARIABLE DETAILS
    \vspace*{0.5cm}
    \noindent\textbf{Eigenschaften
	% '#' has to be escaped
	\footnote{Detailliertere Informationen zur Variable finden sich unter
		\url{https://metadata.fdz.dzhw.eu/\#!/de/variables/var-gsl2008-ds1-cvoc14$}}}\\
	\begin{tabularx}{\hsize}{@{}lX}
	Datentyp: & numerisch \\
	Skalenniveau: & ordinal \\
	Zugangswege: &
	  remote-desktop-suf, 
	  onsite-suf
 \\
    \end{tabularx}



    %TABLE FOR QUESTION DETAILS
    %This has to be tested and has to be improved
    %rausfinden, ob einer Variable mehrere Fragen zugeordnet werden
    %dann evtl. nur die erste verwenden oder etwas anderes tun (Hinweis mehrere Fragen, auflisten mit Link)
				%TABLE FOR QUESTION DETAILS
				\vspace*{0.5cm}
                \noindent\textbf{Frage
	                \footnote{Detailliertere Informationen zur Frage finden sich unter
		              \url{https://metadata.fdz.dzhw.eu/\#!/de/questions/que-gsl2008-ins3-3.6$}}}\\
				\begin{tabularx}{\hsize}{@{}lX}
					Fragenummer: &
					  Fragebogen des DZHW-Studienberechtigtenpanels 2008 - dritte Welle:
					  3.6
 \\
					%--
					Fragetext: & Wie schätzen Sie persönlich Ihre Ausbildungsleistungen im Vergleich zu anderen ein? \\
				\end{tabularx}





				%TABLE FOR THE NOMINAL / ORDINAL VALUES
        		\vspace*{0.5cm}
                \noindent\textbf{Häufigkeiten}

                \vspace*{-\baselineskip}
					%NUMERIC ELEMENTS NEED A HUGH SECOND COLOUMN AND A SMALL FIRST ONE
					\begin{filecontents}{\jobname-cvoc14}
					\begin{longtable}{lXrrr}
					\toprule
					\textbf{Wert} & \textbf{Label} & \textbf{Häufigkeit} & \textbf{Prozent(gültig)} & \textbf{Prozent} \\
					\endhead
					\midrule
					\multicolumn{5}{l}{\textbf{Gültige Werte}}\\
						%DIFFERENT OBSERVATIONS <=20

					-5 &
				% TODO try size/length gt 0; take over for other passages
					\multicolumn{1}{X}{ unterdurchschnittlich   } &


					%2 &
					  \num{2} &
					%--
					  \num[round-mode=places,round-precision=2]{0,21} &
					    \num[round-mode=places,round-precision=2]{0,01} \\
							%????

					-4 &
				% TODO try size/length gt 0; take over for other passages
					\multicolumn{1}{X}{ -4   } &


					%3 &
					  \num{3} &
					%--
					  \num[round-mode=places,round-precision=2]{0,32} &
					    \num[round-mode=places,round-precision=2]{0,01} \\
							%????

					-3 &
				% TODO try size/length gt 0; take over for other passages
					\multicolumn{1}{X}{ -3   } &


					%2 &
					  \num{2} &
					%--
					  \num[round-mode=places,round-precision=2]{0,21} &
					    \num[round-mode=places,round-precision=2]{0,01} \\
							%????

					-2 &
				% TODO try size/length gt 0; take over for other passages
					\multicolumn{1}{X}{ -2   } &


					%9 &
					  \num{9} &
					%--
					  \num[round-mode=places,round-precision=2]{0,96} &
					    \num[round-mode=places,round-precision=2]{0,03} \\
							%????

					-1 &
				% TODO try size/length gt 0; take over for other passages
					\multicolumn{1}{X}{ -1   } &


					%5 &
					  \num{5} &
					%--
					  \num[round-mode=places,round-precision=2]{0,53} &
					    \num[round-mode=places,round-precision=2]{0,02} \\
							%????

					0 &
				% TODO try size/length gt 0; take over for other passages
					\multicolumn{1}{X}{ durchschnittlich   } &


					%84 &
					  \num{84} &
					%--
					  \num[round-mode=places,round-precision=2]{8,97} &
					    \num[round-mode=places,round-precision=2]{0,3} \\
							%????

					1 &
				% TODO try size/length gt 0; take over for other passages
					\multicolumn{1}{X}{ 1   } &


					%79 &
					  \num{79} &
					%--
					  \num[round-mode=places,round-precision=2]{8,44} &
					    \num[round-mode=places,round-precision=2]{0,28} \\
							%????

					2 &
				% TODO try size/length gt 0; take over for other passages
					\multicolumn{1}{X}{ 2   } &


					%182 &
					  \num{182} &
					%--
					  \num[round-mode=places,round-precision=2]{19,44} &
					    \num[round-mode=places,round-precision=2]{0,65} \\
							%????

					3 &
				% TODO try size/length gt 0; take over for other passages
					\multicolumn{1}{X}{ 3   } &


					%292 &
					  \num{292} &
					%--
					  \num[round-mode=places,round-precision=2]{31,2} &
					    \num[round-mode=places,round-precision=2]{1,04} \\
							%????

					4 &
				% TODO try size/length gt 0; take over for other passages
					\multicolumn{1}{X}{ 4   } &


					%181 &
					  \num{181} &
					%--
					  \num[round-mode=places,round-precision=2]{19,34} &
					    \num[round-mode=places,round-precision=2]{0,64} \\
							%????

					5 &
				% TODO try size/length gt 0; take over for other passages
					\multicolumn{1}{X}{ überdurchschnittlich   } &


					%97 &
					  \num{97} &
					%--
					  \num[round-mode=places,round-precision=2]{10,36} &
					    \num[round-mode=places,round-precision=2]{0,34} \\
							%????
						%DIFFERENT OBSERVATIONS >20
					\midrule
					\multicolumn{2}{l}{Summe (gültig)} &
					  \textbf{\num{936}} &
					\textbf{100} &
					  \textbf{\num[round-mode=places,round-precision=2]{3,32}} \\
					%--
					\multicolumn{5}{l}{\textbf{Fehlende Werte}}\\
							-998 &
							keine Angabe &
							  \num{51} &
							 - &
							  \num[round-mode=places,round-precision=2]{0,18} \\
							-995 &
							keine Teilnahme (Panel) &
							  \num{24511} &
							 - &
							  \num[round-mode=places,round-precision=2]{86,97} \\
							-989 &
							filterbedingt fehlend &
							  \num{2684} &
							 - &
							  \num[round-mode=places,round-precision=2]{9,52} \\
					\midrule
					\multicolumn{2}{l}{\textbf{Summe (gesamt)}} &
				      \textbf{\num{28182}} &
				    \textbf{-} &
				    \textbf{100} \\
					\bottomrule
					\end{longtable}
					\end{filecontents}
					\LTXtable{\textwidth}{\jobname-cvoc14}
				\label{tableValues:cvoc14}
				\vspace*{-\baselineskip}
                    \begin{noten}
                	    \note{} Deskritive Maßzahlen:
                	    Anzahl unterschiedlicher Beobachtungen: 11%
                	    ; 
                	      Minimum ($min$): -5; 
                	      Maximum ($max$): 5; 
                	      Median ($\tilde{x}$): 3; 
                	      Modus ($h$): 3
                     \end{noten}


