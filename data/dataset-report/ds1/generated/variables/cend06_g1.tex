%EVERY VARIABLE HAS IT'S OWN PAGE

    \setcounter{footnote}{0}

    %omit vertical space
    \vspace*{-1.8cm}
	\section{cend06\_g1 (6. Tätigkeit: Ende (Jahr))}
	\label{section:cend06_g1}



	%TABLE FOR VARIABLE DETAILS
    \vspace*{0.5cm}
    \noindent\textbf{Eigenschaften
	% '#' has to be escaped
	\footnote{Detailliertere Informationen zur Variable finden sich unter
		\url{https://metadata.fdz.dzhw.eu/\#!/de/variables/var-gsl2008-ds1-cend06_g1$}}}\\
	\begin{tabularx}{\hsize}{@{}lX}
	Datentyp: & numerisch \\
	Skalenniveau: & intervall \\
	Zugangswege: &
	  remote-desktop-suf, 
	  onsite-suf
 \\
    \end{tabularx}



    %TABLE FOR QUESTION DETAILS
    %This has to be tested and has to be improved
    %rausfinden, ob einer Variable mehrere Fragen zugeordnet werden
    %dann evtl. nur die erste verwenden oder etwas anderes tun (Hinweis mehrere Fragen, auflisten mit Link)
				%TABLE FOR QUESTION DETAILS
				\vspace*{0.5cm}
                \noindent\textbf{Frage
	                \footnote{Detailliertere Informationen zur Frage finden sich unter
		              \url{https://metadata.fdz.dzhw.eu/\#!/de/questions/que-gsl2008-ins3-2.1$}}}\\
				\begin{tabularx}{\hsize}{@{}lX}
					Fragenummer: &
					  Fragebogen des DZHW-Studienberechtigtenpanels 2008 - dritte Welle:
					  2.1
 \\
					%--
					Fragetext: & Wir bitten Sie nun, uns in dem folgenden Schema einen Überblick Ihres Werdegangs von Juli 2008 bis Dezember 2012 zu geben.\par  Zeitraum\par  Ende\par  Monat/Jahr \\
				\end{tabularx}





				%TABLE FOR THE NOMINAL / ORDINAL VALUES
        		\vspace*{0.5cm}
                \noindent\textbf{Häufigkeiten}

                \vspace*{-\baselineskip}
					%NUMERIC ELEMENTS NEED A HUGH SECOND COLOUMN AND A SMALL FIRST ONE
					\begin{filecontents}{\jobname-cend06_g1}
					\begin{longtable}{lXrrr}
					\toprule
					\textbf{Wert} & \textbf{Label} & \textbf{Häufigkeit} & \textbf{Prozent(gültig)} & \textbf{Prozent} \\
					\endhead
					\midrule
					\multicolumn{5}{l}{\textbf{Gültige Werte}}\\
						%DIFFERENT OBSERVATIONS <=20

					2009 &
				% TODO try size/length gt 0; take over for other passages
					\multicolumn{1}{X}{ -  } &


					%18 &
					  \num{18} &
					%--
					  \num[round-mode=places,round-precision=2]{1,93} &
					    \num[round-mode=places,round-precision=2]{0,06} \\
							%????

					2010 &
				% TODO try size/length gt 0; take over for other passages
					\multicolumn{1}{X}{ -  } &


					%72 &
					  \num{72} &
					%--
					  \num[round-mode=places,round-precision=2]{7,73} &
					    \num[round-mode=places,round-precision=2]{0,26} \\
							%????

					2011 &
				% TODO try size/length gt 0; take over for other passages
					\multicolumn{1}{X}{ -  } &


					%165 &
					  \num{165} &
					%--
					  \num[round-mode=places,round-precision=2]{17,72} &
					    \num[round-mode=places,round-precision=2]{0,59} \\
							%????

					2012 &
				% TODO try size/length gt 0; take over for other passages
					\multicolumn{1}{X}{ -  } &


					%676 &
					  \num{676} &
					%--
					  \num[round-mode=places,round-precision=2]{72,61} &
					    \num[round-mode=places,round-precision=2]{2,4} \\
							%????
						%DIFFERENT OBSERVATIONS >20
					\midrule
					\multicolumn{2}{l}{Summe (gültig)} &
					  \textbf{\num{931}} &
					\textbf{100} &
					  \textbf{\num[round-mode=places,round-precision=2]{3,3}} \\
					%--
					\multicolumn{5}{l}{\textbf{Fehlende Werte}}\\
							-998 &
							keine Angabe &
							  \num{2740} &
							 - &
							  \num[round-mode=places,round-precision=2]{9,72} \\
							-995 &
							keine Teilnahme (Panel) &
							  \num{24511} &
							 - &
							  \num[round-mode=places,round-precision=2]{86,97} \\
					\midrule
					\multicolumn{2}{l}{\textbf{Summe (gesamt)}} &
				      \textbf{\num{28182}} &
				    \textbf{-} &
				    \textbf{100} \\
					\bottomrule
					\end{longtable}
					\end{filecontents}
					\LTXtable{\textwidth}{\jobname-cend06_g1}
				\label{tableValues:cend06_g1}
				\vspace*{-\baselineskip}
                    \begin{noten}
                	    \note{} Deskritive Maßzahlen:
                	    Anzahl unterschiedlicher Beobachtungen: 4%
                	    ; 
                	      Minimum ($min$): 2009; 
                	      Maximum ($max$): 2012; 
                	      arithmetisches Mittel ($\bar{x}$): \num[round-mode=places,round-precision=2]{2011,6101}; 
                	      Median ($\tilde{x}$): 2012; 
                	      Modus ($h$): 2012; 
                	      Standardabweichung ($s$): \num[round-mode=places,round-precision=2]{0,7135}; 
                	      Schiefe ($v$): \num[round-mode=places,round-precision=2]{-1,8303}; 
                	      Wölbung ($w$): \num[round-mode=places,round-precision=2]{5,639}
                     \end{noten}


