%EVERY VARIABLE HAS IT'S OWN PAGE

    \setcounter{footnote}{0}

    %omit vertical space
    \vspace*{-1.8cm}
	\section{astu03c\_g2 (unsichere Präferenz: 1. Hauptstudienfach (Fächergruppe))}
	\label{section:astu03c_g2}



	%TABLE FOR VARIABLE DETAILS
    \vspace*{0.5cm}
    \noindent\textbf{Eigenschaften
	% '#' has to be escaped
	\footnote{Detailliertere Informationen zur Variable finden sich unter
		\url{https://metadata.fdz.dzhw.eu/\#!/de/variables/var-gsl2008-ds1-astu03c_g2$}}}\\
	\begin{tabularx}{\hsize}{@{}lX}
	Datentyp: & numerisch \\
	Skalenniveau: & nominal \\
	Zugangswege: &
	  remote-desktop-suf, 
	  onsite-suf
 \\
    \end{tabularx}



    %TABLE FOR QUESTION DETAILS
    %This has to be tested and has to be improved
    %rausfinden, ob einer Variable mehrere Fragen zugeordnet werden
    %dann evtl. nur die erste verwenden oder etwas anderes tun (Hinweis mehrere Fragen, auflisten mit Link)
				%TABLE FOR QUESTION DETAILS
				\vspace*{0.5cm}
                \noindent\textbf{Frage
	                \footnote{Detailliertere Informationen zur Frage finden sich unter
		              \url{https://metadata.fdz.dzhw.eu/\#!/de/questions/que-gsl2008-ins1-14$}}}\\
				\begin{tabularx}{\hsize}{@{}lX}
					Fragenummer: &
					  Fragebogen des DZHW-Studienberechtigtenpanels 2008 - erste Welle:
					  14
 \\
					%--
					Fragetext: & Welches Studienfach wird dies voraussichtlich sein? \\
				\end{tabularx}





				%TABLE FOR THE NOMINAL / ORDINAL VALUES
        		\vspace*{0.5cm}
                \noindent\textbf{Häufigkeiten}

                \vspace*{-\baselineskip}
					%NUMERIC ELEMENTS NEED A HUGH SECOND COLOUMN AND A SMALL FIRST ONE
					\begin{filecontents}{\jobname-astu03c_g2}
					\begin{longtable}{lXrrr}
					\toprule
					\textbf{Wert} & \textbf{Label} & \textbf{Häufigkeit} & \textbf{Prozent(gültig)} & \textbf{Prozent} \\
					\endhead
					\midrule
					\multicolumn{5}{l}{\textbf{Gültige Werte}}\\
						%DIFFERENT OBSERVATIONS <=20

					1 &
				% TODO try size/length gt 0; take over for other passages
					\multicolumn{1}{X}{ Sprach- und Kulturwissenschaften   } &


					%1373 &
					  \num{1373} &
					%--
					  \num[round-mode=places,round-precision=2]{20,94} &
					    \num[round-mode=places,round-precision=2]{4,87} \\
							%????

					2 &
				% TODO try size/length gt 0; take over for other passages
					\multicolumn{1}{X}{ Sport   } &


					%120 &
					  \num{120} &
					%--
					  \num[round-mode=places,round-precision=2]{1,83} &
					    \num[round-mode=places,round-precision=2]{0,43} \\
							%????

					3 &
				% TODO try size/length gt 0; take over for other passages
					\multicolumn{1}{X}{ Rechts-, Wirtschafts- und Sozialwissenschaften   } &


					%1926 &
					  \num{1926} &
					%--
					  \num[round-mode=places,round-precision=2]{29,38} &
					    \num[round-mode=places,round-precision=2]{6,83} \\
							%????

					4 &
				% TODO try size/length gt 0; take over for other passages
					\multicolumn{1}{X}{ Mathematik, Naturwissenschaften   } &


					%1004 &
					  \num{1004} &
					%--
					  \num[round-mode=places,round-precision=2]{15,31} &
					    \num[round-mode=places,round-precision=2]{3,56} \\
							%????

					5 &
				% TODO try size/length gt 0; take over for other passages
					\multicolumn{1}{X}{ Humanmedizin/Gesundheitswissenschaften   } &


					%433 &
					  \num{433} &
					%--
					  \num[round-mode=places,round-precision=2]{6,6} &
					    \num[round-mode=places,round-precision=2]{1,54} \\
							%????

					6 &
				% TODO try size/length gt 0; take over for other passages
					\multicolumn{1}{X}{ Veterinärmedizin   } &


					%53 &
					  \num{53} &
					%--
					  \num[round-mode=places,round-precision=2]{0,81} &
					    \num[round-mode=places,round-precision=2]{0,19} \\
							%????

					7 &
				% TODO try size/length gt 0; take over for other passages
					\multicolumn{1}{X}{ Agrar-, Forst- und Ernährungswissenschaften   } &


					%115 &
					  \num{115} &
					%--
					  \num[round-mode=places,round-precision=2]{1,75} &
					    \num[round-mode=places,round-precision=2]{0,41} \\
							%????

					8 &
				% TODO try size/length gt 0; take over for other passages
					\multicolumn{1}{X}{ Ingenieurwissenschaften   } &


					%1053 &
					  \num{1053} &
					%--
					  \num[round-mode=places,round-precision=2]{16,06} &
					    \num[round-mode=places,round-precision=2]{3,74} \\
							%????

					9 &
				% TODO try size/length gt 0; take over for other passages
					\multicolumn{1}{X}{ Kunst, Kunstwissenschaft   } &


					%476 &
					  \num{476} &
					%--
					  \num[round-mode=places,round-precision=2]{7,26} &
					    \num[round-mode=places,round-precision=2]{1,69} \\
							%????

					10 &
				% TODO try size/length gt 0; take over for other passages
					\multicolumn{1}{X}{ Außerhalb der Studienbereichsgliederung   } &


					%3 &
					  \num{3} &
					%--
					  \num[round-mode=places,round-precision=2]{0,05} &
					    \num[round-mode=places,round-precision=2]{0,01} \\
							%????
						%DIFFERENT OBSERVATIONS >20
					\midrule
					\multicolumn{2}{l}{Summe (gültig)} &
					  \textbf{\num{6556}} &
					\textbf{100} &
					  \textbf{\num[round-mode=places,round-precision=2]{23,26}} \\
					%--
					\multicolumn{5}{l}{\textbf{Fehlende Werte}}\\
							-999 &
							weiß nicht &
							  \num{1} &
							 - &
							  \num[round-mode=places,round-precision=2]{0} \\
							-989 &
							filterbedingt fehlend &
							  \num{5803} &
							 - &
							  \num[round-mode=places,round-precision=2]{20,59} \\
							-969 &
							unbekannter fehlender Wert &
							  \num{15822} &
							 - &
							  \num[round-mode=places,round-precision=2]{56,14} \\
					\midrule
					\multicolumn{2}{l}{\textbf{Summe (gesamt)}} &
				      \textbf{\num{28182}} &
				    \textbf{-} &
				    \textbf{100} \\
					\bottomrule
					\end{longtable}
					\end{filecontents}
					\LTXtable{\textwidth}{\jobname-astu03c_g2}
				\label{tableValues:astu03c_g2}
				\vspace*{-\baselineskip}
                    \begin{noten}
                	    \note{} Deskritive Maßzahlen:
                	    Anzahl unterschiedlicher Beobachtungen: 10%
                	    ; 
                	      Modus ($h$): 3
                     \end{noten}


