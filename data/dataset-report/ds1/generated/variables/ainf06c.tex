%EVERY VARIABLE HAS IT'S OWN PAGE

    \setcounter{footnote}{0}

    %omit vertical space
    \vspace*{-1.8cm}
	\section{ainf06c (Informationsquelle Zukunft: persönliche Beratung)}
	\label{section:ainf06c}



	%TABLE FOR VARIABLE DETAILS
    \vspace*{0.5cm}
    \noindent\textbf{Eigenschaften
	% '#' has to be escaped
	\footnote{Detailliertere Informationen zur Variable finden sich unter
		\url{https://metadata.fdz.dzhw.eu/\#!/de/variables/var-gsl2008-ds1-ainf06c$}}}\\
	\begin{tabularx}{\hsize}{@{}lX}
	Datentyp: & numerisch \\
	Skalenniveau: & ordinal \\
	Zugangswege: &
	  remote-desktop-suf, 
	  onsite-suf
 \\
    \end{tabularx}



    %TABLE FOR QUESTION DETAILS
    %This has to be tested and has to be improved
    %rausfinden, ob einer Variable mehrere Fragen zugeordnet werden
    %dann evtl. nur die erste verwenden oder etwas anderes tun (Hinweis mehrere Fragen, auflisten mit Link)
				%TABLE FOR QUESTION DETAILS
				\vspace*{0.5cm}
                \noindent\textbf{Frage
	                \footnote{Detailliertere Informationen zur Frage finden sich unter
		              \url{https://metadata.fdz.dzhw.eu/\#!/de/questions/que-gsl2008-ins1-25$}}}\\
				\begin{tabularx}{\hsize}{@{}lX}
					Fragenummer: &
					  Fragebogen des DZHW-Studienberechtigtenpanels 2008 - erste Welle:
					  25
 \\
					%--
					Fragetext: & Schätzen Sie bitte ein, in welchem Maße Sie aus den folgenden Quellen wichtige Informationen für Ihre Studien-/Ausbildungsplanung erhalten haben und wie hilfreich diese waren.\par  Informationsangebote\par  persönliche Beratung \\
				\end{tabularx}





				%TABLE FOR THE NOMINAL / ORDINAL VALUES
        		\vspace*{0.5cm}
                \noindent\textbf{Häufigkeiten}

                \vspace*{-\baselineskip}
					%NUMERIC ELEMENTS NEED A HUGH SECOND COLOUMN AND A SMALL FIRST ONE
					\begin{filecontents}{\jobname-ainf06c}
					\begin{longtable}{lXrrr}
					\toprule
					\textbf{Wert} & \textbf{Label} & \textbf{Häufigkeit} & \textbf{Prozent(gültig)} & \textbf{Prozent} \\
					\endhead
					\midrule
					\multicolumn{5}{l}{\textbf{Gültige Werte}}\\
						%DIFFERENT OBSERVATIONS <=20

					1 &
				% TODO try size/length gt 0; take over for other passages
					\multicolumn{1}{X}{ sehr hilfreich   } &


					%2602 &
					  \num{2602} &
					%--
					  \num[round-mode=places,round-precision=2]{17,78} &
					    \num[round-mode=places,round-precision=2]{9,23} \\
							%????

					2 &
				% TODO try size/length gt 0; take over for other passages
					\multicolumn{1}{X}{ 2   } &


					%4994 &
					  \num{4994} &
					%--
					  \num[round-mode=places,round-precision=2]{34,13} &
					    \num[round-mode=places,round-precision=2]{17,72} \\
							%????

					3 &
				% TODO try size/length gt 0; take over for other passages
					\multicolumn{1}{X}{ 3   } &


					%4105 &
					  \num{4105} &
					%--
					  \num[round-mode=places,round-precision=2]{28,05} &
					    \num[round-mode=places,round-precision=2]{14,57} \\
							%????

					4 &
				% TODO try size/length gt 0; take over for other passages
					\multicolumn{1}{X}{ 4   } &


					%1831 &
					  \num{1831} &
					%--
					  \num[round-mode=places,round-precision=2]{12,51} &
					    \num[round-mode=places,round-precision=2]{6,5} \\
							%????

					5 &
				% TODO try size/length gt 0; take over for other passages
					\multicolumn{1}{X}{ nicht hilfreich   } &


					%1102 &
					  \num{1102} &
					%--
					  \num[round-mode=places,round-precision=2]{7,53} &
					    \num[round-mode=places,round-precision=2]{3,91} \\
							%????
						%DIFFERENT OBSERVATIONS >20
					\midrule
					\multicolumn{2}{l}{Summe (gültig)} &
					  \textbf{\num{14634}} &
					\textbf{100} &
					  \textbf{\num[round-mode=places,round-precision=2]{51,93}} \\
					%--
					\multicolumn{5}{l}{\textbf{Fehlende Werte}}\\
							-998 &
							keine Angabe &
							  \num{1260} &
							 - &
							  \num[round-mode=places,round-precision=2]{4,47} \\
							-989 &
							filterbedingt fehlend &
							  \num{2966} &
							 - &
							  \num[round-mode=places,round-precision=2]{10,52} \\
							-988 &
							trifft nicht zu &
							  \num{9322} &
							 - &
							  \num[round-mode=places,round-precision=2]{33,08} \\
					\midrule
					\multicolumn{2}{l}{\textbf{Summe (gesamt)}} &
				      \textbf{\num{28182}} &
				    \textbf{-} &
				    \textbf{100} \\
					\bottomrule
					\end{longtable}
					\end{filecontents}
					\LTXtable{\textwidth}{\jobname-ainf06c}
				\label{tableValues:ainf06c}
				\vspace*{-\baselineskip}
                    \begin{noten}
                	    \note{} Deskritive Maßzahlen:
                	    Anzahl unterschiedlicher Beobachtungen: 5%
                	    ; 
                	      Minimum ($min$): 1; 
                	      Maximum ($max$): 5; 
                	      Median ($\tilde{x}$): 2; 
                	      Modus ($h$): 2
                     \end{noten}


