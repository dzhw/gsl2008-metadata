%EVERY VARIABLE HAS IT'S OWN PAGE

    \setcounter{footnote}{0}

    %omit vertical space
    \vspace*{-1.8cm}
	\section{cvoc088\_g1o (8. Tätigkeit: Ausbildungsberuf (KldB-1992-Berufsklasse) (4-Steller))}
	\label{section:cvoc088_g1o}



	%TABLE FOR VARIABLE DETAILS
    \vspace*{0.5cm}
    \noindent\textbf{Eigenschaften
	% '#' has to be escaped
	\footnote{Detailliertere Informationen zur Variable finden sich unter
		\url{https://metadata.fdz.dzhw.eu/\#!/de/variables/var-gsl2008-ds1-cvoc088_g1o$}}}\\
	\begin{tabularx}{\hsize}{@{}lX}
	Datentyp: & numerisch \\
	Skalenniveau: & nominal \\
	Zugangswege: &
	  onsite-suf
 \\
    \end{tabularx}



    %TABLE FOR QUESTION DETAILS
    %This has to be tested and has to be improved
    %rausfinden, ob einer Variable mehrere Fragen zugeordnet werden
    %dann evtl. nur die erste verwenden oder etwas anderes tun (Hinweis mehrere Fragen, auflisten mit Link)
				%TABLE FOR QUESTION DETAILS
				\vspace*{0.5cm}
                \noindent\textbf{Frage
	                \footnote{Detailliertere Informationen zur Frage finden sich unter
		              \url{https://metadata.fdz.dzhw.eu/\#!/de/questions/que-gsl2008-ins3-2.1$}}}\\
				\begin{tabularx}{\hsize}{@{}lX}
					Fragenummer: &
					  Fragebogen des DZHW-Studienberechtigtenpanels 2008 - dritte Welle:
					  2.1
 \\
					%--
					Fragetext: & Wir bitten Sie nun, uns in dem folgenden Schema einen Überblick Ihres Werdegangs von Juli 2008 bis Dezember 2012 zu geben.\par  Berufliche Ausbildung\par  Art der Ausbildung und Ausbildungsberuf\par  (z. B. betriebliche Ausbildung Bürokauffrau/mann) \\
				\end{tabularx}





				%TABLE FOR THE NOMINAL / ORDINAL VALUES
        		\vspace*{0.5cm}
                \noindent\textbf{Häufigkeiten}

                \vspace*{-\baselineskip}
					%NUMERIC ELEMENTS NEED A HUGH SECOND COLOUMN AND A SMALL FIRST ONE
					\begin{filecontents}{\jobname-cvoc088_g1o}
					\begin{longtable}{lXrrr}
					\toprule
					\textbf{Wert} & \textbf{Label} & \textbf{Häufigkeit} & \textbf{Prozent(gültig)} & \textbf{Prozent} \\
					\endhead
					\midrule
					\multicolumn{5}{l}{\textbf{Gültige Werte}}\\
						%DIFFERENT OBSERVATIONS <=20

					621 &
				% TODO try size/length gt 0; take over for other passages
					\multicolumn{1}{X}{ Forstwirt(e/innen) (Waldfacharbeiter/innen), allgemein   } &


					%1 &
					  \num{1} &
					%--
					  \num[round-mode=places,round-precision=2]{12,5} &
					    \num[round-mode=places,round-precision=2]{0} \\
							%????

					6763 &
				% TODO try size/length gt 0; take over for other passages
					\multicolumn{1}{X}{ Filialleiter/innen (Verkaufsstellenleiter/innen)   } &


					%1 &
					  \num{1} &
					%--
					  \num[round-mode=places,round-precision=2]{12,5} &
					    \num[round-mode=places,round-precision=2]{0} \\
							%????

					7851 &
				% TODO try size/length gt 0; take over for other passages
					\multicolumn{1}{X}{ Industriekaufleute   } &


					%1 &
					  \num{1} &
					%--
					  \num[round-mode=places,round-precision=2]{12,5} &
					    \num[round-mode=places,round-precision=2]{0} \\
							%????

					8536 &
				% TODO try size/length gt 0; take over for other passages
					\multicolumn{1}{X}{ Hebammen/Entbindungspfleger   } &


					%1 &
					  \num{1} &
					%--
					  \num[round-mode=places,round-precision=2]{12,5} &
					    \num[round-mode=places,round-precision=2]{0} \\
							%????

					8571 &
				% TODO try size/length gt 0; take over for other passages
					\multicolumn{1}{X}{ Medizinisch-technische Laboratoriumsassistent(en/innen), Medizinische Laboranten(en/innen)   } &


					%1 &
					  \num{1} &
					%--
					  \num[round-mode=places,round-precision=2]{12,5} &
					    \num[round-mode=places,round-precision=2]{0} \\
							%????

					8594 &
				% TODO try size/length gt 0; take over for other passages
					\multicolumn{1}{X}{ Beschäftigungs- und Arbeitstherapeut(en/innen)   } &


					%1 &
					  \num{1} &
					%--
					  \num[round-mode=places,round-precision=2]{12,5} &
					    \num[round-mode=places,round-precision=2]{0} \\
							%????

					8712 &
				% TODO try size/length gt 0; take over for other passages
					\multicolumn{1}{X}{ Hochschuldozent(en/innen) und -assistent(en/innen)   } &


					%1 &
					  \num{1} &
					%--
					  \num[round-mode=places,round-precision=2]{12,5} &
					    \num[round-mode=places,round-precision=2]{0} \\
							%????

					9020 &
				% TODO try size/length gt 0; take over for other passages
					\multicolumn{1}{X}{ Kosmetiker/innen, allgemein   } &


					%1 &
					  \num{1} &
					%--
					  \num[round-mode=places,round-precision=2]{12,5} &
					    \num[round-mode=places,round-precision=2]{0} \\
							%????
						%DIFFERENT OBSERVATIONS >20
					\midrule
					\multicolumn{2}{l}{Summe (gültig)} &
					  \textbf{\num{8}} &
					\textbf{100} &
					  \textbf{\num[round-mode=places,round-precision=2]{0,03}} \\
					%--
					\multicolumn{5}{l}{\textbf{Fehlende Werte}}\\
							-998 &
							keine Angabe &
							  \num{3663} &
							 - &
							  \num[round-mode=places,round-precision=2]{13} \\
							-995 &
							keine Teilnahme (Panel) &
							  \num{24511} &
							 - &
							  \num[round-mode=places,round-precision=2]{86,97} \\
					\midrule
					\multicolumn{2}{l}{\textbf{Summe (gesamt)}} &
				      \textbf{\num{28182}} &
				    \textbf{-} &
				    \textbf{100} \\
					\bottomrule
					\end{longtable}
					\end{filecontents}
					\LTXtable{\textwidth}{\jobname-cvoc088_g1o}
				\label{tableValues:cvoc088_g1o}
				\vspace*{-\baselineskip}
                    \begin{noten}
                	    \note{} Deskritive Maßzahlen:
                	    Anzahl unterschiedlicher Beobachtungen: 8%
                	    ; 
                	      Modus ($h$): multimodal
                     \end{noten}


