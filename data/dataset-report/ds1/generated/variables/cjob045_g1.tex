%EVERY VARIABLE HAS IT'S OWN PAGE

    \setcounter{footnote}{0}

    %omit vertical space
    \vspace*{-1.8cm}
	\section{cjob045\_g1 (5. Tätigkeit: Beruf Erwerbstätigkeit (KldB-1992-Berufsklasse) (4-Steller))}
	\label{section:cjob045_g1}



	%TABLE FOR VARIABLE DETAILS
    \vspace*{0.5cm}
    \noindent\textbf{Eigenschaften
	% '#' has to be escaped
	\footnote{Detailliertere Informationen zur Variable finden sich unter
		\url{https://metadata.fdz.dzhw.eu/\#!/de/variables/var-gsl2008-ds1-cjob045_g1$}}}\\
	\begin{tabularx}{\hsize}{@{}lX}
	Datentyp: & numerisch \\
	Skalenniveau: & nominal \\
	Zugangswege: &
	  remote-desktop-suf, 
	  onsite-suf
 \\
    \end{tabularx}



    %TABLE FOR QUESTION DETAILS
    %This has to be tested and has to be improved
    %rausfinden, ob einer Variable mehrere Fragen zugeordnet werden
    %dann evtl. nur die erste verwenden oder etwas anderes tun (Hinweis mehrere Fragen, auflisten mit Link)
				%TABLE FOR QUESTION DETAILS
				\vspace*{0.5cm}
                \noindent\textbf{Frage
	                \footnote{Detailliertere Informationen zur Frage finden sich unter
		              \url{https://metadata.fdz.dzhw.eu/\#!/de/questions/que-gsl2008-ins3-2.1$}}}\\
				\begin{tabularx}{\hsize}{@{}lX}
					Fragenummer: &
					  Fragebogen des DZHW-Studienberechtigtenpanels 2008 - dritte Welle:
					  2.1
 \\
					%--
					Fragetext: & Wir bitten Sie nun, uns in dem folgenden Schema einen Überblick Ihres Werdegangs von Juli 2008 bis Dezember 2012 zu geben.\par  Studium\par  Name und Ort der Hochschule/Berufsakademie\par  (z. B. "Uni Köln", "FH Merseburg" oder "BA Mosbach") \\
				\end{tabularx}





				%TABLE FOR THE NOMINAL / ORDINAL VALUES
        		\vspace*{0.5cm}
                \noindent\textbf{Häufigkeiten}

                \vspace*{-\baselineskip}
					%NUMERIC ELEMENTS NEED A HUGH SECOND COLOUMN AND A SMALL FIRST ONE
					\begin{filecontents}{\jobname-cjob045_g1}
					\begin{longtable}{lXrrr}
					\toprule
					\textbf{Wert} & \textbf{Label} & \textbf{Häufigkeit} & \textbf{Prozent(gültig)} & \textbf{Prozent} \\
					\endhead
					\midrule
					\multicolumn{5}{l}{\textbf{Gültige Werte}}\\
						%DIFFERENT OBSERVATIONS <=20
								529 & \multicolumn{1}{X}{andere gartenbautechnische Berufe} & %1 &
								  \num{1} &
								%--
								  \num[round-mode=places,round-precision=2]{0,61} &
								  \num[round-mode=places,round-precision=2]{0} \\
								1011 & \multicolumn{1}{X}{Steinmetz(en/innen) und Steinbildhauer/innen, Restaurator(en/innen)} & %1 &
								  \num{1} &
								%--
								  \num[round-mode=places,round-precision=2]{0,61} &
								  \num[round-mode=places,round-precision=2]{0} \\
								2900 & \multicolumn{1}{X}{Werkzeugmechaniker/innen, Werkzeugmacher/innen o.n.F.} & %1 &
								  \num{1} &
								%--
								  \num[round-mode=places,round-precision=2]{0,61} &
								  \num[round-mode=places,round-precision=2]{0} \\
								3041 & \multicolumn{1}{X}{Augenoptiker/innen} & %1 &
								  \num{1} &
								%--
								  \num[round-mode=places,round-precision=2]{0,61} &
								  \num[round-mode=places,round-precision=2]{0} \\
								3071 & \multicolumn{1}{X}{Orthopädiemechaniker/innen} & %1 &
								  \num{1} &
								%--
								  \num[round-mode=places,round-precision=2]{0,61} &
								  \num[round-mode=places,round-precision=2]{0} \\
								3171 & \multicolumn{1}{X}{Kommunikationselektroniker/innen (Informationstechnik)} & %1 &
								  \num{1} &
								%--
								  \num[round-mode=places,round-precision=2]{0,61} &
								  \num[round-mode=places,round-precision=2]{0} \\
								3172 & \multicolumn{1}{X}{Kommunikationselektroniker/innen (Funktechnik)} & %1 &
								  \num{1} &
								%--
								  \num[round-mode=places,round-precision=2]{0,61} &
								  \num[round-mode=places,round-precision=2]{0} \\
								4110 & \multicolumn{1}{X}{Köch(e/innen), allgemein} & %1 &
								  \num{1} &
								%--
								  \num[round-mode=places,round-precision=2]{0,61} &
								  \num[round-mode=places,round-precision=2]{0} \\
								5010 & \multicolumn{1}{X}{Tischler/innen, allgemein} & %1 &
								  \num{1} &
								%--
								  \num[round-mode=places,round-precision=2]{0,61} &
								  \num[round-mode=places,round-precision=2]{0} \\
								5212 & \multicolumn{1}{X}{Güteprüfer/innen, Fertigungsprüfer/innen} & %1 &
								  \num{1} &
								%--
								  \num[round-mode=places,round-precision=2]{0,61} &
								  \num[round-mode=places,round-precision=2]{0} \\
							... & ... & ... & ... & ... \\
								8721 & \multicolumn{1}{X}{Gymnasiallehrer/innen} & %3 &
								  \num{3} &
								%--
								  \num[round-mode=places,round-precision=2]{1,84} &
								  \num[round-mode=places,round-precision=2]{0,01} \\

								8731 & \multicolumn{1}{X}{Grundschullehrer/innen} & %1 &
								  \num{1} &
								%--
								  \num[round-mode=places,round-precision=2]{0,61} &
								  \num[round-mode=places,round-precision=2]{0} \\

								8739 & \multicolumn{1}{X}{Pädagogische Assistent(en/innen)} & %1 &
								  \num{1} &
								%--
								  \num[round-mode=places,round-precision=2]{0,61} &
								  \num[round-mode=places,round-precision=2]{0} \\

								8781 & \multicolumn{1}{X}{Fahrlehrer/innen} & %1 &
								  \num{1} &
								%--
								  \num[round-mode=places,round-precision=2]{0,61} &
								  \num[round-mode=places,round-precision=2]{0} \\

								8814 & \multicolumn{1}{X}{Diplom-Betriebswirt(e/innen) o.n.A.} & %1 &
								  \num{1} &
								%--
								  \num[round-mode=places,round-precision=2]{0,61} &
								  \num[round-mode=places,round-precision=2]{0} \\

								8815 & \multicolumn{1}{X}{Betriebswirt(e/innen) o.n.A.} & %2 &
								  \num{2} &
								%--
								  \num[round-mode=places,round-precision=2]{1,23} &
								  \num[round-mode=places,round-precision=2]{0,01} \\

								8872 & \multicolumn{1}{X}{Marktforscher/innen} & %1 &
								  \num{1} &
								%--
								  \num[round-mode=places,round-precision=2]{0,61} &
								  \num[round-mode=places,round-precision=2]{0} \\

								9120 & \multicolumn{1}{X}{Restaurantfachleute, Kellner/innen, allgemein} & %1 &
								  \num{1} &
								%--
								  \num[round-mode=places,round-precision=2]{0,61} &
								  \num[round-mode=places,round-precision=2]{0} \\

								9125 & \multicolumn{1}{X}{Flugbegleiter/innen} & %2 &
								  \num{2} &
								%--
								  \num[round-mode=places,round-precision=2]{1,23} &
								  \num[round-mode=places,round-precision=2]{0,01} \\

								9145 & \multicolumn{1}{X}{Hotelempfangsfachleute} & %2 &
								  \num{2} &
								%--
								  \num[round-mode=places,round-precision=2]{1,23} &
								  \num[round-mode=places,round-precision=2]{0,01} \\

					\midrule
					\multicolumn{2}{l}{Summe (gültig)} &
					  \textbf{\num{163}} &
					\textbf{100} &
					  \textbf{\num[round-mode=places,round-precision=2]{0,58}} \\
					%--
					\multicolumn{5}{l}{\textbf{Fehlende Werte}}\\
							-998 &
							keine Angabe &
							  \num{3505} &
							 - &
							  \num[round-mode=places,round-precision=2]{12,44} \\
							-995 &
							keine Teilnahme (Panel) &
							  \num{24511} &
							 - &
							  \num[round-mode=places,round-precision=2]{86,97} \\
							-969 &
							unbekannter fehlender Wert &
							  \num{3} &
							 - &
							  \num[round-mode=places,round-precision=2]{0,01} \\
					\midrule
					\multicolumn{2}{l}{\textbf{Summe (gesamt)}} &
				      \textbf{\num{28182}} &
				    \textbf{-} &
				    \textbf{100} \\
					\bottomrule
					\end{longtable}
					\end{filecontents}
					\LTXtable{\textwidth}{\jobname-cjob045_g1}
				\label{tableValues:cjob045_g1}
				\vspace*{-\baselineskip}
                    \begin{noten}
                	    \note{} Deskritive Maßzahlen:
                	    Anzahl unterschiedlicher Beobachtungen: 106%
                	    ; 
                	      Modus ($h$): 8630
                     \end{noten}


