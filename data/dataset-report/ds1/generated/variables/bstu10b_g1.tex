%EVERY VARIABLE HAS IT'S OWN PAGE

    \setcounter{footnote}{0}

    %omit vertical space
    \vspace*{-1.8cm}
	\section{bstu10b\_g1 (1. Qualifikation: 2. Hauptstudienfach (Studienbereich))}
	\label{section:bstu10b_g1}



	%TABLE FOR VARIABLE DETAILS
    \vspace*{0.5cm}
    \noindent\textbf{Eigenschaften
	% '#' has to be escaped
	\footnote{Detailliertere Informationen zur Variable finden sich unter
		\url{https://metadata.fdz.dzhw.eu/\#!/de/variables/var-gsl2008-ds1-bstu10b_g1$}}}\\
	\begin{tabularx}{\hsize}{@{}lX}
	Datentyp: & numerisch \\
	Skalenniveau: & nominal \\
	Zugangswege: &
	  remote-desktop-suf, 
	  onsite-suf
 \\
    \end{tabularx}



    %TABLE FOR QUESTION DETAILS
    %This has to be tested and has to be improved
    %rausfinden, ob einer Variable mehrere Fragen zugeordnet werden
    %dann evtl. nur die erste verwenden oder etwas anderes tun (Hinweis mehrere Fragen, auflisten mit Link)
				%TABLE FOR QUESTION DETAILS
				\vspace*{0.5cm}
                \noindent\textbf{Frage
	                \footnote{Detailliertere Informationen zur Frage finden sich unter
		              \url{https://metadata.fdz.dzhw.eu/\#!/de/questions/que-gsl2008-ins2-22$}}}\\
				\begin{tabularx}{\hsize}{@{}lX}
					Fragenummer: &
					  Fragebogen des DZHW-Studienberechtigtenpanels 2008 - zweite Welle:
					  22
 \\
					%--
					Fragetext: & Bitte machen Sie Angaben zum bereits begonnenen oder geplanten Studium, zur Berufsausbildung bzw. zur beruflichen Tätigkeit. \\
				\end{tabularx}





				%TABLE FOR THE NOMINAL / ORDINAL VALUES
        		\vspace*{0.5cm}
                \noindent\textbf{Häufigkeiten}

                \vspace*{-\baselineskip}
					%NUMERIC ELEMENTS NEED A HUGH SECOND COLOUMN AND A SMALL FIRST ONE
					\begin{filecontents}{\jobname-bstu10b_g1}
					\begin{longtable}{lXrrr}
					\toprule
					\textbf{Wert} & \textbf{Label} & \textbf{Häufigkeit} & \textbf{Prozent(gültig)} & \textbf{Prozent} \\
					\endhead
					\midrule
					\multicolumn{5}{l}{\textbf{Gültige Werte}}\\
						%DIFFERENT OBSERVATIONS <=20
								1 & \multicolumn{1}{X}{Sprach- und Kulturwissenschaften allgemein} & %5 &
								  \num{5} &
								%--
								  \num[round-mode=places,round-precision=2]{0,51} &
								  \num[round-mode=places,round-precision=2]{0,02} \\
								2 & \multicolumn{1}{X}{Evang. Theologie, -Religionslehre} & %16 &
								  \num{16} &
								%--
								  \num[round-mode=places,round-precision=2]{1,62} &
								  \num[round-mode=places,round-precision=2]{0,06} \\
								3 & \multicolumn{1}{X}{Kath. Theologie, Religionslehre} & %19 &
								  \num{19} &
								%--
								  \num[round-mode=places,round-precision=2]{1,92} &
								  \num[round-mode=places,round-precision=2]{0,07} \\
								4 & \multicolumn{1}{X}{Philosophie} & %48 &
								  \num{48} &
								%--
								  \num[round-mode=places,round-precision=2]{4,85} &
								  \num[round-mode=places,round-precision=2]{0,17} \\
								5 & \multicolumn{1}{X}{Geschichte} & %69 &
								  \num{69} &
								%--
								  \num[round-mode=places,round-precision=2]{6,97} &
								  \num[round-mode=places,round-precision=2]{0,24} \\
								6 & \multicolumn{1}{X}{Bibliothekswissenschaft, Dokumentation Publizistik} & %22 &
								  \num{22} &
								%--
								  \num[round-mode=places,round-precision=2]{2,22} &
								  \num[round-mode=places,round-precision=2]{0,08} \\
								7 & \multicolumn{1}{X}{Allgemeine und vergleichende Literatur- und Sprachwissenschaft} & %9 &
								  \num{9} &
								%--
								  \num[round-mode=places,round-precision=2]{0,91} &
								  \num[round-mode=places,round-precision=2]{0,03} \\
								8 & \multicolumn{1}{X}{Altphilologie (klass. Phililologie), Neugriechisch} & %15 &
								  \num{15} &
								%--
								  \num[round-mode=places,round-precision=2]{1,52} &
								  \num[round-mode=places,round-precision=2]{0,05} \\
								9 & \multicolumn{1}{X}{Germanistik (Deutsch, germanische Sprachen ohne Anglistik)} & %69 &
								  \num{69} &
								%--
								  \num[round-mode=places,round-precision=2]{6,97} &
								  \num[round-mode=places,round-precision=2]{0,24} \\
								10 & \multicolumn{1}{X}{Anglistik, Amerikanistik} & %50 &
								  \num{50} &
								%--
								  \num[round-mode=places,round-precision=2]{5,05} &
								  \num[round-mode=places,round-precision=2]{0,18} \\
							... & ... & ... & ... & ... \\
								65 & \multicolumn{1}{X}{Verkehrstechnick, Nautik} & %7 &
								  \num{7} &
								%--
								  \num[round-mode=places,round-precision=2]{0,71} &
								  \num[round-mode=places,round-precision=2]{0,02} \\

								66 & \multicolumn{1}{X}{Architektur, Innenarchitektur} & %5 &
								  \num{5} &
								%--
								  \num[round-mode=places,round-precision=2]{0,51} &
								  \num[round-mode=places,round-precision=2]{0,02} \\

								67 & \multicolumn{1}{X}{Raumplanung} & %2 &
								  \num{2} &
								%--
								  \num[round-mode=places,round-precision=2]{0,2} &
								  \num[round-mode=places,round-precision=2]{0,01} \\

								68 & \multicolumn{1}{X}{Bauningenieurwesen} & %8 &
								  \num{8} &
								%--
								  \num[round-mode=places,round-precision=2]{0,81} &
								  \num[round-mode=places,round-precision=2]{0,03} \\

								69 & \multicolumn{1}{X}{Vermessungswesen} & %1 &
								  \num{1} &
								%--
								  \num[round-mode=places,round-precision=2]{0,1} &
								  \num[round-mode=places,round-precision=2]{0} \\

								74 & \multicolumn{1}{X}{Kunst, Kunstwissenschaft allgemein} & %15 &
								  \num{15} &
								%--
								  \num[round-mode=places,round-precision=2]{1,52} &
								  \num[round-mode=places,round-precision=2]{0,05} \\

								75 & \multicolumn{1}{X}{Bildende Kunst} & %3 &
								  \num{3} &
								%--
								  \num[round-mode=places,round-precision=2]{0,3} &
								  \num[round-mode=places,round-precision=2]{0,01} \\

								76 & \multicolumn{1}{X}{Gestaltung} & %7 &
								  \num{7} &
								%--
								  \num[round-mode=places,round-precision=2]{0,71} &
								  \num[round-mode=places,round-precision=2]{0,02} \\

								77 & \multicolumn{1}{X}{Darstellende Kunst, Film und Fernsehen, Theaterwissenschaft} & %5 &
								  \num{5} &
								%--
								  \num[round-mode=places,round-precision=2]{0,51} &
								  \num[round-mode=places,round-precision=2]{0,02} \\

								78 & \multicolumn{1}{X}{Musik, Musikwissenschaft} & %11 &
								  \num{11} &
								%--
								  \num[round-mode=places,round-precision=2]{1,11} &
								  \num[round-mode=places,round-precision=2]{0,04} \\

					\midrule
					\multicolumn{2}{l}{Summe (gültig)} &
					  \textbf{\num{990}} &
					\textbf{100} &
					  \textbf{\num[round-mode=places,round-precision=2]{3,51}} \\
					%--
					\multicolumn{5}{l}{\textbf{Fehlende Werte}}\\
							-999 &
							weiß nicht &
							  \num{1} &
							 - &
							  \num[round-mode=places,round-precision=2]{0} \\
							-998 &
							keine Angabe &
							  \num{3467} &
							 - &
							  \num[round-mode=places,round-precision=2]{12,3} \\
							-995 &
							keine Teilnahme (Panel) &
							  \num{22249} &
							 - &
							  \num[round-mode=places,round-precision=2]{78,95} \\
							-989 &
							filterbedingt fehlend &
							  \num{1475} &
							 - &
							  \num[round-mode=places,round-precision=2]{5,23} \\
					\midrule
					\multicolumn{2}{l}{\textbf{Summe (gesamt)}} &
				      \textbf{\num{28182}} &
				    \textbf{-} &
				    \textbf{100} \\
					\bottomrule
					\end{longtable}
					\end{filecontents}
					\LTXtable{\textwidth}{\jobname-bstu10b_g1}
				\label{tableValues:bstu10b_g1}
				\vspace*{-\baselineskip}
                    \begin{noten}
                	    \note{} Deskritive Maßzahlen:
                	    Anzahl unterschiedlicher Beobachtungen: 57%
                	    ; 
                	      Modus ($h$): 30
                     \end{noten}


