%EVERY VARIABLE HAS IT'S OWN PAGE

    \setcounter{footnote}{0}

    %omit vertical space
    \vspace*{-1.8cm}
	\section{cvoc085\_g2 (5. Tätigkeit: Ausbildungsberuf (KldB-1992-Berufsordnung) (3-Steller))}
	\label{section:cvoc085_g2}



	%TABLE FOR VARIABLE DETAILS
    \vspace*{0.5cm}
    \noindent\textbf{Eigenschaften
	% '#' has to be escaped
	\footnote{Detailliertere Informationen zur Variable finden sich unter
		\url{https://metadata.fdz.dzhw.eu/\#!/de/variables/var-gsl2008-ds1-cvoc085_g2$}}}\\
	\begin{tabularx}{\hsize}{@{}lX}
	Datentyp: & numerisch \\
	Skalenniveau: & nominal \\
	Zugangswege: &
	  remote-desktop-suf, 
	  onsite-suf
 \\
    \end{tabularx}



    %TABLE FOR QUESTION DETAILS
    %This has to be tested and has to be improved
    %rausfinden, ob einer Variable mehrere Fragen zugeordnet werden
    %dann evtl. nur die erste verwenden oder etwas anderes tun (Hinweis mehrere Fragen, auflisten mit Link)
				%TABLE FOR QUESTION DETAILS
				\vspace*{0.5cm}
                \noindent\textbf{Frage
	                \footnote{Detailliertere Informationen zur Frage finden sich unter
		              \url{https://metadata.fdz.dzhw.eu/\#!/de/questions/que-gsl2008-ins3-2.1$}}}\\
				\begin{tabularx}{\hsize}{@{}lX}
					Fragenummer: &
					  Fragebogen des DZHW-Studienberechtigtenpanels 2008 - dritte Welle:
					  2.1
 \\
					%--
					Fragetext: & Wir bitten Sie nun, uns in dem folgenden Schema einen Überblick Ihres Werdegangs von Juli 2008 bis Dezember 2012 zu geben. \\
				\end{tabularx}





				%TABLE FOR THE NOMINAL / ORDINAL VALUES
        		\vspace*{0.5cm}
                \noindent\textbf{Häufigkeiten}

                \vspace*{-\baselineskip}
					%NUMERIC ELEMENTS NEED A HUGH SECOND COLOUMN AND A SMALL FIRST ONE
					\begin{filecontents}{\jobname-cvoc085_g2}
					\begin{longtable}{lXrrr}
					\toprule
					\textbf{Wert} & \textbf{Label} & \textbf{Häufigkeit} & \textbf{Prozent(gültig)} & \textbf{Prozent} \\
					\endhead
					\midrule
					\multicolumn{5}{l}{\textbf{Gültige Werte}}\\
						%DIFFERENT OBSERVATIONS <=20
								23 & \multicolumn{1}{X}{Tier-, Pferde-, Fischwirte und -wirtinnen} & %1 &
								  \num{1} &
								%--
								  \num[round-mode=places,round-precision=2]{1} &
								  \num[round-mode=places,round-precision=2]{0} \\
								51 & \multicolumn{1}{X}{Gärtner/Gärtnerinnen, Gartenarbeiter/Gartenarbeiterinnen} & %1 &
								  \num{1} &
								%--
								  \num[round-mode=places,round-precision=2]{1} &
								  \num[round-mode=places,round-precision=2]{0} \\
								270 & \multicolumn{1}{X}{Industriemechaniker/Industriemechanikerinnen o.n.F., Mechaniker/Mechanikerinnen o.n.A.} & %4 &
								  \num{4} &
								%--
								  \num[round-mode=places,round-precision=2]{4} &
								  \num[round-mode=places,round-precision=2]{0,01} \\
								281 & \multicolumn{1}{X}{Kraftfahrzeug-, Zweiradmechaniker und -mechanikerinnen} & %1 &
								  \num{1} &
								%--
								  \num[round-mode=places,round-precision=2]{1} &
								  \num[round-mode=places,round-precision=2]{0} \\
								290 & \multicolumn{1}{X}{Werkzeugmechaniker/Werkzeugmechanikerinnen, Werkzeugmacher/Werkzeugmacherinnen o.n.F.} & %1 &
								  \num{1} &
								%--
								  \num[round-mode=places,round-precision=2]{1} &
								  \num[round-mode=places,round-precision=2]{0} \\
								302 & \multicolumn{1}{X}{Edelmetallschmiede/Edelmetallschmiedinnen} & %1 &
								  \num{1} &
								%--
								  \num[round-mode=places,round-precision=2]{1} &
								  \num[round-mode=places,round-precision=2]{0} \\
								303 & \multicolumn{1}{X}{Zahntechniker/Zahntechnikerinnen} & %1 &
								  \num{1} &
								%--
								  \num[round-mode=places,round-precision=2]{1} &
								  \num[round-mode=places,round-precision=2]{0} \\
								305 & \multicolumn{1}{X}{Musikinstrumentenbauer/Musikinstrumentenbauerinnen} & %1 &
								  \num{1} &
								%--
								  \num[round-mode=places,round-precision=2]{1} &
								  \num[round-mode=places,round-precision=2]{0} \\
								315 & \multicolumn{1}{X}{Radio- und Fernsehtechniker/Radio- und Fernsehtechnikerinnen (Rundfunkmechaniker/innen) und verwandte Berufe} & %1 &
								  \num{1} &
								%--
								  \num[round-mode=places,round-precision=2]{1} &
								  \num[round-mode=places,round-precision=2]{0} \\
								351 & \multicolumn{1}{X}{Oberbekleidungsschneider/Oberbekleidungsschneiderinnen} & %2 &
								  \num{2} &
								%--
								  \num[round-mode=places,round-precision=2]{2} &
								  \num[round-mode=places,round-precision=2]{0,01} \\
							... & ... & ... & ... & ... \\
								858 & \multicolumn{1}{X}{Pharmazeutisch-technische Assistenten/Assistentinnen} & %1 &
								  \num{1} &
								%--
								  \num[round-mode=places,round-precision=2]{1} &
								  \num[round-mode=places,round-precision=2]{0} \\

								861 & \multicolumn{1}{X}{Sozialarbeiter/Sozialarbeiterinnen, Sozialpädagogen/Sozialpädagoginnen} & %1 &
								  \num{1} &
								%--
								  \num[round-mode=places,round-precision=2]{1} &
								  \num[round-mode=places,round-precision=2]{0} \\

								863 & \multicolumn{1}{X}{Erzieher/Erzieherinnen} & %3 &
								  \num{3} &
								%--
								  \num[round-mode=places,round-precision=2]{3} &
								  \num[round-mode=places,round-precision=2]{0,01} \\

								864 & \multicolumn{1}{X}{Altenpfleger/Altenpflegerinnen} & %2 &
								  \num{2} &
								%--
								  \num[round-mode=places,round-precision=2]{2} &
								  \num[round-mode=places,round-precision=2]{0,01} \\

								866 & \multicolumn{1}{X}{Heilerziehungspfleger/Heilerziehungspflegerinnen} & %1 &
								  \num{1} &
								%--
								  \num[round-mode=places,round-precision=2]{1} &
								  \num[round-mode=places,round-precision=2]{0} \\

								871 & \multicolumn{1}{X}{Hochschullehrer/Hochschullehrerinnen und verwandte Berufe} & %1 &
								  \num{1} &
								%--
								  \num[round-mode=places,round-precision=2]{1} &
								  \num[round-mode=places,round-precision=2]{0} \\

								873 & \multicolumn{1}{X}{Grund-, Haupt-, Real-, Sonderschullehrer und -lehrerinnen} & %1 &
								  \num{1} &
								%--
								  \num[round-mode=places,round-precision=2]{1} &
								  \num[round-mode=places,round-precision=2]{0} \\

								881 & \multicolumn{1}{X}{Wirtschaftswissenschaftler/Wirtschaftswissenschaftlerinnen, a.n.g.} & %1 &
								  \num{1} &
								%--
								  \num[round-mode=places,round-precision=2]{1} &
								  \num[round-mode=places,round-precision=2]{0} \\

								884 & \multicolumn{1}{X}{Sozialwissenschaftler/Sozialwissenschaftlerinnen, a.n.g.} & %1 &
								  \num{1} &
								%--
								  \num[round-mode=places,round-precision=2]{1} &
								  \num[round-mode=places,round-precision=2]{0} \\

								912 & \multicolumn{1}{X}{Restaurantfachleute, Stewards/Stewardessen} & %1 &
								  \num{1} &
								%--
								  \num[round-mode=places,round-precision=2]{1} &
								  \num[round-mode=places,round-precision=2]{0} \\

					\midrule
					\multicolumn{2}{l}{Summe (gültig)} &
					  \textbf{\num{100}} &
					\textbf{100} &
					  \textbf{\num[round-mode=places,round-precision=2]{0,35}} \\
					%--
					\multicolumn{5}{l}{\textbf{Fehlende Werte}}\\
							-998 &
							keine Angabe &
							  \num{3571} &
							 - &
							  \num[round-mode=places,round-precision=2]{12,67} \\
							-995 &
							keine Teilnahme (Panel) &
							  \num{24511} &
							 - &
							  \num[round-mode=places,round-precision=2]{86,97} \\
					\midrule
					\multicolumn{2}{l}{\textbf{Summe (gesamt)}} &
				      \textbf{\num{28182}} &
				    \textbf{-} &
				    \textbf{100} \\
					\bottomrule
					\end{longtable}
					\end{filecontents}
					\LTXtable{\textwidth}{\jobname-cvoc085_g2}
				\label{tableValues:cvoc085_g2}
				\vspace*{-\baselineskip}
                    \begin{noten}
                	    \note{} Deskritive Maßzahlen:
                	    Anzahl unterschiedlicher Beobachtungen: 59%
                	    ; 
                	      Modus ($h$): 853
                     \end{noten}


