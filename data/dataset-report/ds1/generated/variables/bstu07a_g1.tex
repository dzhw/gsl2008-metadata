%EVERY VARIABLE HAS IT'S OWN PAGE

    \setcounter{footnote}{0}

    %omit vertical space
    \vspace*{-1.8cm}
	\section{bstu07a\_g1 (1. Nennung Idee: Studienfach)}
	\label{section:bstu07a_g1}



	%TABLE FOR VARIABLE DETAILS
    \vspace*{0.5cm}
    \noindent\textbf{Eigenschaften
	% '#' has to be escaped
	\footnote{Detailliertere Informationen zur Variable finden sich unter
		\url{https://metadata.fdz.dzhw.eu/\#!/de/variables/var-gsl2008-ds1-bstu07a_g1$}}}\\
	\begin{tabularx}{\hsize}{@{}lX}
	Datentyp: & numerisch \\
	Skalenniveau: & nominal \\
	Zugangswege: &
	  remote-desktop-suf, 
	  onsite-suf
 \\
    \end{tabularx}



    %TABLE FOR QUESTION DETAILS
    %This has to be tested and has to be improved
    %rausfinden, ob einer Variable mehrere Fragen zugeordnet werden
    %dann evtl. nur die erste verwenden oder etwas anderes tun (Hinweis mehrere Fragen, auflisten mit Link)
				%TABLE FOR QUESTION DETAILS
				\vspace*{0.5cm}
                \noindent\textbf{Frage
	                \footnote{Detailliertere Informationen zur Frage finden sich unter
		              \url{https://metadata.fdz.dzhw.eu/\#!/de/questions/que-gsl2008-ins2-18$}}}\\
				\begin{tabularx}{\hsize}{@{}lX}
					Fragenummer: &
					  Fragebogen des DZHW-Studienberechtigtenpanels 2008 - zweite Welle:
					  18
 \\
					%--
					Fragetext: & Für welchen nächsten Schritt Ihres nachschulischen Werdegangs haben Sie sich entschieden?\par  ich habe mich noch nicht endgültig entschieden, werde aber wahrscheinlich … \\
				\end{tabularx}





				%TABLE FOR THE NOMINAL / ORDINAL VALUES
        		\vspace*{0.5cm}
                \noindent\textbf{Häufigkeiten}

                \vspace*{-\baselineskip}
					%NUMERIC ELEMENTS NEED A HUGH SECOND COLOUMN AND A SMALL FIRST ONE
					\begin{filecontents}{\jobname-bstu07a_g1}
					\begin{longtable}{lXrrr}
					\toprule
					\textbf{Wert} & \textbf{Label} & \textbf{Häufigkeit} & \textbf{Prozent(gültig)} & \textbf{Prozent} \\
					\endhead
					\midrule
					\multicolumn{5}{l}{\textbf{Gültige Werte}}\\
						%DIFFERENT OBSERVATIONS <=20
								3 & \multicolumn{1}{X}{Agrarwissenschaft/Landwirtschaft} & %1 &
								  \num{1} &
								%--
								  \num[round-mode=places,round-precision=2]{1,61} &
								  \num[round-mode=places,round-precision=2]{0} \\
								4 & \multicolumn{1}{X}{Interdisziplinäre Studien (Schwerpunkt Sprach- und Kulturwissenschaften)} & %1 &
								  \num{1} &
								%--
								  \num[round-mode=places,round-precision=2]{1,61} &
								  \num[round-mode=places,round-precision=2]{0} \\
								8 & \multicolumn{1}{X}{Anglistik/Englisch} & %1 &
								  \num{1} &
								%--
								  \num[round-mode=places,round-precision=2]{1,61} &
								  \num[round-mode=places,round-precision=2]{0} \\
								21 & \multicolumn{1}{X}{Betriebswirtschaftslehre} & %4 &
								  \num{4} &
								%--
								  \num[round-mode=places,round-precision=2]{6,45} &
								  \num[round-mode=places,round-precision=2]{0,01} \\
								26 & \multicolumn{1}{X}{Biologie} & %1 &
								  \num{1} &
								%--
								  \num[round-mode=places,round-precision=2]{1,61} &
								  \num[round-mode=places,round-precision=2]{0} \\
								52 & \multicolumn{1}{X}{Erziehungswissenschaft (Pädagogik)} & %3 &
								  \num{3} &
								%--
								  \num[round-mode=places,round-precision=2]{4,84} &
								  \num[round-mode=places,round-precision=2]{0,01} \\
								67 & \multicolumn{1}{X}{Germanistik/Deutsch} & %2 &
								  \num{2} &
								%--
								  \num[round-mode=places,round-precision=2]{3,23} &
								  \num[round-mode=places,round-precision=2]{0,01} \\
								69 & \multicolumn{1}{X}{Graphikdesign/Kommunikationsgestaltung} & %3 &
								  \num{3} &
								%--
								  \num[round-mode=places,round-precision=2]{4,84} &
								  \num[round-mode=places,round-precision=2]{0,01} \\
								92 & \multicolumn{1}{X}{Kunstgeschichte, Kunstwissenschaft} & %1 &
								  \num{1} &
								%--
								  \num[round-mode=places,round-precision=2]{1,61} &
								  \num[round-mode=places,round-precision=2]{0} \\
								104 & \multicolumn{1}{X}{Maschinenbau/-wesen} & %2 &
								  \num{2} &
								%--
								  \num[round-mode=places,round-precision=2]{3,23} &
								  \num[round-mode=places,round-precision=2]{0,01} \\
							... & ... & ... & ... & ... \\
								226 & \multicolumn{1}{X}{Verfahrenstechnik} & %1 &
								  \num{1} &
								%--
								  \num[round-mode=places,round-precision=2]{1,61} &
								  \num[round-mode=places,round-precision=2]{0} \\

								231 & \multicolumn{1}{X}{Druck- und Reproduktionstechnik} & %1 &
								  \num{1} &
								%--
								  \num[round-mode=places,round-precision=2]{1,61} &
								  \num[round-mode=places,round-precision=2]{0} \\

								232 & \multicolumn{1}{X}{Gesundheitswissenschaft/-management} & %2 &
								  \num{2} &
								%--
								  \num[round-mode=places,round-precision=2]{3,23} &
								  \num[round-mode=places,round-precision=2]{0,01} \\

								242 & \multicolumn{1}{X}{Innenarchitektur} & %1 &
								  \num{1} &
								%--
								  \num[round-mode=places,round-precision=2]{1,61} &
								  \num[round-mode=places,round-precision=2]{0} \\

								245 & \multicolumn{1}{X}{Sozialpädagogik} & %2 &
								  \num{2} &
								%--
								  \num[round-mode=places,round-precision=2]{3,23} &
								  \num[round-mode=places,round-precision=2]{0,01} \\

								263 & \multicolumn{1}{X}{Polizei/Verfassungsschutz} & %3 &
								  \num{3} &
								%--
								  \num[round-mode=places,round-precision=2]{4,84} &
								  \num[round-mode=places,round-precision=2]{0,01} \\

								268 & \multicolumn{1}{X}{Verkehrswesen} & %1 &
								  \num{1} &
								%--
								  \num[round-mode=places,round-precision=2]{1,61} &
								  \num[round-mode=places,round-precision=2]{0} \\

								274 & \multicolumn{1}{X}{Tourismuswirtschaft} & %1 &
								  \num{1} &
								%--
								  \num[round-mode=places,round-precision=2]{1,61} &
								  \num[round-mode=places,round-precision=2]{0} \\

								583 & \multicolumn{1}{X}{Industrie} & %1 &
								  \num{1} &
								%--
								  \num[round-mode=places,round-precision=2]{1,61} &
								  \num[round-mode=places,round-precision=2]{0} \\

								610 & \multicolumn{1}{X}{Buchhandel/Verlagswirtschaft} & %1 &
								  \num{1} &
								%--
								  \num[round-mode=places,round-precision=2]{1,61} &
								  \num[round-mode=places,round-precision=2]{0} \\

					\midrule
					\multicolumn{2}{l}{Summe (gültig)} &
					  \textbf{\num{62}} &
					\textbf{100} &
					  \textbf{\num[round-mode=places,round-precision=2]{0,22}} \\
					%--
					\multicolumn{5}{l}{\textbf{Fehlende Werte}}\\
							-998 &
							keine Angabe &
							  \num{18} &
							 - &
							  \num[round-mode=places,round-precision=2]{0,06} \\
							-995 &
							keine Teilnahme (Panel) &
							  \num{22249} &
							 - &
							  \num[round-mode=places,round-precision=2]{78,95} \\
							-989 &
							filterbedingt fehlend &
							  \num{5821} &
							 - &
							  \num[round-mode=places,round-precision=2]{20,66} \\
							-969 &
							unbekannter fehlender Wert &
							  \num{32} &
							 - &
							  \num[round-mode=places,round-precision=2]{0,11} \\
					\midrule
					\multicolumn{2}{l}{\textbf{Summe (gesamt)}} &
				      \textbf{\num{28182}} &
				    \textbf{-} &
				    \textbf{100} \\
					\bottomrule
					\end{longtable}
					\end{filecontents}
					\LTXtable{\textwidth}{\jobname-bstu07a_g1}
				\label{tableValues:bstu07a_g1}
				\vspace*{-\baselineskip}
                    \begin{noten}
                	    \note{} Deskritive Maßzahlen:
                	    Anzahl unterschiedlicher Beobachtungen: 35%
                	    ; 
                	      Modus ($h$): 208
                     \end{noten}


