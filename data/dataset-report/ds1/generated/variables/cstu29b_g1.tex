%EVERY VARIABLE HAS IT'S OWN PAGE

    \setcounter{footnote}{0}

    %omit vertical space
    \vspace*{-1.8cm}
	\section{cstu29b\_g1 (Wechsel Hochschule: zu Bundesland der HS)}
	\label{section:cstu29b_g1}



	%TABLE FOR VARIABLE DETAILS
    \vspace*{0.5cm}
    \noindent\textbf{Eigenschaften
	% '#' has to be escaped
	\footnote{Detailliertere Informationen zur Variable finden sich unter
		\url{https://metadata.fdz.dzhw.eu/\#!/de/variables/var-gsl2008-ds1-cstu29b_g1$}}}\\
	\begin{tabularx}{\hsize}{@{}lX}
	Datentyp: & numerisch \\
	Skalenniveau: & nominal \\
	Zugangswege: &
	  remote-desktop-suf, 
	  onsite-suf
 \\
    \end{tabularx}



    %TABLE FOR QUESTION DETAILS
    %This has to be tested and has to be improved
    %rausfinden, ob einer Variable mehrere Fragen zugeordnet werden
    %dann evtl. nur die erste verwenden oder etwas anderes tun (Hinweis mehrere Fragen, auflisten mit Link)
				%TABLE FOR QUESTION DETAILS
				\vspace*{0.5cm}
                \noindent\textbf{Frage
	                \footnote{Detailliertere Informationen zur Frage finden sich unter
		              \url{https://metadata.fdz.dzhw.eu/\#!/de/questions/que-gsl2008-ins3-4.8$}}}\\
				\begin{tabularx}{\hsize}{@{}lX}
					Fragenummer: &
					  Fragebogen des DZHW-Studienberechtigtenpanels 2008 - dritte Welle:
					  4.8
 \\
					%--
					Fragetext: & Von welcher Hochschule zu welcher Hochschule haben Sie gewechselt bzw. werden Sie wechseln? \\
				\end{tabularx}





				%TABLE FOR THE NOMINAL / ORDINAL VALUES
        		\vspace*{0.5cm}
                \noindent\textbf{Häufigkeiten}

                \vspace*{-\baselineskip}
					%NUMERIC ELEMENTS NEED A HUGH SECOND COLOUMN AND A SMALL FIRST ONE
					\begin{filecontents}{\jobname-cstu29b_g1}
					\begin{longtable}{lXrrr}
					\toprule
					\textbf{Wert} & \textbf{Label} & \textbf{Häufigkeit} & \textbf{Prozent(gültig)} & \textbf{Prozent} \\
					\endhead
					\midrule
					\multicolumn{5}{l}{\textbf{Gültige Werte}}\\
						%DIFFERENT OBSERVATIONS <=20

					1 &
				% TODO try size/length gt 0; take over for other passages
					\multicolumn{1}{X}{ Schleswig-Holstein   } &


					%13 &
					  \num{13} &
					%--
					  \num[round-mode=places,round-precision=2]{2,32} &
					    \num[round-mode=places,round-precision=2]{0,05} \\
							%????

					2 &
				% TODO try size/length gt 0; take over for other passages
					\multicolumn{1}{X}{ Hamburg   } &


					%23 &
					  \num{23} &
					%--
					  \num[round-mode=places,round-precision=2]{4,11} &
					    \num[round-mode=places,round-precision=2]{0,08} \\
							%????

					3 &
				% TODO try size/length gt 0; take over for other passages
					\multicolumn{1}{X}{ Niedersachsen   } &


					%47 &
					  \num{47} &
					%--
					  \num[round-mode=places,round-precision=2]{8,39} &
					    \num[round-mode=places,round-precision=2]{0,17} \\
							%????

					4 &
				% TODO try size/length gt 0; take over for other passages
					\multicolumn{1}{X}{ Bremen   } &


					%10 &
					  \num{10} &
					%--
					  \num[round-mode=places,round-precision=2]{1,79} &
					    \num[round-mode=places,round-precision=2]{0,04} \\
							%????

					5 &
				% TODO try size/length gt 0; take over for other passages
					\multicolumn{1}{X}{ Nordrhein-Westfalen   } &


					%108 &
					  \num{108} &
					%--
					  \num[round-mode=places,round-precision=2]{19,29} &
					    \num[round-mode=places,round-precision=2]{0,38} \\
							%????

					6 &
				% TODO try size/length gt 0; take over for other passages
					\multicolumn{1}{X}{ Hessen   } &


					%43 &
					  \num{43} &
					%--
					  \num[round-mode=places,round-precision=2]{7,68} &
					    \num[round-mode=places,round-precision=2]{0,15} \\
							%????

					7 &
				% TODO try size/length gt 0; take over for other passages
					\multicolumn{1}{X}{ Rheinland-Pfalz   } &


					%25 &
					  \num{25} &
					%--
					  \num[round-mode=places,round-precision=2]{4,46} &
					    \num[round-mode=places,round-precision=2]{0,09} \\
							%????

					8 &
				% TODO try size/length gt 0; take over for other passages
					\multicolumn{1}{X}{ Baden-Württemberg   } &


					%69 &
					  \num{69} &
					%--
					  \num[round-mode=places,round-precision=2]{12,32} &
					    \num[round-mode=places,round-precision=2]{0,24} \\
							%????

					9 &
				% TODO try size/length gt 0; take over for other passages
					\multicolumn{1}{X}{ Bayern   } &


					%77 &
					  \num{77} &
					%--
					  \num[round-mode=places,round-precision=2]{13,75} &
					    \num[round-mode=places,round-precision=2]{0,27} \\
							%????

					10 &
				% TODO try size/length gt 0; take over for other passages
					\multicolumn{1}{X}{ Saarland   } &


					%4 &
					  \num{4} &
					%--
					  \num[round-mode=places,round-precision=2]{0,71} &
					    \num[round-mode=places,round-precision=2]{0,01} \\
							%????

					11 &
				% TODO try size/length gt 0; take over for other passages
					\multicolumn{1}{X}{ Berlin   } &


					%45 &
					  \num{45} &
					%--
					  \num[round-mode=places,round-precision=2]{8,04} &
					    \num[round-mode=places,round-precision=2]{0,16} \\
							%????

					12 &
				% TODO try size/length gt 0; take over for other passages
					\multicolumn{1}{X}{ Brandenburg   } &


					%9 &
					  \num{9} &
					%--
					  \num[round-mode=places,round-precision=2]{1,61} &
					    \num[round-mode=places,round-precision=2]{0,03} \\
							%????

					13 &
				% TODO try size/length gt 0; take over for other passages
					\multicolumn{1}{X}{ Mecklenburg-Vorpommern   } &


					%10 &
					  \num{10} &
					%--
					  \num[round-mode=places,round-precision=2]{1,79} &
					    \num[round-mode=places,round-precision=2]{0,04} \\
							%????

					14 &
				% TODO try size/length gt 0; take over for other passages
					\multicolumn{1}{X}{ Sachsen   } &


					%42 &
					  \num{42} &
					%--
					  \num[round-mode=places,round-precision=2]{7,5} &
					    \num[round-mode=places,round-precision=2]{0,15} \\
							%????

					15 &
				% TODO try size/length gt 0; take over for other passages
					\multicolumn{1}{X}{ Sachsen-Anhalt   } &


					%14 &
					  \num{14} &
					%--
					  \num[round-mode=places,round-precision=2]{2,5} &
					    \num[round-mode=places,round-precision=2]{0,05} \\
							%????

					16 &
				% TODO try size/length gt 0; take over for other passages
					\multicolumn{1}{X}{ Thüringen   } &


					%20 &
					  \num{20} &
					%--
					  \num[round-mode=places,round-precision=2]{3,57} &
					    \num[round-mode=places,round-precision=2]{0,07} \\
							%????

					9000 &
				% TODO try size/length gt 0; take over for other passages
					\multicolumn{1}{X}{ sonstige deutsche Hochschule   } &


					%1 &
					  \num{1} &
					%--
					  \num[round-mode=places,round-precision=2]{0,18} &
					    \num[round-mode=places,round-precision=2]{0} \\
							%????
						%DIFFERENT OBSERVATIONS >20
					\midrule
					\multicolumn{2}{l}{Summe (gültig)} &
					  \textbf{\num{560}} &
					\textbf{100} &
					  \textbf{\num[round-mode=places,round-precision=2]{1,99}} \\
					%--
					\multicolumn{5}{l}{\textbf{Fehlende Werte}}\\
							-998 &
							keine Angabe &
							  \num{2} &
							 - &
							  \num[round-mode=places,round-precision=2]{0,01} \\
							-995 &
							keine Teilnahme (Panel) &
							  \num{24511} &
							 - &
							  \num[round-mode=places,round-precision=2]{86,97} \\
							-989 &
							filterbedingt fehlend &
							  \num{3037} &
							 - &
							  \num[round-mode=places,round-precision=2]{10,78} \\
							-988 &
							trifft nicht zu &
							  \num{70} &
							 - &
							  \num[round-mode=places,round-precision=2]{0,25} \\
							-969 &
							unbekannter fehlender Wert &
							  \num{2} &
							 - &
							  \num[round-mode=places,round-precision=2]{0,01} \\
					\midrule
					\multicolumn{2}{l}{\textbf{Summe (gesamt)}} &
				      \textbf{\num{28182}} &
				    \textbf{-} &
				    \textbf{100} \\
					\bottomrule
					\end{longtable}
					\end{filecontents}
					\LTXtable{\textwidth}{\jobname-cstu29b_g1}
				\label{tableValues:cstu29b_g1}
				\vspace*{-\baselineskip}
                    \begin{noten}
                	    \note{} Deskritive Maßzahlen:
                	    Anzahl unterschiedlicher Beobachtungen: 17%
                	    ; 
                	      Modus ($h$): 5
                     \end{noten}


