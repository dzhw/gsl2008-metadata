%EVERY VARIABLE HAS IT'S OWN PAGE

    \setcounter{footnote}{0}

    %omit vertical space
    \vspace*{-1.8cm}
	\section{bdem12b\_g2 (Mutter: Beruf (KldB-1992-Berufsordnung) (3-Steller))}
	\label{section:bdem12b_g2}



	%TABLE FOR VARIABLE DETAILS
    \vspace*{0.5cm}
    \noindent\textbf{Eigenschaften
	% '#' has to be escaped
	\footnote{Detailliertere Informationen zur Variable finden sich unter
		\url{https://metadata.fdz.dzhw.eu/\#!/de/variables/var-gsl2008-ds1-bdem12b_g2$}}}\\
	\begin{tabularx}{\hsize}{@{}lX}
	Datentyp: & numerisch \\
	Skalenniveau: & nominal \\
	Zugangswege: &
	  remote-desktop-suf, 
	  onsite-suf
 \\
    \end{tabularx}



    %TABLE FOR QUESTION DETAILS
    %This has to be tested and has to be improved
    %rausfinden, ob einer Variable mehrere Fragen zugeordnet werden
    %dann evtl. nur die erste verwenden oder etwas anderes tun (Hinweis mehrere Fragen, auflisten mit Link)
				%TABLE FOR QUESTION DETAILS
				\vspace*{0.5cm}
                \noindent\textbf{Frage
	                \footnote{Detailliertere Informationen zur Frage finden sich unter
		              \url{https://metadata.fdz.dzhw.eu/\#!/de/questions/que-gsl2008-ins2-44$}}}\\
				\begin{tabularx}{\hsize}{@{}lX}
					Fragenummer: &
					  Fragebogen des DZHW-Studienberechtigtenpanels 2008 - zweite Welle:
					  44
 \\
					%--
					Fragetext: & Welchen Beruf üben/übten Ihre Eltern aktuell bzw. zuletzt hauptberuflich aus? \\
				\end{tabularx}





				%TABLE FOR THE NOMINAL / ORDINAL VALUES
        		\vspace*{0.5cm}
                \noindent\textbf{Häufigkeiten}

                \vspace*{-\baselineskip}
					%NUMERIC ELEMENTS NEED A HUGH SECOND COLOUMN AND A SMALL FIRST ONE
					\begin{filecontents}{\jobname-bdem12b_g2}
					\begin{longtable}{lXrrr}
					\toprule
					\textbf{Wert} & \textbf{Label} & \textbf{Häufigkeit} & \textbf{Prozent(gültig)} & \textbf{Prozent} \\
					\endhead
					\midrule
					\multicolumn{5}{l}{\textbf{Gültige Werte}}\\
						%DIFFERENT OBSERVATIONS <=20
								11 & \multicolumn{1}{X}{Landwirte/Landwirtinnen, Pflanzenschützer/Pflanzenschützerinnen} & %21 &
								  \num{21} &
								%--
								  \num[round-mode=places,round-precision=2]{0,38} &
								  \num[round-mode=places,round-precision=2]{0,07} \\
								12 & \multicolumn{1}{X}{Winzer/Winzerinnen} & %1 &
								  \num{1} &
								%--
								  \num[round-mode=places,round-precision=2]{0,02} &
								  \num[round-mode=places,round-precision=2]{0} \\
								13 & \multicolumn{1}{X}{Landarbeitskräfte} & %1 &
								  \num{1} &
								%--
								  \num[round-mode=places,round-precision=2]{0,02} &
								  \num[round-mode=places,round-precision=2]{0} \\
								23 & \multicolumn{1}{X}{Tier-, Pferde-, Fischwirte und -wirtinnen} & %3 &
								  \num{3} &
								%--
								  \num[round-mode=places,round-precision=2]{0,05} &
								  \num[round-mode=places,round-precision=2]{0,01} \\
								24 & \multicolumn{1}{X}{Tierpfleger/Tierpflegerinnen und verwandte Berufe, a.n.g.} & %7 &
								  \num{7} &
								%--
								  \num[round-mode=places,round-precision=2]{0,13} &
								  \num[round-mode=places,round-precision=2]{0,02} \\
								31 & \multicolumn{1}{X}{Verwalter/Verwalterinnen in der Land- und Tierwirtschaft} & %2 &
								  \num{2} &
								%--
								  \num[round-mode=places,round-precision=2]{0,04} &
								  \num[round-mode=places,round-precision=2]{0,01} \\
								32 & \multicolumn{1}{X}{Land-, Tierwirtschaftsberater und -beraterinnen, Agraringenieur/Agraringenieurinnen, Agrartechniker/Agrartechnikerinnen} & %5 &
								  \num{5} &
								%--
								  \num[round-mode=places,round-precision=2]{0,09} &
								  \num[round-mode=places,round-precision=2]{0,02} \\
								51 & \multicolumn{1}{X}{Gärtner/Gärtnerinnen, Gartenarbeiter/Gartenarbeiterinnen} & %17 &
								  \num{17} &
								%--
								  \num[round-mode=places,round-precision=2]{0,31} &
								  \num[round-mode=places,round-precision=2]{0,06} \\
								52 & \multicolumn{1}{X}{Ingenieure/Ingenieurinnen, Techniker/Technikerinnen in Gartenbau und Landespflege} & %2 &
								  \num{2} &
								%--
								  \num[round-mode=places,round-precision=2]{0,04} &
								  \num[round-mode=places,round-precision=2]{0,01} \\
								53 & \multicolumn{1}{X}{Floristen/Floristinnen} & %20 &
								  \num{20} &
								%--
								  \num[round-mode=places,round-precision=2]{0,36} &
								  \num[round-mode=places,round-precision=2]{0,07} \\
							... & ... & ... & ... & ... \\
								993 & \multicolumn{1}{X}{Vorarbeiter/Vorarbeiterinnen, Gruppenleiter/Gruppenleiterinnen ohne nähere Tätigkeitsangabe} & %2 &
								  \num{2} &
								%--
								  \num[round-mode=places,round-precision=2]{0,04} &
								  \num[round-mode=places,round-precision=2]{0,01} \\

								995 & \multicolumn{1}{X}{Selbständige ohne nähere Tätigkeitsangabe} & %8 &
								  \num{8} &
								%--
								  \num[round-mode=places,round-precision=2]{0,14} &
								  \num[round-mode=places,round-precision=2]{0,03} \\

								996 & \multicolumn{1}{X}{Beratungs-, Planungsfachleute ohne nähere Tätigkeitsangabe} & %2 &
								  \num{2} &
								%--
								  \num[round-mode=places,round-precision=2]{0,04} &
								  \num[round-mode=places,round-precision=2]{0,01} \\

								997 & \multicolumn{1}{X}{Sonstige Arbeitskräfte ohne nähere Tätigkeitsangabe} & %14 &
								  \num{14} &
								%--
								  \num[round-mode=places,round-precision=2]{0,25} &
								  \num[round-mode=places,round-precision=2]{0,05} \\

								9912 & \multicolumn{1}{X}{Student} & %3 &
								  \num{3} &
								%--
								  \num[round-mode=places,round-precision=2]{0,05} &
								  \num[round-mode=places,round-precision=2]{0,01} \\

								9995 & \multicolumn{1}{X}{krank/arbeitsunfähig} & %4 &
								  \num{4} &
								%--
								  \num[round-mode=places,round-precision=2]{0,07} &
								  \num[round-mode=places,round-precision=2]{0,01} \\

								9996 & \multicolumn{1}{X}{verstorben} & %7 &
								  \num{7} &
								%--
								  \num[round-mode=places,round-precision=2]{0,13} &
								  \num[round-mode=places,round-precision=2]{0,02} \\

								9997 & \multicolumn{1}{X}{Arbeitslos} & %22 &
								  \num{22} &
								%--
								  \num[round-mode=places,round-precision=2]{0,4} &
								  \num[round-mode=places,round-precision=2]{0,08} \\

								9998 & \multicolumn{1}{X}{Rentner} & %10 &
								  \num{10} &
								%--
								  \num[round-mode=places,round-precision=2]{0,18} &
								  \num[round-mode=places,round-precision=2]{0,04} \\

								9999 & \multicolumn{1}{X}{Hausfrau/Hausmann} & %206 &
								  \num{206} &
								%--
								  \num[round-mode=places,round-precision=2]{3,73} &
								  \num[round-mode=places,round-precision=2]{0,73} \\

					\midrule
					\multicolumn{2}{l}{Summe (gültig)} &
					  \textbf{\num{5518}} &
					\textbf{100} &
					  \textbf{\num[round-mode=places,round-precision=2]{19,58}} \\
					%--
					\multicolumn{5}{l}{\textbf{Fehlende Werte}}\\
							-999 &
							weiß nicht &
							  \num{2} &
							 - &
							  \num[round-mode=places,round-precision=2]{0,01} \\
							-998 &
							keine Angabe &
							  \num{126} &
							 - &
							  \num[round-mode=places,round-precision=2]{0,45} \\
							-995 &
							keine Teilnahme (Panel) &
							  \num{22249} &
							 - &
							  \num[round-mode=places,round-precision=2]{78,95} \\
							-988 &
							trifft nicht zu &
							  \num{1} &
							 - &
							  \num[round-mode=places,round-precision=2]{0} \\
							-969 &
							unbekannter fehlender Wert &
							  \num{286} &
							 - &
							  \num[round-mode=places,round-precision=2]{1,01} \\
					\midrule
					\multicolumn{2}{l}{\textbf{Summe (gesamt)}} &
				      \textbf{\num{28182}} &
				    \textbf{-} &
				    \textbf{100} \\
					\bottomrule
					\end{longtable}
					\end{filecontents}
					\LTXtable{\textwidth}{\jobname-bdem12b_g2}
				\label{tableValues:bdem12b_g2}
				\vspace*{-\baselineskip}
                    \begin{noten}
                	    \note{} Deskritive Maßzahlen:
                	    Anzahl unterschiedlicher Beobachtungen: 241%
                	    ; 
                	      Modus ($h$): 853
                     \end{noten}


