%EVERY VARIABLE HAS IT'S OWN PAGE

    \setcounter{footnote}{0}

    %omit vertical space
    \vspace*{-1.8cm}
	\section{cbeg08\_g2 (8. Tätigkeit: Beginn (Monat))}
	\label{section:cbeg08_g2}



	%TABLE FOR VARIABLE DETAILS
    \vspace*{0.5cm}
    \noindent\textbf{Eigenschaften
	% '#' has to be escaped
	\footnote{Detailliertere Informationen zur Variable finden sich unter
		\url{https://metadata.fdz.dzhw.eu/\#!/de/variables/var-gsl2008-ds1-cbeg08_g2$}}}\\
	\begin{tabularx}{\hsize}{@{}lX}
	Datentyp: & numerisch \\
	Skalenniveau: & ordinal \\
	Zugangswege: &
	  remote-desktop-suf, 
	  onsite-suf
 \\
    \end{tabularx}



    %TABLE FOR QUESTION DETAILS
    %This has to be tested and has to be improved
    %rausfinden, ob einer Variable mehrere Fragen zugeordnet werden
    %dann evtl. nur die erste verwenden oder etwas anderes tun (Hinweis mehrere Fragen, auflisten mit Link)
				%TABLE FOR QUESTION DETAILS
				\vspace*{0.5cm}
                \noindent\textbf{Frage
	                \footnote{Detailliertere Informationen zur Frage finden sich unter
		              \url{https://metadata.fdz.dzhw.eu/\#!/de/questions/que-gsl2008-ins3-2.1$}}}\\
				\begin{tabularx}{\hsize}{@{}lX}
					Fragenummer: &
					  Fragebogen des DZHW-Studienberechtigtenpanels 2008 - dritte Welle:
					  2.1
 \\
					%--
					Fragetext: & Wir bitten Sie nun, uns in dem folgenden Schema einen Überblick Ihres Werdegangs von Juli 2008 bis Dezember 2012 zu geben.\par  Zeitraum\par  Beginn\par  Monat/Jahr \\
				\end{tabularx}





				%TABLE FOR THE NOMINAL / ORDINAL VALUES
        		\vspace*{0.5cm}
                \noindent\textbf{Häufigkeiten}

                \vspace*{-\baselineskip}
					%NUMERIC ELEMENTS NEED A HUGH SECOND COLOUMN AND A SMALL FIRST ONE
					\begin{filecontents}{\jobname-cbeg08_g2}
					\begin{longtable}{lXrrr}
					\toprule
					\textbf{Wert} & \textbf{Label} & \textbf{Häufigkeit} & \textbf{Prozent(gültig)} & \textbf{Prozent} \\
					\endhead
					\midrule
					\multicolumn{5}{l}{\textbf{Gültige Werte}}\\
						%DIFFERENT OBSERVATIONS <=20

					1 &
				% TODO try size/length gt 0; take over for other passages
					\multicolumn{1}{X}{ Januar   } &


					%3 &
					  \num{3} &
					%--
					  \num[round-mode=places,round-precision=2]{1,07} &
					    \num[round-mode=places,round-precision=2]{0,01} \\
							%????

					2 &
				% TODO try size/length gt 0; take over for other passages
					\multicolumn{1}{X}{ Februar   } &


					%15 &
					  \num{15} &
					%--
					  \num[round-mode=places,round-precision=2]{5,36} &
					    \num[round-mode=places,round-precision=2]{0,05} \\
							%????

					3 &
				% TODO try size/length gt 0; take over for other passages
					\multicolumn{1}{X}{ März   } &


					%15 &
					  \num{15} &
					%--
					  \num[round-mode=places,round-precision=2]{5,36} &
					    \num[round-mode=places,round-precision=2]{0,05} \\
							%????

					4 &
				% TODO try size/length gt 0; take over for other passages
					\multicolumn{1}{X}{ April   } &


					%21 &
					  \num{21} &
					%--
					  \num[round-mode=places,round-precision=2]{7,5} &
					    \num[round-mode=places,round-precision=2]{0,07} \\
							%????

					5 &
				% TODO try size/length gt 0; take over for other passages
					\multicolumn{1}{X}{ Mai   } &


					%16 &
					  \num{16} &
					%--
					  \num[round-mode=places,round-precision=2]{5,71} &
					    \num[round-mode=places,round-precision=2]{0,06} \\
							%????

					6 &
				% TODO try size/length gt 0; take over for other passages
					\multicolumn{1}{X}{ Juni   } &


					%13 &
					  \num{13} &
					%--
					  \num[round-mode=places,round-precision=2]{4,64} &
					    \num[round-mode=places,round-precision=2]{0,05} \\
							%????

					7 &
				% TODO try size/length gt 0; take over for other passages
					\multicolumn{1}{X}{ Juli   } &


					%11 &
					  \num{11} &
					%--
					  \num[round-mode=places,round-precision=2]{3,93} &
					    \num[round-mode=places,round-precision=2]{0,04} \\
							%????

					8 &
				% TODO try size/length gt 0; take over for other passages
					\multicolumn{1}{X}{ August   } &


					%30 &
					  \num{30} &
					%--
					  \num[round-mode=places,round-precision=2]{10,71} &
					    \num[round-mode=places,round-precision=2]{0,11} \\
							%????

					9 &
				% TODO try size/length gt 0; take over for other passages
					\multicolumn{1}{X}{ September   } &


					%43 &
					  \num{43} &
					%--
					  \num[round-mode=places,round-precision=2]{15,36} &
					    \num[round-mode=places,round-precision=2]{0,15} \\
							%????

					10 &
				% TODO try size/length gt 0; take over for other passages
					\multicolumn{1}{X}{ Oktober   } &


					%84 &
					  \num{84} &
					%--
					  \num[round-mode=places,round-precision=2]{30} &
					    \num[round-mode=places,round-precision=2]{0,3} \\
							%????

					11 &
				% TODO try size/length gt 0; take over for other passages
					\multicolumn{1}{X}{ November   } &


					%20 &
					  \num{20} &
					%--
					  \num[round-mode=places,round-precision=2]{7,14} &
					    \num[round-mode=places,round-precision=2]{0,07} \\
							%????

					12 &
				% TODO try size/length gt 0; take over for other passages
					\multicolumn{1}{X}{ Dezember   } &


					%9 &
					  \num{9} &
					%--
					  \num[round-mode=places,round-precision=2]{3,21} &
					    \num[round-mode=places,round-precision=2]{0,03} \\
							%????
						%DIFFERENT OBSERVATIONS >20
					\midrule
					\multicolumn{2}{l}{Summe (gültig)} &
					  \textbf{\num{280}} &
					\textbf{100} &
					  \textbf{\num[round-mode=places,round-precision=2]{0,99}} \\
					%--
					\multicolumn{5}{l}{\textbf{Fehlende Werte}}\\
							-998 &
							keine Angabe &
							  \num{3391} &
							 - &
							  \num[round-mode=places,round-precision=2]{12,03} \\
							-995 &
							keine Teilnahme (Panel) &
							  \num{24511} &
							 - &
							  \num[round-mode=places,round-precision=2]{86,97} \\
					\midrule
					\multicolumn{2}{l}{\textbf{Summe (gesamt)}} &
				      \textbf{\num{28182}} &
				    \textbf{-} &
				    \textbf{100} \\
					\bottomrule
					\end{longtable}
					\end{filecontents}
					\LTXtable{\textwidth}{\jobname-cbeg08_g2}
				\label{tableValues:cbeg08_g2}
				\vspace*{-\baselineskip}
                    \begin{noten}
                	    \note{} Deskritive Maßzahlen:
                	    Anzahl unterschiedlicher Beobachtungen: 12%
                	    ; 
                	      Minimum ($min$): 1; 
                	      Maximum ($max$): 12; 
                	      Median ($\tilde{x}$): 9; 
                	      Modus ($h$): 10
                     \end{noten}


