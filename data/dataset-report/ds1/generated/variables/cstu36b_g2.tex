%EVERY VARIABLE HAS IT'S OWN PAGE

    \setcounter{footnote}{0}

    %omit vertical space
    \vspace*{-1.8cm}
	\section{cstu36b\_g2 (künftiges Studium: 2. Hauptfach (Studienbereich))}
	\label{section:cstu36b_g2}



	%TABLE FOR VARIABLE DETAILS
    \vspace*{0.5cm}
    \noindent\textbf{Eigenschaften
	% '#' has to be escaped
	\footnote{Detailliertere Informationen zur Variable finden sich unter
		\url{https://metadata.fdz.dzhw.eu/\#!/de/variables/var-gsl2008-ds1-cstu36b_g2$}}}\\
	\begin{tabularx}{\hsize}{@{}lX}
	Datentyp: & numerisch \\
	Skalenniveau: & nominal \\
	Zugangswege: &
	  remote-desktop-suf, 
	  onsite-suf
 \\
    \end{tabularx}



    %TABLE FOR QUESTION DETAILS
    %This has to be tested and has to be improved
    %rausfinden, ob einer Variable mehrere Fragen zugeordnet werden
    %dann evtl. nur die erste verwenden oder etwas anderes tun (Hinweis mehrere Fragen, auflisten mit Link)
				%TABLE FOR QUESTION DETAILS
				\vspace*{0.5cm}
                \noindent\textbf{Frage
	                \footnote{Detailliertere Informationen zur Frage finden sich unter
		              \url{https://metadata.fdz.dzhw.eu/\#!/de/questions/que-gsl2008-ins3-4.19$}}}\\
				\begin{tabularx}{\hsize}{@{}lX}
					Fragenummer: &
					  Fragebogen des DZHW-Studienberechtigtenpanels 2008 - dritte Welle:
					  4.19
 \\
					%--
					Fragetext: & Welches Fach wollen Sie studieren? \\
				\end{tabularx}





				%TABLE FOR THE NOMINAL / ORDINAL VALUES
        		\vspace*{0.5cm}
                \noindent\textbf{Häufigkeiten}

                \vspace*{-\baselineskip}
					%NUMERIC ELEMENTS NEED A HUGH SECOND COLOUMN AND A SMALL FIRST ONE
					\begin{filecontents}{\jobname-cstu36b_g2}
					\begin{longtable}{lXrrr}
					\toprule
					\textbf{Wert} & \textbf{Label} & \textbf{Häufigkeit} & \textbf{Prozent(gültig)} & \textbf{Prozent} \\
					\endhead
					\midrule
					\multicolumn{5}{l}{\textbf{Gültige Werte}}\\
						%DIFFERENT OBSERVATIONS <=20
								1 & \multicolumn{1}{X}{Sprach- und Kulturwissenschaften allgemein} & %2 &
								  \num{2} &
								%--
								  \num[round-mode=places,round-precision=2]{0,6} &
								  \num[round-mode=places,round-precision=2]{0,01} \\
								2 & \multicolumn{1}{X}{Evang. Theologie, -Religionslehre} & %11 &
								  \num{11} &
								%--
								  \num[round-mode=places,round-precision=2]{3,27} &
								  \num[round-mode=places,round-precision=2]{0,04} \\
								3 & \multicolumn{1}{X}{Kath. Theologie, Religionslehre} & %6 &
								  \num{6} &
								%--
								  \num[round-mode=places,round-precision=2]{1,79} &
								  \num[round-mode=places,round-precision=2]{0,02} \\
								4 & \multicolumn{1}{X}{Philosophie} & %11 &
								  \num{11} &
								%--
								  \num[round-mode=places,round-precision=2]{3,27} &
								  \num[round-mode=places,round-precision=2]{0,04} \\
								5 & \multicolumn{1}{X}{Geschichte} & %14 &
								  \num{14} &
								%--
								  \num[round-mode=places,round-precision=2]{4,17} &
								  \num[round-mode=places,round-precision=2]{0,05} \\
								6 & \multicolumn{1}{X}{Bibliothekswissenschaft, Dokumentation Publizistik} & %3 &
								  \num{3} &
								%--
								  \num[round-mode=places,round-precision=2]{0,89} &
								  \num[round-mode=places,round-precision=2]{0,01} \\
								7 & \multicolumn{1}{X}{Allgemeine und vergleichende Literatur- und Sprachwissenschaft} & %4 &
								  \num{4} &
								%--
								  \num[round-mode=places,round-precision=2]{1,19} &
								  \num[round-mode=places,round-precision=2]{0,01} \\
								8 & \multicolumn{1}{X}{Altphilologie (klass. Phililologie), Neugriechisch} & %5 &
								  \num{5} &
								%--
								  \num[round-mode=places,round-precision=2]{1,49} &
								  \num[round-mode=places,round-precision=2]{0,02} \\
								9 & \multicolumn{1}{X}{Germanistik (Deutsch, germanische Sprachen ohne Anglistik)} & %28 &
								  \num{28} &
								%--
								  \num[round-mode=places,round-precision=2]{8,33} &
								  \num[round-mode=places,round-precision=2]{0,1} \\
								10 & \multicolumn{1}{X}{Anglistik, Amerikanistik} & %17 &
								  \num{17} &
								%--
								  \num[round-mode=places,round-precision=2]{5,06} &
								  \num[round-mode=places,round-precision=2]{0,06} \\
							... & ... & ... & ... & ... \\
								60 & \multicolumn{1}{X}{Ernährungs- und Haushaltswissenschaft} & %1 &
								  \num{1} &
								%--
								  \num[round-mode=places,round-precision=2]{0,3} &
								  \num[round-mode=places,round-precision=2]{0} \\

								63 & \multicolumn{1}{X}{Maschinenbau/Verfahrenstechnik} & %12 &
								  \num{12} &
								%--
								  \num[round-mode=places,round-precision=2]{3,57} &
								  \num[round-mode=places,round-precision=2]{0,04} \\

								64 & \multicolumn{1}{X}{Elektrotechnik} & %3 &
								  \num{3} &
								%--
								  \num[round-mode=places,round-precision=2]{0,89} &
								  \num[round-mode=places,round-precision=2]{0,01} \\

								65 & \multicolumn{1}{X}{Verkehrstechnick, Nautik} & %1 &
								  \num{1} &
								%--
								  \num[round-mode=places,round-precision=2]{0,3} &
								  \num[round-mode=places,round-precision=2]{0} \\

								67 & \multicolumn{1}{X}{Raumplanung} & %1 &
								  \num{1} &
								%--
								  \num[round-mode=places,round-precision=2]{0,3} &
								  \num[round-mode=places,round-precision=2]{0} \\

								68 & \multicolumn{1}{X}{Bauningenieurwesen} & %5 &
								  \num{5} &
								%--
								  \num[round-mode=places,round-precision=2]{1,49} &
								  \num[round-mode=places,round-precision=2]{0,02} \\

								74 & \multicolumn{1}{X}{Kunst, Kunstwissenschaft allgemein} & %5 &
								  \num{5} &
								%--
								  \num[round-mode=places,round-precision=2]{1,49} &
								  \num[round-mode=places,round-precision=2]{0,02} \\

								75 & \multicolumn{1}{X}{Bildende Kunst} & %1 &
								  \num{1} &
								%--
								  \num[round-mode=places,round-precision=2]{0,3} &
								  \num[round-mode=places,round-precision=2]{0} \\

								77 & \multicolumn{1}{X}{Darstellende Kunst, Film und Fernsehen, Theaterwissenschaft} & %1 &
								  \num{1} &
								%--
								  \num[round-mode=places,round-precision=2]{0,3} &
								  \num[round-mode=places,round-precision=2]{0} \\

								78 & \multicolumn{1}{X}{Musik, Musikwissenschaft} & %5 &
								  \num{5} &
								%--
								  \num[round-mode=places,round-precision=2]{1,49} &
								  \num[round-mode=places,round-precision=2]{0,02} \\

					\midrule
					\multicolumn{2}{l}{Summe (gültig)} &
					  \textbf{\num{336}} &
					\textbf{100} &
					  \textbf{\num[round-mode=places,round-precision=2]{1,19}} \\
					%--
					\multicolumn{5}{l}{\textbf{Fehlende Werte}}\\
							-998 &
							keine Angabe &
							  \num{1} &
							 - &
							  \num[round-mode=places,round-precision=2]{0} \\
							-995 &
							keine Teilnahme (Panel) &
							  \num{24511} &
							 - &
							  \num[round-mode=places,round-precision=2]{86,97} \\
							-989 &
							filterbedingt fehlend &
							  \num{3333} &
							 - &
							  \num[round-mode=places,round-precision=2]{11,83} \\
							-969 &
							unbekannter fehlender Wert &
							  \num{1} &
							 - &
							  \num[round-mode=places,round-precision=2]{0} \\
					\midrule
					\multicolumn{2}{l}{\textbf{Summe (gesamt)}} &
				      \textbf{\num{28182}} &
				    \textbf{-} &
				    \textbf{100} \\
					\bottomrule
					\end{longtable}
					\end{filecontents}
					\LTXtable{\textwidth}{\jobname-cstu36b_g2}
				\label{tableValues:cstu36b_g2}
				\vspace*{-\baselineskip}
                    \begin{noten}
                	    \note{} Deskritive Maßzahlen:
                	    Anzahl unterschiedlicher Beobachtungen: 46%
                	    ; 
                	      Modus ($h$): 30
                     \end{noten}


