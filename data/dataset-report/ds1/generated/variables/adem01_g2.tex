%EVERY VARIABLE HAS IT'S OWN PAGE

    \setcounter{footnote}{0}

    %omit vertical space
    \vspace*{-1.8cm}
	\section{adem01\_g2 (Geburtsmonat)}
	\label{section:adem01_g2}



	%TABLE FOR VARIABLE DETAILS
    \vspace*{0.5cm}
    \noindent\textbf{Eigenschaften
	% '#' has to be escaped
	\footnote{Detailliertere Informationen zur Variable finden sich unter
		\url{https://metadata.fdz.dzhw.eu/\#!/de/variables/var-gsl2008-ds1-adem01_g2$}}}\\
	\begin{tabularx}{\hsize}{@{}lX}
	Datentyp: & numerisch \\
	Skalenniveau: & ordinal \\
	Zugangswege: &
	  remote-desktop-suf, 
	  onsite-suf
 \\
    \end{tabularx}



    %TABLE FOR QUESTION DETAILS
    %This has to be tested and has to be improved
    %rausfinden, ob einer Variable mehrere Fragen zugeordnet werden
    %dann evtl. nur die erste verwenden oder etwas anderes tun (Hinweis mehrere Fragen, auflisten mit Link)
				%TABLE FOR QUESTION DETAILS
				\vspace*{0.5cm}
                \noindent\textbf{Frage
	                \footnote{Detailliertere Informationen zur Frage finden sich unter
		              \url{https://metadata.fdz.dzhw.eu/\#!/de/questions/que-gsl2008-ins1-28$}}}\\
				\begin{tabularx}{\hsize}{@{}lX}
					Fragenummer: &
					  Fragebogen des DZHW-Studienberechtigtenpanels 2008 - erste Welle:
					  28
 \\
					%--
					Fragetext: & Ihr Geburtsjahr und Ihr Geburtsmonat (Monat) \\
				\end{tabularx}





				%TABLE FOR THE NOMINAL / ORDINAL VALUES
        		\vspace*{0.5cm}
                \noindent\textbf{Häufigkeiten}

                \vspace*{-\baselineskip}
					%NUMERIC ELEMENTS NEED A HUGH SECOND COLOUMN AND A SMALL FIRST ONE
					\begin{filecontents}{\jobname-adem01_g2}
					\begin{longtable}{lXrrr}
					\toprule
					\textbf{Wert} & \textbf{Label} & \textbf{Häufigkeit} & \textbf{Prozent(gültig)} & \textbf{Prozent} \\
					\endhead
					\midrule
					\multicolumn{5}{l}{\textbf{Gültige Werte}}\\
						%DIFFERENT OBSERVATIONS <=20

					1 &
				% TODO try size/length gt 0; take over for other passages
					\multicolumn{1}{X}{ Januar   } &


					%2254 &
					  \num{2254} &
					%--
					  \num[round-mode=places,round-precision=2]{8,24} &
					    \num[round-mode=places,round-precision=2]{8} \\
							%????

					2 &
				% TODO try size/length gt 0; take over for other passages
					\multicolumn{1}{X}{ Februar   } &


					%2110 &
					  \num{2110} &
					%--
					  \num[round-mode=places,round-precision=2]{7,72} &
					    \num[round-mode=places,round-precision=2]{7,49} \\
							%????

					3 &
				% TODO try size/length gt 0; take over for other passages
					\multicolumn{1}{X}{ März   } &


					%2328 &
					  \num{2328} &
					%--
					  \num[round-mode=places,round-precision=2]{8,51} &
					    \num[round-mode=places,round-precision=2]{8,26} \\
							%????

					4 &
				% TODO try size/length gt 0; take over for other passages
					\multicolumn{1}{X}{ April   } &


					%2185 &
					  \num{2185} &
					%--
					  \num[round-mode=places,round-precision=2]{7,99} &
					    \num[round-mode=places,round-precision=2]{7,75} \\
							%????

					5 &
				% TODO try size/length gt 0; take over for other passages
					\multicolumn{1}{X}{ Mai   } &


					%2316 &
					  \num{2316} &
					%--
					  \num[round-mode=places,round-precision=2]{8,47} &
					    \num[round-mode=places,round-precision=2]{8,22} \\
							%????

					6 &
				% TODO try size/length gt 0; take over for other passages
					\multicolumn{1}{X}{ Juni   } &


					%2234 &
					  \num{2234} &
					%--
					  \num[round-mode=places,round-precision=2]{8,17} &
					    \num[round-mode=places,round-precision=2]{7,93} \\
							%????

					7 &
				% TODO try size/length gt 0; take over for other passages
					\multicolumn{1}{X}{ Juli   } &


					%2539 &
					  \num{2539} &
					%--
					  \num[round-mode=places,round-precision=2]{9,29} &
					    \num[round-mode=places,round-precision=2]{9,01} \\
							%????

					8 &
				% TODO try size/length gt 0; take over for other passages
					\multicolumn{1}{X}{ August   } &


					%2388 &
					  \num{2388} &
					%--
					  \num[round-mode=places,round-precision=2]{8,73} &
					    \num[round-mode=places,round-precision=2]{8,47} \\
							%????

					9 &
				% TODO try size/length gt 0; take over for other passages
					\multicolumn{1}{X}{ September   } &


					%2343 &
					  \num{2343} &
					%--
					  \num[round-mode=places,round-precision=2]{8,57} &
					    \num[round-mode=places,round-precision=2]{8,31} \\
							%????

					10 &
				% TODO try size/length gt 0; take over for other passages
					\multicolumn{1}{X}{ Oktober   } &


					%2285 &
					  \num{2285} &
					%--
					  \num[round-mode=places,round-precision=2]{8,36} &
					    \num[round-mode=places,round-precision=2]{8,11} \\
							%????

					11 &
				% TODO try size/length gt 0; take over for other passages
					\multicolumn{1}{X}{ November   } &


					%2101 &
					  \num{2101} &
					%--
					  \num[round-mode=places,round-precision=2]{7,68} &
					    \num[round-mode=places,round-precision=2]{7,46} \\
							%????

					12 &
				% TODO try size/length gt 0; take over for other passages
					\multicolumn{1}{X}{ Dezember   } &


					%2257 &
					  \num{2257} &
					%--
					  \num[round-mode=places,round-precision=2]{8,26} &
					    \num[round-mode=places,round-precision=2]{8,01} \\
							%????
						%DIFFERENT OBSERVATIONS >20
					\midrule
					\multicolumn{2}{l}{Summe (gültig)} &
					  \textbf{\num{27340}} &
					\textbf{100} &
					  \textbf{\num[round-mode=places,round-precision=2]{97,01}} \\
					%--
					\multicolumn{5}{l}{\textbf{Fehlende Werte}}\\
							-998 &
							keine Angabe &
							  \num{802} &
							 - &
							  \num[round-mode=places,round-precision=2]{2,85} \\
							-968 &
							unplausibler Wert &
							  \num{40} &
							 - &
							  \num[round-mode=places,round-precision=2]{0,14} \\
					\midrule
					\multicolumn{2}{l}{\textbf{Summe (gesamt)}} &
				      \textbf{\num{28182}} &
				    \textbf{-} &
				    \textbf{100} \\
					\bottomrule
					\end{longtable}
					\end{filecontents}
					\LTXtable{\textwidth}{\jobname-adem01_g2}
				\label{tableValues:adem01_g2}
				\vspace*{-\baselineskip}
                    \begin{noten}
                	    \note{} Deskritive Maßzahlen:
                	    Anzahl unterschiedlicher Beobachtungen: 12%
                	    ; 
                	      Minimum ($min$): 1; 
                	      Maximum ($max$): 12; 
                	      Median ($\tilde{x}$): 7; 
                	      Modus ($h$): 7
                     \end{noten}


