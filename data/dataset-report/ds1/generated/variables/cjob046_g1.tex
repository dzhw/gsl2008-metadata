%EVERY VARIABLE HAS IT'S OWN PAGE

    \setcounter{footnote}{0}

    %omit vertical space
    \vspace*{-1.8cm}
	\section{cjob046\_g1 (6. Tätigkeit: Beruf Erwerbstätigkeit (KldB-1992-Berufsklasse) (4-Steller))}
	\label{section:cjob046_g1}



	%TABLE FOR VARIABLE DETAILS
    \vspace*{0.5cm}
    \noindent\textbf{Eigenschaften
	% '#' has to be escaped
	\footnote{Detailliertere Informationen zur Variable finden sich unter
		\url{https://metadata.fdz.dzhw.eu/\#!/de/variables/var-gsl2008-ds1-cjob046_g1$}}}\\
	\begin{tabularx}{\hsize}{@{}lX}
	Datentyp: & numerisch \\
	Skalenniveau: & nominal \\
	Zugangswege: &
	  remote-desktop-suf, 
	  onsite-suf
 \\
    \end{tabularx}



    %TABLE FOR QUESTION DETAILS
    %This has to be tested and has to be improved
    %rausfinden, ob einer Variable mehrere Fragen zugeordnet werden
    %dann evtl. nur die erste verwenden oder etwas anderes tun (Hinweis mehrere Fragen, auflisten mit Link)
				%TABLE FOR QUESTION DETAILS
				\vspace*{0.5cm}
                \noindent\textbf{Frage
	                \footnote{Detailliertere Informationen zur Frage finden sich unter
		              \url{https://metadata.fdz.dzhw.eu/\#!/de/questions/que-gsl2008-ins3-2.1$}}}\\
				\begin{tabularx}{\hsize}{@{}lX}
					Fragenummer: &
					  Fragebogen des DZHW-Studienberechtigtenpanels 2008 - dritte Welle:
					  2.1
 \\
					%--
					Fragetext: & Wir bitten Sie nun, uns in dem folgenden Schema einen Überblick Ihres Werdegangs von Juli 2008 bis Dezember 2012 zu geben.\par  Studium\par  Name und Ort der Hochschule/Berufsakademie\par  (z. B. "Uni Köln", "FH Merseburg" oder "BA Mosbach") \\
				\end{tabularx}





				%TABLE FOR THE NOMINAL / ORDINAL VALUES
        		\vspace*{0.5cm}
                \noindent\textbf{Häufigkeiten}

                \vspace*{-\baselineskip}
					%NUMERIC ELEMENTS NEED A HUGH SECOND COLOUMN AND A SMALL FIRST ONE
					\begin{filecontents}{\jobname-cjob046_g1}
					\begin{longtable}{lXrrr}
					\toprule
					\textbf{Wert} & \textbf{Label} & \textbf{Häufigkeit} & \textbf{Prozent(gültig)} & \textbf{Prozent} \\
					\endhead
					\midrule
					\multicolumn{5}{l}{\textbf{Gültige Werte}}\\
						%DIFFERENT OBSERVATIONS <=20
								238 & \multicolumn{1}{X}{Pferdewirt(e/innen) (Pferdezucht,-haltung)} & %2 &
								  \num{2} &
								%--
								  \num[round-mode=places,round-precision=2]{2,41} &
								  \num[round-mode=places,round-precision=2]{0,01} \\
								511 & \multicolumn{1}{X}{Landschaftsgärtner/innen} & %1 &
								  \num{1} &
								%--
								  \num[round-mode=places,round-precision=2]{1,2} &
								  \num[round-mode=places,round-precision=2]{0} \\
								2200 & \multicolumn{1}{X}{Zerspanungsmechaniker/innen o.n.F.} & %1 &
								  \num{1} &
								%--
								  \num[round-mode=places,round-precision=2]{1,2} &
								  \num[round-mode=places,round-precision=2]{0} \\
								2700 & \multicolumn{1}{X}{Industriemechaniker/innen o.n.F., Mechaniker/innen o.n.A.} & %1 &
								  \num{1} &
								%--
								  \num[round-mode=places,round-precision=2]{1,2} &
								  \num[round-mode=places,round-precision=2]{0} \\
								3221 & \multicolumn{1}{X}{Montierer/innen o.n.A.} & %1 &
								  \num{1} &
								%--
								  \num[round-mode=places,round-precision=2]{1,2} &
								  \num[round-mode=places,round-precision=2]{0} \\
								3513 & \multicolumn{1}{X}{Damenschneider/innen} & %1 &
								  \num{1} &
								%--
								  \num[round-mode=places,round-precision=2]{1,2} &
								  \num[round-mode=places,round-precision=2]{0} \\
								5010 & \multicolumn{1}{X}{Tischler/innen, allgemein} & %1 &
								  \num{1} &
								%--
								  \num[round-mode=places,round-precision=2]{1,2} &
								  \num[round-mode=places,round-precision=2]{0} \\
								5220 & \multicolumn{1}{X}{Warenaufmacher/innen, Versandfertigmacher/innen, allgemein} & %1 &
								  \num{1} &
								%--
								  \num[round-mode=places,round-precision=2]{1,2} &
								  \num[round-mode=places,round-precision=2]{0} \\
								6001 & \multicolumn{1}{X}{Forschungs-, Entwicklungs-, Versuchsingenieur(e/innen) o.n.F.} & %1 &
								  \num{1} &
								%--
								  \num[round-mode=places,round-precision=2]{1,2} &
								  \num[round-mode=places,round-precision=2]{0} \\
								6009 & \multicolumn{1}{X}{andere Ingenieur(e/innen) o.n.F.} & %1 &
								  \num{1} &
								%--
								  \num[round-mode=places,round-precision=2]{1,2} &
								  \num[round-mode=places,round-precision=2]{0} \\
							... & ... & ... & ... & ... \\
								8610 & \multicolumn{1}{X}{Sozialarbeiter/innen, Sozialpädagog(en/innen) o.n.A.} & %5 &
								  \num{5} &
								%--
								  \num[round-mode=places,round-precision=2]{6,02} &
								  \num[round-mode=places,round-precision=2]{0,02} \\

								8611 & \multicolumn{1}{X}{- in der Arbeit mit Kindern, Jugendlichen und Familien} & %1 &
								  \num{1} &
								%--
								  \num[round-mode=places,round-precision=2]{1,2} &
								  \num[round-mode=places,round-precision=2]{0} \\

								8630 & \multicolumn{1}{X}{Erzieher/innen o.n.A.} & %2 &
								  \num{2} &
								%--
								  \num[round-mode=places,round-precision=2]{2,41} &
								  \num[round-mode=places,round-precision=2]{0,01} \\

								8633 & \multicolumn{1}{X}{Heimerzieher/innen} & %1 &
								  \num{1} &
								%--
								  \num[round-mode=places,round-precision=2]{1,2} &
								  \num[round-mode=places,round-precision=2]{0} \\

								8694 & \multicolumn{1}{X}{Tagesmütter, -väter} & %1 &
								  \num{1} &
								%--
								  \num[round-mode=places,round-precision=2]{1,2} &
								  \num[round-mode=places,round-precision=2]{0} \\

								8702 & \multicolumn{1}{X}{Lehramtsanwärter/innen o.n.A.} & %1 &
								  \num{1} &
								%--
								  \num[round-mode=places,round-precision=2]{1,2} &
								  \num[round-mode=places,round-precision=2]{0} \\

								8733 & \multicolumn{1}{X}{Realschullehrer/innen} & %1 &
								  \num{1} &
								%--
								  \num[round-mode=places,round-precision=2]{1,2} &
								  \num[round-mode=places,round-precision=2]{0} \\

								8792 & \multicolumn{1}{X}{Freizeitlehrer/innen, -pädagog(en/innen)} & %1 &
								  \num{1} &
								%--
								  \num[round-mode=places,round-precision=2]{1,2} &
								  \num[round-mode=places,round-precision=2]{0} \\

								8841 & \multicolumn{1}{X}{Diplom-Sozialwirt(e/innen) o.n.A.} & %1 &
								  \num{1} &
								%--
								  \num[round-mode=places,round-precision=2]{1,2} &
								  \num[round-mode=places,round-precision=2]{0} \\

								9120 & \multicolumn{1}{X}{Restaurantfachleute, Kellner/innen, allgemein} & %1 &
								  \num{1} &
								%--
								  \num[round-mode=places,round-precision=2]{1,2} &
								  \num[round-mode=places,round-precision=2]{0} \\

					\midrule
					\multicolumn{2}{l}{Summe (gültig)} &
					  \textbf{\num{83}} &
					\textbf{100} &
					  \textbf{\num[round-mode=places,round-precision=2]{0,29}} \\
					%--
					\multicolumn{5}{l}{\textbf{Fehlende Werte}}\\
							-998 &
							keine Angabe &
							  \num{3588} &
							 - &
							  \num[round-mode=places,round-precision=2]{12,73} \\
							-995 &
							keine Teilnahme (Panel) &
							  \num{24511} &
							 - &
							  \num[round-mode=places,round-precision=2]{86,97} \\
					\midrule
					\multicolumn{2}{l}{\textbf{Summe (gesamt)}} &
				      \textbf{\num{28182}} &
				    \textbf{-} &
				    \textbf{100} \\
					\bottomrule
					\end{longtable}
					\end{filecontents}
					\LTXtable{\textwidth}{\jobname-cjob046_g1}
				\label{tableValues:cjob046_g1}
				\vspace*{-\baselineskip}
                    \begin{noten}
                	    \note{} Deskritive Maßzahlen:
                	    Anzahl unterschiedlicher Beobachtungen: 66%
                	    ; 
                	      Modus ($h$): 8610
                     \end{noten}


