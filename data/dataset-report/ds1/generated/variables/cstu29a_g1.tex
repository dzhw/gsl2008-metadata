%EVERY VARIABLE HAS IT'S OWN PAGE

    \setcounter{footnote}{0}

    %omit vertical space
    \vspace*{-1.8cm}
	\section{cstu29a\_g1 (Wechsel Hochschule: von Bundesland der HS)}
	\label{section:cstu29a_g1}



	%TABLE FOR VARIABLE DETAILS
    \vspace*{0.5cm}
    \noindent\textbf{Eigenschaften
	% '#' has to be escaped
	\footnote{Detailliertere Informationen zur Variable finden sich unter
		\url{https://metadata.fdz.dzhw.eu/\#!/de/variables/var-gsl2008-ds1-cstu29a_g1$}}}\\
	\begin{tabularx}{\hsize}{@{}lX}
	Datentyp: & numerisch \\
	Skalenniveau: & nominal \\
	Zugangswege: &
	  remote-desktop-suf, 
	  onsite-suf
 \\
    \end{tabularx}



    %TABLE FOR QUESTION DETAILS
    %This has to be tested and has to be improved
    %rausfinden, ob einer Variable mehrere Fragen zugeordnet werden
    %dann evtl. nur die erste verwenden oder etwas anderes tun (Hinweis mehrere Fragen, auflisten mit Link)
				%TABLE FOR QUESTION DETAILS
				\vspace*{0.5cm}
                \noindent\textbf{Frage
	                \footnote{Detailliertere Informationen zur Frage finden sich unter
		              \url{https://metadata.fdz.dzhw.eu/\#!/de/questions/que-gsl2008-ins3-4.8$}}}\\
				\begin{tabularx}{\hsize}{@{}lX}
					Fragenummer: &
					  Fragebogen des DZHW-Studienberechtigtenpanels 2008 - dritte Welle:
					  4.8
 \\
					%--
					Fragetext: & Von welcher Hochschule zu welcher Hochschule haben Sie gewechselt bzw. werden Sie wechseln? \\
				\end{tabularx}





				%TABLE FOR THE NOMINAL / ORDINAL VALUES
        		\vspace*{0.5cm}
                \noindent\textbf{Häufigkeiten}

                \vspace*{-\baselineskip}
					%NUMERIC ELEMENTS NEED A HUGH SECOND COLOUMN AND A SMALL FIRST ONE
					\begin{filecontents}{\jobname-cstu29a_g1}
					\begin{longtable}{lXrrr}
					\toprule
					\textbf{Wert} & \textbf{Label} & \textbf{Häufigkeit} & \textbf{Prozent(gültig)} & \textbf{Prozent} \\
					\endhead
					\midrule
					\multicolumn{5}{l}{\textbf{Gültige Werte}}\\
						%DIFFERENT OBSERVATIONS <=20

					1 &
				% TODO try size/length gt 0; take over for other passages
					\multicolumn{1}{X}{ Schleswig-Holstein   } &


					%12 &
					  \num{12} &
					%--
					  \num[round-mode=places,round-precision=2]{2,01} &
					    \num[round-mode=places,round-precision=2]{0,04} \\
							%????

					2 &
				% TODO try size/length gt 0; take over for other passages
					\multicolumn{1}{X}{ Hamburg   } &


					%24 &
					  \num{24} &
					%--
					  \num[round-mode=places,round-precision=2]{4,02} &
					    \num[round-mode=places,round-precision=2]{0,09} \\
							%????

					3 &
				% TODO try size/length gt 0; take over for other passages
					\multicolumn{1}{X}{ Niedersachsen   } &


					%43 &
					  \num{43} &
					%--
					  \num[round-mode=places,round-precision=2]{7,2} &
					    \num[round-mode=places,round-precision=2]{0,15} \\
							%????

					4 &
				% TODO try size/length gt 0; take over for other passages
					\multicolumn{1}{X}{ Bremen   } &


					%8 &
					  \num{8} &
					%--
					  \num[round-mode=places,round-precision=2]{1,34} &
					    \num[round-mode=places,round-precision=2]{0,03} \\
							%????

					5 &
				% TODO try size/length gt 0; take over for other passages
					\multicolumn{1}{X}{ Nordrhein-Westfalen   } &


					%107 &
					  \num{107} &
					%--
					  \num[round-mode=places,round-precision=2]{17,92} &
					    \num[round-mode=places,round-precision=2]{0,38} \\
							%????

					6 &
				% TODO try size/length gt 0; take over for other passages
					\multicolumn{1}{X}{ Hessen   } &


					%39 &
					  \num{39} &
					%--
					  \num[round-mode=places,round-precision=2]{6,53} &
					    \num[round-mode=places,round-precision=2]{0,14} \\
							%????

					7 &
				% TODO try size/length gt 0; take over for other passages
					\multicolumn{1}{X}{ Rheinland-Pfalz   } &


					%44 &
					  \num{44} &
					%--
					  \num[round-mode=places,round-precision=2]{7,37} &
					    \num[round-mode=places,round-precision=2]{0,16} \\
							%????

					8 &
				% TODO try size/length gt 0; take over for other passages
					\multicolumn{1}{X}{ Baden-Württemberg   } &


					%110 &
					  \num{110} &
					%--
					  \num[round-mode=places,round-precision=2]{18,43} &
					    \num[round-mode=places,round-precision=2]{0,39} \\
							%????

					9 &
				% TODO try size/length gt 0; take over for other passages
					\multicolumn{1}{X}{ Bayern   } &


					%65 &
					  \num{65} &
					%--
					  \num[round-mode=places,round-precision=2]{10,89} &
					    \num[round-mode=places,round-precision=2]{0,23} \\
							%????

					10 &
				% TODO try size/length gt 0; take over for other passages
					\multicolumn{1}{X}{ Saarland   } &


					%3 &
					  \num{3} &
					%--
					  \num[round-mode=places,round-precision=2]{0,5} &
					    \num[round-mode=places,round-precision=2]{0,01} \\
							%????

					11 &
				% TODO try size/length gt 0; take over for other passages
					\multicolumn{1}{X}{ Berlin   } &


					%24 &
					  \num{24} &
					%--
					  \num[round-mode=places,round-precision=2]{4,02} &
					    \num[round-mode=places,round-precision=2]{0,09} \\
							%????

					12 &
				% TODO try size/length gt 0; take over for other passages
					\multicolumn{1}{X}{ Brandenburg   } &


					%21 &
					  \num{21} &
					%--
					  \num[round-mode=places,round-precision=2]{3,52} &
					    \num[round-mode=places,round-precision=2]{0,07} \\
							%????

					13 &
				% TODO try size/length gt 0; take over for other passages
					\multicolumn{1}{X}{ Mecklenburg-Vorpommern   } &


					%16 &
					  \num{16} &
					%--
					  \num[round-mode=places,round-precision=2]{2,68} &
					    \num[round-mode=places,round-precision=2]{0,06} \\
							%????

					14 &
				% TODO try size/length gt 0; take over for other passages
					\multicolumn{1}{X}{ Sachsen   } &


					%30 &
					  \num{30} &
					%--
					  \num[round-mode=places,round-precision=2]{5,03} &
					    \num[round-mode=places,round-precision=2]{0,11} \\
							%????

					15 &
				% TODO try size/length gt 0; take over for other passages
					\multicolumn{1}{X}{ Sachsen-Anhalt   } &


					%22 &
					  \num{22} &
					%--
					  \num[round-mode=places,round-precision=2]{3,69} &
					    \num[round-mode=places,round-precision=2]{0,08} \\
							%????

					16 &
				% TODO try size/length gt 0; take over for other passages
					\multicolumn{1}{X}{ Thüringen   } &


					%29 &
					  \num{29} &
					%--
					  \num[round-mode=places,round-precision=2]{4,86} &
					    \num[round-mode=places,round-precision=2]{0,1} \\
							%????
						%DIFFERENT OBSERVATIONS >20
					\midrule
					\multicolumn{2}{l}{Summe (gültig)} &
					  \textbf{\num{597}} &
					\textbf{100} &
					  \textbf{\num[round-mode=places,round-precision=2]{2,12}} \\
					%--
					\multicolumn{5}{l}{\textbf{Fehlende Werte}}\\
							-999 &
							weiß nicht &
							  \num{1} &
							 - &
							  \num[round-mode=places,round-precision=2]{0} \\
							-998 &
							keine Angabe &
							  \num{1} &
							 - &
							  \num[round-mode=places,round-precision=2]{0} \\
							-995 &
							keine Teilnahme (Panel) &
							  \num{24511} &
							 - &
							  \num[round-mode=places,round-precision=2]{86,97} \\
							-989 &
							filterbedingt fehlend &
							  \num{3037} &
							 - &
							  \num[round-mode=places,round-precision=2]{10,78} \\
							-988 &
							trifft nicht zu &
							  \num{33} &
							 - &
							  \num[round-mode=places,round-precision=2]{0,12} \\
							-969 &
							unbekannter fehlender Wert &
							  \num{2} &
							 - &
							  \num[round-mode=places,round-precision=2]{0,01} \\
					\midrule
					\multicolumn{2}{l}{\textbf{Summe (gesamt)}} &
				      \textbf{\num{28182}} &
				    \textbf{-} &
				    \textbf{100} \\
					\bottomrule
					\end{longtable}
					\end{filecontents}
					\LTXtable{\textwidth}{\jobname-cstu29a_g1}
				\label{tableValues:cstu29a_g1}
				\vspace*{-\baselineskip}
                    \begin{noten}
                	    \note{} Deskritive Maßzahlen:
                	    Anzahl unterschiedlicher Beobachtungen: 16%
                	    ; 
                	      Modus ($h$): 8
                     \end{noten}


