%EVERY VARIABLE HAS IT'S OWN PAGE

    \setcounter{footnote}{0}

    %omit vertical space
    \vspace*{-1.8cm}
	\section{bstu07a\_g2 (1. Nennung: Studienfach (Studienbereich))}
	\label{section:bstu07a_g2}



	%TABLE FOR VARIABLE DETAILS
    \vspace*{0.5cm}
    \noindent\textbf{Eigenschaften
	% '#' has to be escaped
	\footnote{Detailliertere Informationen zur Variable finden sich unter
		\url{https://metadata.fdz.dzhw.eu/\#!/de/variables/var-gsl2008-ds1-bstu07a_g2$}}}\\
	\begin{tabularx}{\hsize}{@{}lX}
	Datentyp: & numerisch \\
	Skalenniveau: & nominal \\
	Zugangswege: &
	  remote-desktop-suf, 
	  onsite-suf
 \\
    \end{tabularx}



    %TABLE FOR QUESTION DETAILS
    %This has to be tested and has to be improved
    %rausfinden, ob einer Variable mehrere Fragen zugeordnet werden
    %dann evtl. nur die erste verwenden oder etwas anderes tun (Hinweis mehrere Fragen, auflisten mit Link)
				%TABLE FOR QUESTION DETAILS
				\vspace*{0.5cm}
                \noindent\textbf{Frage
	                \footnote{Detailliertere Informationen zur Frage finden sich unter
		              \url{https://metadata.fdz.dzhw.eu/\#!/de/questions/que-gsl2008-ins2-18$}}}\\
				\begin{tabularx}{\hsize}{@{}lX}
					Fragenummer: &
					  Fragebogen des DZHW-Studienberechtigtenpanels 2008 - zweite Welle:
					  18
 \\
					%--
					Fragetext: & Für welchen nächsten Schritt Ihres nachschulischen Werdegangs haben Sie sich entschieden? \\
				\end{tabularx}





				%TABLE FOR THE NOMINAL / ORDINAL VALUES
        		\vspace*{0.5cm}
                \noindent\textbf{Häufigkeiten}

                \vspace*{-\baselineskip}
					%NUMERIC ELEMENTS NEED A HUGH SECOND COLOUMN AND A SMALL FIRST ONE
					\begin{filecontents}{\jobname-bstu07a_g2}
					\begin{longtable}{lXrrr}
					\toprule
					\textbf{Wert} & \textbf{Label} & \textbf{Häufigkeit} & \textbf{Prozent(gültig)} & \textbf{Prozent} \\
					\endhead
					\midrule
					\multicolumn{5}{l}{\textbf{Gültige Werte}}\\
						%DIFFERENT OBSERVATIONS <=20
								1 & \multicolumn{1}{X}{Sprach- und Kulturwissenschaften allgemein} & %1 &
								  \num{1} &
								%--
								  \num[round-mode=places,round-precision=2]{1,61} &
								  \num[round-mode=places,round-precision=2]{0} \\
								4 & \multicolumn{1}{X}{Philosophie} & %1 &
								  \num{1} &
								%--
								  \num[round-mode=places,round-precision=2]{1,61} &
								  \num[round-mode=places,round-precision=2]{0} \\
								6 & \multicolumn{1}{X}{Bibliothekswissenschaft, Dokumentation Publizistik} & %1 &
								  \num{1} &
								%--
								  \num[round-mode=places,round-precision=2]{1,61} &
								  \num[round-mode=places,round-precision=2]{0} \\
								9 & \multicolumn{1}{X}{Germanistik (Deutsch, germanische Sprachen ohne Anglistik)} & %2 &
								  \num{2} &
								%--
								  \num[round-mode=places,round-precision=2]{3,23} &
								  \num[round-mode=places,round-precision=2]{0,01} \\
								10 & \multicolumn{1}{X}{Anglistik, Amerikanistik} & %1 &
								  \num{1} &
								%--
								  \num[round-mode=places,round-precision=2]{1,61} &
								  \num[round-mode=places,round-precision=2]{0} \\
								11 & \multicolumn{1}{X}{Romanistik} & %1 &
								  \num{1} &
								%--
								  \num[round-mode=places,round-precision=2]{1,61} &
								  \num[round-mode=places,round-precision=2]{0} \\
								15 & \multicolumn{1}{X}{Psychologie} & %1 &
								  \num{1} &
								%--
								  \num[round-mode=places,round-precision=2]{1,61} &
								  \num[round-mode=places,round-precision=2]{0} \\
								16 & \multicolumn{1}{X}{Erziehungswissenschaften} & %4 &
								  \num{4} &
								%--
								  \num[round-mode=places,round-precision=2]{6,45} &
								  \num[round-mode=places,round-precision=2]{0,01} \\
								17 & \multicolumn{1}{X}{Sonderpädagogik} & %2 &
								  \num{2} &
								%--
								  \num[round-mode=places,round-precision=2]{3,23} &
								  \num[round-mode=places,round-precision=2]{0,01} \\
								25 & \multicolumn{1}{X}{Politikwissenschaften} & %1 &
								  \num{1} &
								%--
								  \num[round-mode=places,round-precision=2]{1,61} &
								  \num[round-mode=places,round-precision=2]{0} \\
							... & ... & ... & ... & ... \\
								42 & \multicolumn{1}{X}{Biologie} & %1 &
								  \num{1} &
								%--
								  \num[round-mode=places,round-precision=2]{1,61} &
								  \num[round-mode=places,round-precision=2]{0} \\

								48 & \multicolumn{1}{X}{Gesundheitswissenschaften allgemein} & %2 &
								  \num{2} &
								%--
								  \num[round-mode=places,round-precision=2]{3,23} &
								  \num[round-mode=places,round-precision=2]{0,01} \\

								49 & \multicolumn{1}{X}{Humanmedizin (ohne Zahnmedizin)} & %4 &
								  \num{4} &
								%--
								  \num[round-mode=places,round-precision=2]{6,45} &
								  \num[round-mode=places,round-precision=2]{0,01} \\

								51 & \multicolumn{1}{X}{Veterinärmedizin} & %1 &
								  \num{1} &
								%--
								  \num[round-mode=places,round-precision=2]{1,61} &
								  \num[round-mode=places,round-precision=2]{0} \\

								58 & \multicolumn{1}{X}{Agrarwissenschaften, Lebensmittel- und Getränketechnologie} & %1 &
								  \num{1} &
								%--
								  \num[round-mode=places,round-precision=2]{1,61} &
								  \num[round-mode=places,round-precision=2]{0} \\

								63 & \multicolumn{1}{X}{Maschinenbau/Verfahrenstechnik} & %4 &
								  \num{4} &
								%--
								  \num[round-mode=places,round-precision=2]{6,45} &
								  \num[round-mode=places,round-precision=2]{0,01} \\

								65 & \multicolumn{1}{X}{Verkehrstechnick, Nautik} & %1 &
								  \num{1} &
								%--
								  \num[round-mode=places,round-precision=2]{1,61} &
								  \num[round-mode=places,round-precision=2]{0} \\

								66 & \multicolumn{1}{X}{Architektur, Innenarchitektur} & %1 &
								  \num{1} &
								%--
								  \num[round-mode=places,round-precision=2]{1,61} &
								  \num[round-mode=places,round-precision=2]{0} \\

								74 & \multicolumn{1}{X}{Kunst, Kunstwissenschaft allgemein} & %1 &
								  \num{1} &
								%--
								  \num[round-mode=places,round-precision=2]{1,61} &
								  \num[round-mode=places,round-precision=2]{0} \\

								76 & \multicolumn{1}{X}{Gestaltung} & %3 &
								  \num{3} &
								%--
								  \num[round-mode=places,round-precision=2]{4,84} &
								  \num[round-mode=places,round-precision=2]{0,01} \\

					\midrule
					\multicolumn{2}{l}{Summe (gültig)} &
					  \textbf{\num{62}} &
					\textbf{100} &
					  \textbf{\num[round-mode=places,round-precision=2]{0,22}} \\
					%--
					\multicolumn{5}{l}{\textbf{Fehlende Werte}}\\
							-998 &
							keine Angabe &
							  \num{18} &
							 - &
							  \num[round-mode=places,round-precision=2]{0,06} \\
							-995 &
							keine Teilnahme (Panel) &
							  \num{22249} &
							 - &
							  \num[round-mode=places,round-precision=2]{78,95} \\
							-989 &
							filterbedingt fehlend &
							  \num{5821} &
							 - &
							  \num[round-mode=places,round-precision=2]{20,66} \\
							-969 &
							unbekannter fehlender Wert &
							  \num{32} &
							 - &
							  \num[round-mode=places,round-precision=2]{0,11} \\
					\midrule
					\multicolumn{2}{l}{\textbf{Summe (gesamt)}} &
				      \textbf{\num{28182}} &
				    \textbf{-} &
				    \textbf{100} \\
					\bottomrule
					\end{longtable}
					\end{filecontents}
					\LTXtable{\textwidth}{\jobname-bstu07a_g2}
				\label{tableValues:bstu07a_g2}
				\vspace*{-\baselineskip}
                    \begin{noten}
                	    \note{} Deskritive Maßzahlen:
                	    Anzahl unterschiedlicher Beobachtungen: 26%
                	    ; 
                	      Modus ($h$): 30
                     \end{noten}


