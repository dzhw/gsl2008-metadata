%EVERY VARIABLE HAS IT'S OWN PAGE

    \setcounter{footnote}{0}

    %omit vertical space
    \vspace*{-1.8cm}
	\section{bstu07b\_g1 (2. Nennung Idee: Studienfach)}
	\label{section:bstu07b_g1}



	%TABLE FOR VARIABLE DETAILS
    \vspace*{0.5cm}
    \noindent\textbf{Eigenschaften
	% '#' has to be escaped
	\footnote{Detailliertere Informationen zur Variable finden sich unter
		\url{https://metadata.fdz.dzhw.eu/\#!/de/variables/var-gsl2008-ds1-bstu07b_g1$}}}\\
	\begin{tabularx}{\hsize}{@{}lX}
	Datentyp: & numerisch \\
	Skalenniveau: & nominal \\
	Zugangswege: &
	  remote-desktop-suf, 
	  onsite-suf
 \\
    \end{tabularx}



    %TABLE FOR QUESTION DETAILS
    %This has to be tested and has to be improved
    %rausfinden, ob einer Variable mehrere Fragen zugeordnet werden
    %dann evtl. nur die erste verwenden oder etwas anderes tun (Hinweis mehrere Fragen, auflisten mit Link)
				%TABLE FOR QUESTION DETAILS
				\vspace*{0.5cm}
                \noindent\textbf{Frage
	                \footnote{Detailliertere Informationen zur Frage finden sich unter
		              \url{https://metadata.fdz.dzhw.eu/\#!/de/questions/que-gsl2008-ins2-18$}}}\\
				\begin{tabularx}{\hsize}{@{}lX}
					Fragenummer: &
					  Fragebogen des DZHW-Studienberechtigtenpanels 2008 - zweite Welle:
					  18
 \\
					%--
					Fragetext: & Für welchen nächsten Schritt Ihres nachschulischen Werdegangs haben Sie sich entschieden?\par  ich habe mich noch nicht endgültig entschieden, werde aber wahrscheinlich … \\
				\end{tabularx}





				%TABLE FOR THE NOMINAL / ORDINAL VALUES
        		\vspace*{0.5cm}
                \noindent\textbf{Häufigkeiten}

                \vspace*{-\baselineskip}
					%NUMERIC ELEMENTS NEED A HUGH SECOND COLOUMN AND A SMALL FIRST ONE
					\begin{filecontents}{\jobname-bstu07b_g1}
					\begin{longtable}{lXrrr}
					\toprule
					\textbf{Wert} & \textbf{Label} & \textbf{Häufigkeit} & \textbf{Prozent(gültig)} & \textbf{Prozent} \\
					\endhead
					\midrule
					\multicolumn{5}{l}{\textbf{Gültige Werte}}\\
						%DIFFERENT OBSERVATIONS <=20
								1 & \multicolumn{1}{X}{Ägyptologie} & %1 &
								  \num{1} &
								%--
								  \num[round-mode=places,round-precision=2]{2,38} &
								  \num[round-mode=places,round-precision=2]{0} \\
								17 & \multicolumn{1}{X}{Bauingenieurwesen/Ingenieurbau} & %1 &
								  \num{1} &
								%--
								  \num[round-mode=places,round-precision=2]{2,38} &
								  \num[round-mode=places,round-precision=2]{0} \\
								21 & \multicolumn{1}{X}{Betriebswirtschaftslehre} & %4 &
								  \num{4} &
								%--
								  \num[round-mode=places,round-precision=2]{9,52} &
								  \num[round-mode=places,round-precision=2]{0,01} \\
								26 & \multicolumn{1}{X}{Biologie} & %1 &
								  \num{1} &
								%--
								  \num[round-mode=places,round-precision=2]{2,38} &
								  \num[round-mode=places,round-precision=2]{0} \\
								35 & \multicolumn{1}{X}{Darstellende Kunst/Bühnenkunst/Regie} & %1 &
								  \num{1} &
								%--
								  \num[round-mode=places,round-precision=2]{2,38} &
								  \num[round-mode=places,round-precision=2]{0} \\
								50 & \multicolumn{1}{X}{Geographie/Erdkunde} & %1 &
								  \num{1} &
								%--
								  \num[round-mode=places,round-precision=2]{2,38} &
								  \num[round-mode=places,round-precision=2]{0} \\
								52 & \multicolumn{1}{X}{Erziehungswissenschaft (Pädagogik)} & %1 &
								  \num{1} &
								%--
								  \num[round-mode=places,round-precision=2]{2,38} &
								  \num[round-mode=places,round-precision=2]{0} \\
								57 & \multicolumn{1}{X}{Luft- und Raumfahrttechnik} & %1 &
								  \num{1} &
								%--
								  \num[round-mode=places,round-precision=2]{2,38} &
								  \num[round-mode=places,round-precision=2]{0} \\
								68 & \multicolumn{1}{X}{Geschichte} & %1 &
								  \num{1} &
								%--
								  \num[round-mode=places,round-precision=2]{2,38} &
								  \num[round-mode=places,round-precision=2]{0} \\
								79 & \multicolumn{1}{X}{Informatik} & %2 &
								  \num{2} &
								%--
								  \num[round-mode=places,round-precision=2]{4,76} &
								  \num[round-mode=places,round-precision=2]{0,01} \\
							... & ... & ... & ... & ... \\
								181 & \multicolumn{1}{X}{Wirtschaftspädagogik} & %1 &
								  \num{1} &
								%--
								  \num[round-mode=places,round-precision=2]{2,38} &
								  \num[round-mode=places,round-precision=2]{0} \\

								184 & \multicolumn{1}{X}{Wirtschaftswissenschaften} & %4 &
								  \num{4} &
								%--
								  \num[round-mode=places,round-precision=2]{9,52} &
								  \num[round-mode=places,round-precision=2]{0,01} \\

								185 & \multicolumn{1}{X}{Zahnmedizin} & %1 &
								  \num{1} &
								%--
								  \num[round-mode=places,round-precision=2]{2,38} &
								  \num[round-mode=places,round-precision=2]{0} \\

								188 & \multicolumn{1}{X}{Allgemeine Literaturwissenschaft} & %1 &
								  \num{1} &
								%--
								  \num[round-mode=places,round-precision=2]{2,38} &
								  \num[round-mode=places,round-precision=2]{0} \\

								190 & \multicolumn{1}{X}{Sonderpädagogik} & %1 &
								  \num{1} &
								%--
								  \num[round-mode=places,round-precision=2]{2,38} &
								  \num[round-mode=places,round-precision=2]{0} \\

								208 & \multicolumn{1}{X}{Soziale Arbeit} & %1 &
								  \num{1} &
								%--
								  \num[round-mode=places,round-precision=2]{2,38} &
								  \num[round-mode=places,round-precision=2]{0} \\

								245 & \multicolumn{1}{X}{Sozialpädagogik} & %4 &
								  \num{4} &
								%--
								  \num[round-mode=places,round-precision=2]{9,52} &
								  \num[round-mode=places,round-precision=2]{0,01} \\

								282 & \multicolumn{1}{X}{Biotechnologie} & %1 &
								  \num{1} &
								%--
								  \num[round-mode=places,round-precision=2]{2,38} &
								  \num[round-mode=places,round-precision=2]{0} \\

								320 & \multicolumn{1}{X}{Ernährungswissenschaft} & %1 &
								  \num{1} &
								%--
								  \num[round-mode=places,round-precision=2]{2,38} &
								  \num[round-mode=places,round-precision=2]{0} \\

								566 & \multicolumn{1}{X}{Sozialwirtschaft} & %1 &
								  \num{1} &
								%--
								  \num[round-mode=places,round-precision=2]{2,38} &
								  \num[round-mode=places,round-precision=2]{0} \\

					\midrule
					\multicolumn{2}{l}{Summe (gültig)} &
					  \textbf{\num{42}} &
					\textbf{100} &
					  \textbf{\num[round-mode=places,round-precision=2]{0,15}} \\
					%--
					\multicolumn{5}{l}{\textbf{Fehlende Werte}}\\
							-998 &
							keine Angabe &
							  \num{17} &
							 - &
							  \num[round-mode=places,round-precision=2]{0,06} \\
							-995 &
							keine Teilnahme (Panel) &
							  \num{22249} &
							 - &
							  \num[round-mode=places,round-precision=2]{78,95} \\
							-989 &
							filterbedingt fehlend &
							  \num{5821} &
							 - &
							  \num[round-mode=places,round-precision=2]{20,66} \\
							-969 &
							unbekannter fehlender Wert &
							  \num{53} &
							 - &
							  \num[round-mode=places,round-precision=2]{0,19} \\
					\midrule
					\multicolumn{2}{l}{\textbf{Summe (gesamt)}} &
				      \textbf{\num{28182}} &
				    \textbf{-} &
				    \textbf{100} \\
					\bottomrule
					\end{longtable}
					\end{filecontents}
					\LTXtable{\textwidth}{\jobname-bstu07b_g1}
				\label{tableValues:bstu07b_g1}
				\vspace*{-\baselineskip}
                    \begin{noten}
                	    \note{} Deskritive Maßzahlen:
                	    Anzahl unterschiedlicher Beobachtungen: 30%
                	    ; 
                	      Modus ($h$): multimodal
                     \end{noten}


