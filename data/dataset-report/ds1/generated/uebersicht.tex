%% LaTeX2e file `uebersicht.tex'
%% generated by the `filecontents' environment
%% from source `Main' on 2018/03/20.
%%
\begin{longtable}{P{12em}Q}\toprule
  \textbf{Studienreihe}
     & DZHW-Absolventenstudien\\\midrule
  \textbf{Kohorte}
     & Absol\-vent(inn)en\-kohorte 2005 (5. Kohorte der Studienreihe)\\\midrule
  \textbf{Erhebende Institution}
     & Deutsches Zentrum für Hochschul- und Wissenschaftsforschung (DZHW)\\\midrule
  \textbf{Förderung}
     & Bundesministerium für Bildung und Forschung (BMBF)\\\midrule
  \textbf{Projektmitarbeiter(innen)}\par \textbf{(\underline{Projektleitung})}
     & \underline{Kolja Briedis}, Michael Grotheer, Sören Isleib, \underline{Karl-Heinz Minks}, Nicolai Netz\\\midrule
  \textbf{Themen}
     & Studienverlauf\par Übergang in den Beruf\par Beruf"|licher Werdegang\par Weiterqualifizierung\\\midrule
  \textbf{Erhebungsdesign}
     & Kohorten-Panel-Design\\\midrule
  \textbf{Grundgesamtheit}
     & Hochschulabsolvent(inn)en, die im Wintersemester 2004\slash2005 oder im Sommersemester 2005 ihren ersten berufsqualifizierenden Studienabschluss an einer staatlich anerkannten Hochschule in der Bundesrepublik Deutschland erworben haben (mit Ausnahme der Absolvent(inn)en von Bundeswehrhochschulen, Verwaltungsfachhochschulen, Berufsakademien und Fernhochschulen)\\\midrule
  \textbf{Stichproben}           & Absolvent(inn)en traditioneller Studiengänge:\par quotierte geschichtete Klumpenstichprobe\par Absolvent(inn)en aus Bachelor-Studiengängen:\par bewusste Auswahl\\\midrule
  \textbf{Erhebungsmethode}
     & \href{https://www.google.de/search?q=Standardisiert & spell=1 & sa=X & ved=0ahUKEwiTk9W1jNLLAhUhA3MKHXpnAQUQvwUIGigA}{Standardisiert}e postalische Befragung\\\midrule
  \textbf{Erhebungszeitraum}
     & 1. Welle: Januar 2006 bis Mai 2007\par 2. Welle: Dezember 2010 bis September 2011\\\midrule
  \textbf{Auswertbare Fälle} & 1. Welle: n = 11.788 (davon 1.622 Bachelor-Absolvent(inn)en)\par 2. Welle: n = 6.459 (davon 797 Bachelor-Absolvent(inn)en)\\\midrule
  \textbf{Rücklaufquote}
     & 1. Welle: 24,7\,\%\par 2. Welle: 60,3\,\%\\\midrule
  \textbf{Datenprodukte und}\par \textbf{Zugangswege}
     & CUF: Download\par SUF: Download, Remote-Desktop, On-Site\\\midrule
  \textbf{Datensatzstruktur}
 & Personendatensätze im wide-Format\par Episodendatensätze im long-Format\\\midrule
  \textbf{Besonderheiten der Daten}
 & Getrennte Datensätze für Absolvent(inn)en traditioneller Studiengänge und Bachelorabsolvent(inn)en wegen unterschiedlicher Stichprobenziehung\\\midrule
  \textbf{DOI}
 & xx.xxxx/FDZ-DZHW:gds2005:x.x.x.\\\midrule
  \textbf{Weitere Informationen}
 & \url{https://fdz.dzhw.eu}\\\midrule
  \multicolumn{2}{p{\dimexpr\hsize-2\tabcolsep}}{\textbf{Projektpublikationen*}\par Briedis, K. (2007). \textit{Übergänge und Erfahrungen nach dem Hochschulabschluss. Ergebnisse der HIS-Absolventenbefragung 2005} (HIS: Forum Hochschule 13/2007). Hannover: HIS.\par Briedis, K. \& Minks, K.-H. (2007). \textit{Generation Praktikum. Mythos oder Massenphänomen}. Hannover: HIS.\par Grotheer, M., Isleib, S., Netz, N. \& Briedis, K. (2012). \textit{Hochqualifiziert und gefragt. Ergebnisse der zweiten HIS-HF Absolventenbefragung des Jahrgangs 2005} (HIS: Forum Hochschule 14/2012). Hannover: HIS.\par {\small* Alle Projektpublikationen werden auf der Website des Projektes (\href{http://www.dzhw.eu/projekte/pr_show?pr_id=298}{http://www.dzhw.eu/projekte/pr\_show?pr\_id=298}) zum Download bereitgestellt.\par}~\vspace{-\baselineskip}}\\\midrule
  \multicolumn{2}{p{\dimexpr\hsize-2\tabcolsep}}{\textbf{Publikationen zum Datensatz (Auswahl)}\par Schaeper, H. (2009). Development of competencies and teaching--learning arrangements in higher education: findings from Germany. \textit{Studies in Higher Education, 34} (6), 677--697. \doi{doi:10.1080/03075070802669207}\par Jaksztat, S. (2014). Bildungsherkunft und Promotionen: Wie beeinflusst das elterliche Bildungsniveau den Übergang in die Promotionsphase? \textit{Zeitschrift für Soziologie, 43} (4), 286--301.\par Schaeper, H., Grotheer, M. \& Brandt, G. (2014). Familiengründung von Hochschulabsolventinnen. Eine empirische Untersuchung verschiedener Examenskohorten. In D. Konietzka \& M. Kreyenfeld (Hrsg.), \textit{Ein Leben ohne Kinder} (2. Aufl., S.~47--80). Wiesbaden: Springer VS. \doi{doi:10.1007/978-3-531-94149-3\_2}\par Kratz, F. \& Netz, N. (2016). Which mechanisms explain monetary returns to international student mobility? \textit{Studies in Higher Education.} \doi{doi:10.1080/03075079.2016.1172307}}\\\midrule
  \end{longtable}
