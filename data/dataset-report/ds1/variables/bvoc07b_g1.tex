%EVERY VARIABLE HAS IT'S OWN PAGE

    \setcounter{footnote}{0}

    %omit vertical space
    \vspace*{-1.8cm}
	\section{bvoc07b\_g1 (Alternative Qualifizierung: Ausbildungsberuf (KldB-1992-Berufsordnung) (3-Stelle)}
	\label{section:bvoc07b_g1}



	% TABLE FOR VARIABLE DETAILS
  % '#' has to be escaped
    \vspace*{0.5cm}
    \noindent\textbf{Eigenschaften\footnote{Detailliertere Informationen zur Variable finden sich unter
		\url{https://metadata.fdz.dzhw.eu/\#!/de/variables/var-gsl2008-ds1-bvoc07b_g1$}}}\\
	\begin{tabularx}{\hsize}{@{}lX}
	Datentyp: & numerisch \\
	Skalenniveau: & nominal \\
	Zugangswege: &
	  remote-desktop-suf, 
	  onsite-suf
 \\
    \end{tabularx}



    %TABLE FOR QUESTION DETAILS
    %This has to be tested and has to be improved
    %rausfinden, ob einer Variable mehrere Fragen zugeordnet werden
    %dann evtl. nur die erste verwenden oder etwas anderes tun (Hinweis mehrere Fragen, auflisten mit Link)
				%TABLE FOR QUESTION DETAILS
				\vspace*{0.5cm}
                \noindent\textbf{Frage\footnote{Detailliertere Informationen zur Frage finden sich unter
		              \url{https://metadata.fdz.dzhw.eu/\#!/de/questions/que-gsl2008-ins2-30$}}}\\
				\begin{tabularx}{\hsize}{@{}lX}
					Fragenummer: &
					  Fragebogen des DZHW-Studienberechtigtenpanels 2008 - zweite Welle:
					  30
 \\
					%--
					Fragetext: & Was werden/wollen Sie statt dessen tun?\par  eine (andere) Berufsausbildung absolvieren, und zwar: \\
				\end{tabularx}





				%TABLE FOR THE NOMINAL / ORDINAL VALUES
        		\vspace*{0.5cm}
                \noindent\textbf{Häufigkeiten}

                \vspace*{-\baselineskip}
					%NUMERIC ELEMENTS NEED A HUGH SECOND COLOUMN AND A SMALL FIRST ONE
					\begin{filecontents}{\jobname-bvoc07b_g1}
					\begin{longtable}{lXrrr}
					\toprule
					\textbf{Wert} & \textbf{Label} & \textbf{Häufigkeit} & \textbf{Prozent(gültig)} & \textbf{Prozent} \\
					\endhead
					\midrule
					\multicolumn{5}{l}{\textbf{Gültige Werte}}\\
						%DIFFERENT OBSERVATIONS <=20

					23 &
				% TODO try size/length gt 0; take over for other passages
					\multicolumn{1}{X}{ Tier-, Pferde-, Fischwirte und -wirtinnen   } &


					%1 &
					  \num{1} &
					%--
					  \num[round-mode=places,round-precision=2]{8.33} &
					    \num[round-mode=places,round-precision=2]{0} \\
							%????

					622 &
				% TODO try size/length gt 0; take over for other passages
					\multicolumn{1}{X}{ Elektrotechniker/Elektrotechnikerinnen   } &


					%1 &
					  \num{1} &
					%--
					  \num[round-mode=places,round-precision=2]{8.33} &
					    \num[round-mode=places,round-precision=2]{0} \\
							%????

					691 &
				% TODO try size/length gt 0; take over for other passages
					\multicolumn{1}{X}{ Bankfachleute   } &


					%1 &
					  \num{1} &
					%--
					  \num[round-mode=places,round-precision=2]{8.33} &
					    \num[round-mode=places,round-precision=2]{0} \\
							%????

					704 &
				% TODO try size/length gt 0; take over for other passages
					\multicolumn{1}{X}{ Handelsmakler/Handelsmaklerinnen, Immobilienkaufleute   } &


					%1 &
					  \num{1} &
					%--
					  \num[round-mode=places,round-precision=2]{8.33} &
					    \num[round-mode=places,round-precision=2]{0} \\
							%????

					754 &
				% TODO try size/length gt 0; take over for other passages
					\multicolumn{1}{X}{ Fachgehilfen/Fachgehilfinnen in steuer- und wirtschaftsberatenden Berufen, Steuerfachleute, a.n.g.   } &


					%1 &
					  \num{1} &
					%--
					  \num[round-mode=places,round-precision=2]{8.33} &
					    \num[round-mode=places,round-precision=2]{0} \\
							%????

					774 &
				% TODO try size/length gt 0; take over for other passages
					\multicolumn{1}{X}{ Datenverarbeitungsfachleute, Informatiker/Informatikerinnen o.n.A.   } &


					%1 &
					  \num{1} &
					%--
					  \num[round-mode=places,round-precision=2]{8.33} &
					    \num[round-mode=places,round-precision=2]{0} \\
							%????

					785 &
				% TODO try size/length gt 0; take over for other passages
					\multicolumn{1}{X}{ Industriekaufleute, Technische Kaufleute, Betriebswirte/Betriebswirtinnen (ohne Diplom), a.n.g.   } &


					%2 &
					  \num{2} &
					%--
					  \num[round-mode=places,round-precision=2]{16.67} &
					    \num[round-mode=places,round-precision=2]{0.01} \\
							%????

					834 &
				% TODO try size/length gt 0; take over for other passages
					\multicolumn{1}{X}{ Bildende Künstler/Künstlerinnen (angewandte Kunst)   } &


					%1 &
					  \num{1} &
					%--
					  \num[round-mode=places,round-precision=2]{8.33} &
					    \num[round-mode=places,round-precision=2]{0} \\
							%????

					853 &
				% TODO try size/length gt 0; take over for other passages
					\multicolumn{1}{X}{ Krankenschwestern/-pfleger, Hebammen/Entbindungspfleger   } &


					%2 &
					  \num{2} &
					%--
					  \num[round-mode=places,round-precision=2]{16.67} &
					    \num[round-mode=places,round-precision=2]{0.01} \\
							%????

					859 &
				% TODO try size/length gt 0; take over for other passages
					\multicolumn{1}{X}{ Therapeutische Berufe, a.n.g.   } &


					%1 &
					  \num{1} &
					%--
					  \num[round-mode=places,round-precision=2]{8.33} &
					    \num[round-mode=places,round-precision=2]{0} \\
							%????
						%DIFFERENT OBSERVATIONS >20
					\midrule
					\multicolumn{2}{l}{Summe (gültig)} &
					  \textbf{\num{12}} &
					\textbf{100} &
					  \textbf{\num[round-mode=places,round-precision=2]{0.04}} \\
					%--
					\multicolumn{5}{l}{\textbf{Fehlende Werte}}\\
							-995 &
							keine Teilnahme (Panel) &
							  \num{22249} &
							 - &
							  \num[round-mode=places,round-precision=2]{78.95} \\
							-989 &
							filterbedingt fehlend &
							  \num{5921} &
							 - &
							  \num[round-mode=places,round-precision=2]{21.01} \\
					\midrule
					\multicolumn{2}{l}{\textbf{Summe (gesamt)}} &
				      \textbf{\num{28182}} &
				    \textbf{-} &
				    \textbf{100} \\
					\bottomrule
					\end{longtable}
					\end{filecontents}
					\LTXtable{\textwidth}{\jobname-bvoc07b_g1}
				\label{tableValues:bvoc07b_g1}
				\vspace*{-\baselineskip}
                    \begin{noten}
                	    \note{} Deskriptive Maßzahlen:
                	    Anzahl unterschiedlicher Beobachtungen: 10%
                	    ; 
                	      Modus ($h$): multimodal
                     \end{noten}


