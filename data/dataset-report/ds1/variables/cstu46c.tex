%EVERY VARIABLE HAS IT'S OWN PAGE

    \setcounter{footnote}{0}

    %omit vertical space
    \vspace*{-1.8cm}
	\section{cstu46c (Grund Studienpause: Termin Studienbeginn)}
	\label{section:cstu46c}



	% TABLE FOR VARIABLE DETAILS
  % '#' has to be escaped
    \vspace*{0.5cm}
    \noindent\textbf{Eigenschaften\footnote{Detailliertere Informationen zur Variable finden sich unter
		\url{https://metadata.fdz.dzhw.eu/\#!/de/variables/var-gsl2008-ds1-cstu46c$}}}\\
	\begin{tabularx}{\hsize}{@{}lX}
	Datentyp: & numerisch \\
	Skalenniveau: & nominal \\
	Zugangswege: &
	  remote-desktop-suf, 
	  onsite-suf
 \\
    \end{tabularx}



    %TABLE FOR QUESTION DETAILS
    %This has to be tested and has to be improved
    %rausfinden, ob einer Variable mehrere Fragen zugeordnet werden
    %dann evtl. nur die erste verwenden oder etwas anderes tun (Hinweis mehrere Fragen, auflisten mit Link)
				%TABLE FOR QUESTION DETAILS
				\vspace*{0.5cm}
                \noindent\textbf{Frage\footnote{Detailliertere Informationen zur Frage finden sich unter
		              \url{https://metadata.fdz.dzhw.eu/\#!/de/questions/que-gsl2008-ins3-5.8$}}}\\
				\begin{tabularx}{\hsize}{@{}lX}
					Fragenummer: &
					  Fragebogen des DZHW-Studienberechtigtenpanels 2008 - dritte Welle:
					  5.8
 \\
					%--
					Fragetext: & Aus welchen Gründen legen Sie diese Studienpause ein?\par  Studienbeginn nur zu einem bestimmten Zeitpunkt möglich \\
				\end{tabularx}





				%TABLE FOR THE NOMINAL / ORDINAL VALUES
        		\vspace*{0.5cm}
                \noindent\textbf{Häufigkeiten}

                \vspace*{-\baselineskip}
					%NUMERIC ELEMENTS NEED A HUGH SECOND COLOUMN AND A SMALL FIRST ONE
					\begin{filecontents}{\jobname-cstu46c}
					\begin{longtable}{lXrrr}
					\toprule
					\textbf{Wert} & \textbf{Label} & \textbf{Häufigkeit} & \textbf{Prozent(gültig)} & \textbf{Prozent} \\
					\endhead
					\midrule
					\multicolumn{5}{l}{\textbf{Gültige Werte}}\\
						%DIFFERENT OBSERVATIONS <=20

					0 &
				% TODO try size/length gt 0; take over for other passages
					\multicolumn{1}{X}{ nicht genannt   } &


					%288 &
					  \num{288} &
					%--
					  \num[round-mode=places,round-precision=2]{73.28} &
					    \num[round-mode=places,round-precision=2]{1.02} \\
							%????

					1 &
				% TODO try size/length gt 0; take over for other passages
					\multicolumn{1}{X}{ genannt   } &


					%105 &
					  \num{105} &
					%--
					  \num[round-mode=places,round-precision=2]{26.72} &
					    \num[round-mode=places,round-precision=2]{0.37} \\
							%????
						%DIFFERENT OBSERVATIONS >20
					\midrule
					\multicolumn{2}{l}{Summe (gültig)} &
					  \textbf{\num{393}} &
					\textbf{\num{100}} &
					  \textbf{\num[round-mode=places,round-precision=2]{1.39}} \\
					%--
					\multicolumn{5}{l}{\textbf{Fehlende Werte}}\\
							-998 &
							keine Angabe &
							  \num{3} &
							 - &
							  \num[round-mode=places,round-precision=2]{0.01} \\
							-995 &
							keine Teilnahme (Panel) &
							  \num{24511} &
							 - &
							  \num[round-mode=places,round-precision=2]{86.97} \\
							-989 &
							filterbedingt fehlend &
							  \num{3275} &
							 - &
							  \num[round-mode=places,round-precision=2]{11.62} \\
					\midrule
					\multicolumn{2}{l}{\textbf{Summe (gesamt)}} &
				      \textbf{\num{28182}} &
				    \textbf{-} &
				    \textbf{\num{100}} \\
					\bottomrule
					\end{longtable}
					\end{filecontents}
					\LTXtable{\textwidth}{\jobname-cstu46c}
				\label{tableValues:cstu46c}
				\vspace*{-\baselineskip}
                    \begin{noten}
                	    \note{} Deskriptive Maßzahlen:
                	    Anzahl unterschiedlicher Beobachtungen: 2%
                	    ; 
                	      Modus ($h$): 0
                     \end{noten}

