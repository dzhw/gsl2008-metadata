%EVERY VARIABLE HAS IT'S OWN PAGE

    \setcounter{footnote}{0}

    %omit vertical space
    \vspace*{-1.8cm}
	\section{bstu15a\_g2 (2. Qualifikation: 1. Hauptstudienfach (Fächergruppe))}
	\label{section:bstu15a_g2}



	% TABLE FOR VARIABLE DETAILS
  % '#' has to be escaped
    \vspace*{0.5cm}
    \noindent\textbf{Eigenschaften\footnote{Detailliertere Informationen zur Variable finden sich unter
		\url{https://metadata.fdz.dzhw.eu/\#!/de/variables/var-gsl2008-ds1-bstu15a_g2$}}}\\
	\begin{tabularx}{\hsize}{@{}lX}
	Datentyp: & numerisch \\
	Skalenniveau: & nominal \\
	Zugangswege: &
	  remote-desktop-suf, 
	  onsite-suf
 \\
    \end{tabularx}



    %TABLE FOR QUESTION DETAILS
    %This has to be tested and has to be improved
    %rausfinden, ob einer Variable mehrere Fragen zugeordnet werden
    %dann evtl. nur die erste verwenden oder etwas anderes tun (Hinweis mehrere Fragen, auflisten mit Link)
				%TABLE FOR QUESTION DETAILS
				\vspace*{0.5cm}
                \noindent\textbf{Frage\footnote{Detailliertere Informationen zur Frage finden sich unter
		              \url{https://metadata.fdz.dzhw.eu/\#!/de/questions/que-gsl2008-ins2-24$}}}\\
				\begin{tabularx}{\hsize}{@{}lX}
					Fragenummer: &
					  Fragebogen des DZHW-Studienberechtigtenpanels 2008 - zweite Welle:
					  24
 \\
					%--
					Fragetext: & Angaben zum möglichen/beabsichtigten Studium: \\
				\end{tabularx}





				%TABLE FOR THE NOMINAL / ORDINAL VALUES
        		\vspace*{0.5cm}
                \noindent\textbf{Häufigkeiten}

                \vspace*{-\baselineskip}
					%NUMERIC ELEMENTS NEED A HUGH SECOND COLOUMN AND A SMALL FIRST ONE
					\begin{filecontents}{\jobname-bstu15a_g2}
					\begin{longtable}{lXrrr}
					\toprule
					\textbf{Wert} & \textbf{Label} & \textbf{Häufigkeit} & \textbf{Prozent(gültig)} & \textbf{Prozent} \\
					\endhead
					\midrule
					\multicolumn{5}{l}{\textbf{Gültige Werte}}\\
						%DIFFERENT OBSERVATIONS <=20

					1 &
				% TODO try size/length gt 0; take over for other passages
					\multicolumn{1}{X}{ Sprach- und Kulturwissenschaften   } &


					%117 &
					  \num{117} &
					%--
					  \num[round-mode=places,round-precision=2]{14.7} &
					    \num[round-mode=places,round-precision=2]{0.42} \\
							%????

					2 &
				% TODO try size/length gt 0; take over for other passages
					\multicolumn{1}{X}{ Sport   } &


					%5 &
					  \num{5} &
					%--
					  \num[round-mode=places,round-precision=2]{0.63} &
					    \num[round-mode=places,round-precision=2]{0.02} \\
							%????

					3 &
				% TODO try size/length gt 0; take over for other passages
					\multicolumn{1}{X}{ Rechts-, Wirtschafts- und Sozialwissenschaften   } &


					%341 &
					  \num{341} &
					%--
					  \num[round-mode=places,round-precision=2]{42.84} &
					    \num[round-mode=places,round-precision=2]{1.21} \\
							%????

					4 &
				% TODO try size/length gt 0; take over for other passages
					\multicolumn{1}{X}{ Mathematik, Naturwissenschaften   } &


					%49 &
					  \num{49} &
					%--
					  \num[round-mode=places,round-precision=2]{6.16} &
					    \num[round-mode=places,round-precision=2]{0.17} \\
							%????

					5 &
				% TODO try size/length gt 0; take over for other passages
					\multicolumn{1}{X}{ Humanmedizin/Gesundheitswissenschaften   } &


					%134 &
					  \num{134} &
					%--
					  \num[round-mode=places,round-precision=2]{16.83} &
					    \num[round-mode=places,round-precision=2]{0.48} \\
							%????

					6 &
				% TODO try size/length gt 0; take over for other passages
					\multicolumn{1}{X}{ Veterinärmedizin   } &


					%13 &
					  \num{13} &
					%--
					  \num[round-mode=places,round-precision=2]{1.63} &
					    \num[round-mode=places,round-precision=2]{0.05} \\
							%????

					7 &
				% TODO try size/length gt 0; take over for other passages
					\multicolumn{1}{X}{ Agrar-, Forst- und Ernährungswissenschaften   } &


					%23 &
					  \num{23} &
					%--
					  \num[round-mode=places,round-precision=2]{2.89} &
					    \num[round-mode=places,round-precision=2]{0.08} \\
							%????

					8 &
				% TODO try size/length gt 0; take over for other passages
					\multicolumn{1}{X}{ Ingenieurwissenschaften   } &


					%81 &
					  \num{81} &
					%--
					  \num[round-mode=places,round-precision=2]{10.18} &
					    \num[round-mode=places,round-precision=2]{0.29} \\
							%????

					9 &
				% TODO try size/length gt 0; take over for other passages
					\multicolumn{1}{X}{ Kunst, Kunstwissenschaft   } &


					%33 &
					  \num{33} &
					%--
					  \num[round-mode=places,round-precision=2]{4.15} &
					    \num[round-mode=places,round-precision=2]{0.12} \\
							%????
						%DIFFERENT OBSERVATIONS >20
					\midrule
					\multicolumn{2}{l}{Summe (gültig)} &
					  \textbf{\num{796}} &
					\textbf{\num{100}} &
					  \textbf{\num[round-mode=places,round-precision=2]{2.82}} \\
					%--
					\multicolumn{5}{l}{\textbf{Fehlende Werte}}\\
							-999 &
							weiß nicht &
							  \num{13} &
							 - &
							  \num[round-mode=places,round-precision=2]{0.05} \\
							-998 &
							keine Angabe &
							  \num{104} &
							 - &
							  \num[round-mode=places,round-precision=2]{0.37} \\
							-995 &
							keine Teilnahme (Panel) &
							  \num{22249} &
							 - &
							  \num[round-mode=places,round-precision=2]{78.95} \\
							-989 &
							filterbedingt fehlend &
							  \num{5020} &
							 - &
							  \num[round-mode=places,round-precision=2]{17.81} \\
					\midrule
					\multicolumn{2}{l}{\textbf{Summe (gesamt)}} &
				      \textbf{\num{28182}} &
				    \textbf{-} &
				    \textbf{\num{100}} \\
					\bottomrule
					\end{longtable}
					\end{filecontents}
					\LTXtable{\textwidth}{\jobname-bstu15a_g2}
				\label{tableValues:bstu15a_g2}
				\vspace*{-\baselineskip}
                    \begin{noten}
                	    \note{} Deskriptive Maßzahlen:
                	    Anzahl unterschiedlicher Beobachtungen: 9%
                	    ; 
                	      Modus ($h$): 3
                     \end{noten}

