%EVERY VARIABLE HAS IT'S OWN PAGE

    \setcounter{footnote}{0}

    %omit vertical space
    \vspace*{-1.8cm}
	\section{cstu2113c (13. Tätigkeit: angestrebter Abschluss)}
	\label{section:cstu2113c}



	% TABLE FOR VARIABLE DETAILS
  % '#' has to be escaped
    \vspace*{0.5cm}
    \noindent\textbf{Eigenschaften\footnote{Detailliertere Informationen zur Variable finden sich unter
		\url{https://metadata.fdz.dzhw.eu/\#!/de/variables/var-gsl2008-ds1-cstu2113c$}}}\\
	\begin{tabularx}{\hsize}{@{}lX}
	Datentyp: & numerisch \\
	Skalenniveau: & nominal \\
	Zugangswege: &
	  remote-desktop-suf, 
	  onsite-suf
 \\
    \end{tabularx}



    %TABLE FOR QUESTION DETAILS
    %This has to be tested and has to be improved
    %rausfinden, ob einer Variable mehrere Fragen zugeordnet werden
    %dann evtl. nur die erste verwenden oder etwas anderes tun (Hinweis mehrere Fragen, auflisten mit Link)
		\vspace*{0.5cm}
		\noindent\textbf{Frage}\\
		Dieser Variable ist keine Frage zugeordnet.





				%TABLE FOR THE NOMINAL / ORDINAL VALUES
        		\vspace*{0.5cm}
                \noindent\textbf{Häufigkeiten}

                \vspace*{-\baselineskip}
					%NUMERIC ELEMENTS NEED A HUGH SECOND COLOUMN AND A SMALL FIRST ONE
					\begin{filecontents}{\jobname-cstu2113c}
					\begin{longtable}{lXrrr}
					\toprule
					\textbf{Wert} & \textbf{Label} & \textbf{Häufigkeit} & \textbf{Prozent(gültig)} & \textbf{Prozent} \\
					\endhead
					\midrule
					\multicolumn{5}{l}{\textbf{Gültige Werte}}\\
						%DIFFERENT OBSERVATIONS <=20

					3 &
				% TODO try size/length gt 0; take over for other passages
					\multicolumn{1}{X}{ B.A. an Fachhochschule   } &


					%1 &
					  \num{1} &
					%--
					  \num[round-mode=places,round-precision=2]{11.11} &
					    \num[round-mode=places,round-precision=2]{0} \\
							%????

					6 &
				% TODO try size/length gt 0; take over for other passages
					\multicolumn{1}{X}{ M.A. an Universität (außer LA)   } &


					%1 &
					  \num{1} &
					%--
					  \num[round-mode=places,round-precision=2]{11.11} &
					    \num[round-mode=places,round-precision=2]{0} \\
							%????

					10 &
				% TODO try size/length gt 0; take over for other passages
					\multicolumn{1}{X}{ LA B.A./M.A. Gymnasium   } &


					%1 &
					  \num{1} &
					%--
					  \num[round-mode=places,round-precision=2]{11.11} &
					    \num[round-mode=places,round-precision=2]{0} \\
							%????

					17 &
				% TODO try size/length gt 0; take over for other passages
					\multicolumn{1}{X}{ Staatsexamen (außer LA)   } &


					%5 &
					  \num{5} &
					%--
					  \num[round-mode=places,round-precision=2]{55.56} &
					    \num[round-mode=places,round-precision=2]{0.02} \\
							%????

					21 &
				% TODO try size/length gt 0; take over for other passages
					\multicolumn{1}{X}{ Abschluss im Ausland   } &


					%1 &
					  \num{1} &
					%--
					  \num[round-mode=places,round-precision=2]{11.11} &
					    \num[round-mode=places,round-precision=2]{0} \\
							%????
						%DIFFERENT OBSERVATIONS >20
					\midrule
					\multicolumn{2}{l}{Summe (gültig)} &
					  \textbf{\num{9}} &
					\textbf{\num{100}} &
					  \textbf{\num[round-mode=places,round-precision=2]{0.03}} \\
					%--
					\multicolumn{5}{l}{\textbf{Fehlende Werte}}\\
							-998 &
							keine Angabe &
							  \num{3662} &
							 - &
							  \num[round-mode=places,round-precision=2]{12.99} \\
							-995 &
							keine Teilnahme (Panel) &
							  \num{24511} &
							 - &
							  \num[round-mode=places,round-precision=2]{86.97} \\
					\midrule
					\multicolumn{2}{l}{\textbf{Summe (gesamt)}} &
				      \textbf{\num{28182}} &
				    \textbf{-} &
				    \textbf{\num{100}} \\
					\bottomrule
					\end{longtable}
					\end{filecontents}
					\LTXtable{\textwidth}{\jobname-cstu2113c}
				\label{tableValues:cstu2113c}
				\vspace*{-\baselineskip}
                    \begin{noten}
                	    \note{} Deskriptive Maßzahlen:
                	    Anzahl unterschiedlicher Beobachtungen: 5%
                	    ; 
                	      Modus ($h$): 17
                     \end{noten}

