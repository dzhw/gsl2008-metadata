%EVERY VARIABLE HAS IT'S OWN PAGE

    \setcounter{footnote}{0}

    %omit vertical space
    \vspace*{-1.8cm}
	\section{asys03\_g2 (Schulart (ermittelt: allgemeinb. vs. berufl. Schulen))}
	\label{section:asys03_g2}



	% TABLE FOR VARIABLE DETAILS
  % '#' has to be escaped
    \vspace*{0.5cm}
    \noindent\textbf{Eigenschaften\footnote{Detailliertere Informationen zur Variable finden sich unter
		\url{https://metadata.fdz.dzhw.eu/\#!/de/variables/var-gsl2008-ds1-asys03_g2$}}}\\
	\begin{tabularx}{\hsize}{@{}lX}
	Datentyp: & numerisch \\
	Skalenniveau: & nominal \\
	Zugangswege: &
	  remote-desktop-suf, 
	  onsite-suf
 \\
    \end{tabularx}



    %TABLE FOR QUESTION DETAILS
    %This has to be tested and has to be improved
    %rausfinden, ob einer Variable mehrere Fragen zugeordnet werden
    %dann evtl. nur die erste verwenden oder etwas anderes tun (Hinweis mehrere Fragen, auflisten mit Link)
		\vspace*{0.5cm}
		\noindent\textbf{Frage}\\
		Dieser Variable ist keine Frage zugeordnet.





				%TABLE FOR THE NOMINAL / ORDINAL VALUES
        		\vspace*{0.5cm}
                \noindent\textbf{Häufigkeiten}

                \vspace*{-\baselineskip}
					%NUMERIC ELEMENTS NEED A HUGH SECOND COLOUMN AND A SMALL FIRST ONE
					\begin{filecontents}{\jobname-asys03_g2}
					\begin{longtable}{lXrrr}
					\toprule
					\textbf{Wert} & \textbf{Label} & \textbf{Häufigkeit} & \textbf{Prozent(gültig)} & \textbf{Prozent} \\
					\endhead
					\midrule
					\multicolumn{5}{l}{\textbf{Gültige Werte}}\\
						%DIFFERENT OBSERVATIONS <=20

					18 &
				% TODO try size/length gt 0; take over for other passages
					\multicolumn{1}{X}{ allgemeinbildene Schulen   } &


					%19782 &
					  \num{19782} &
					%--
					  \num[round-mode=places,round-precision=2]{70.19} &
					    \num[round-mode=places,round-precision=2]{70.19} \\
							%????

					19 &
				% TODO try size/length gt 0; take over for other passages
					\multicolumn{1}{X}{ berufliche Schulen   } &


					%8400 &
					  \num{8400} &
					%--
					  \num[round-mode=places,round-precision=2]{29.81} &
					    \num[round-mode=places,round-precision=2]{29.81} \\
							%????
						%DIFFERENT OBSERVATIONS >20
					\midrule
					\multicolumn{2}{l}{Summe (gültig)} &
					  \textbf{\num{28182}} &
					\textbf{100} &
					  \textbf{\num[round-mode=places,round-precision=2]{100}} \\
					%--
					\multicolumn{5}{l}{\textbf{Fehlende Werte}}\\
						& & 0 & 0 & 0 \\
					\midrule
					\multicolumn{2}{l}{\textbf{Summe (gesamt)}} &
				      \textbf{\num{28182}} &
				    \textbf{-} &
				    \textbf{100} \\
					\bottomrule
					\end{longtable}
					\end{filecontents}
					\LTXtable{\textwidth}{\jobname-asys03_g2}
				\label{tableValues:asys03_g2}
				\vspace*{-\baselineskip}
                    \begin{noten}
                	    \note{} Deskriptive Maßzahlen:
                	    Anzahl unterschiedlicher Beobachtungen: 2%
                	    ; 
                	      Modus ($h$): 18
                     \end{noten}


