%EVERY VARIABLE HAS IT'S OWN PAGE

    \setcounter{footnote}{0}

    %omit vertical space
    \vspace*{-1.8cm}
	\section{cski02g (Veränderung seit Schulzeit: wirtschaftlich)}
	\label{section:cski02g}



	% TABLE FOR VARIABLE DETAILS
  % '#' has to be escaped
    \vspace*{0.5cm}
    \noindent\textbf{Eigenschaften\footnote{Detailliertere Informationen zur Variable finden sich unter
		\url{https://metadata.fdz.dzhw.eu/\#!/de/variables/var-gsl2008-ds1-cski02g$}}}\\
	\begin{tabularx}{\hsize}{@{}lX}
	Datentyp: & numerisch \\
	Skalenniveau: & ordinal \\
	Zugangswege: &
	  remote-desktop-suf, 
	  onsite-suf
 \\
    \end{tabularx}



    %TABLE FOR QUESTION DETAILS
    %This has to be tested and has to be improved
    %rausfinden, ob einer Variable mehrere Fragen zugeordnet werden
    %dann evtl. nur die erste verwenden oder etwas anderes tun (Hinweis mehrere Fragen, auflisten mit Link)
				%TABLE FOR QUESTION DETAILS
				\vspace*{0.5cm}
                \noindent\textbf{Frage\footnote{Detailliertere Informationen zur Frage finden sich unter
		              \url{https://metadata.fdz.dzhw.eu/\#!/de/questions/que-gsl2008-ins3-1.4$}}}\\
				\begin{tabularx}{\hsize}{@{}lX}
					Fragenummer: &
					  Fragebogen des DZHW-Studienberechtigtenpanels 2008 - dritte Welle:
					  1.4
 \\
					%--
					Fragetext: & Kein Mensch ist auf allen Gebieten gleich leistungsstark. In welchen der folgenden Bereiche…\par  (b) haben sich diese seit der Schule verändert?\par  im wirtschaftlichen Bereich \\
				\end{tabularx}





				%TABLE FOR THE NOMINAL / ORDINAL VALUES
        		\vspace*{0.5cm}
                \noindent\textbf{Häufigkeiten}

                \vspace*{-\baselineskip}
					%NUMERIC ELEMENTS NEED A HUGH SECOND COLOUMN AND A SMALL FIRST ONE
					\begin{filecontents}{\jobname-cski02g}
					\begin{longtable}{lXrrr}
					\toprule
					\textbf{Wert} & \textbf{Label} & \textbf{Häufigkeit} & \textbf{Prozent(gültig)} & \textbf{Prozent} \\
					\endhead
					\midrule
					\multicolumn{5}{l}{\textbf{Gültige Werte}}\\
						%DIFFERENT OBSERVATIONS <=20

					-4 &
				% TODO try size/length gt 0; take over for other passages
					\multicolumn{1}{X}{ verschlechtert   } &


					%2 &
					  \num{2} &
					%--
					  \num[round-mode=places,round-precision=2]{0.05} &
					    \num[round-mode=places,round-precision=2]{0.01} \\
							%????

					-3 &
				% TODO try size/length gt 0; take over for other passages
					\multicolumn{1}{X}{ -3   } &


					%25 &
					  \num{25} &
					%--
					  \num[round-mode=places,round-precision=2]{0.68} &
					    \num[round-mode=places,round-precision=2]{0.09} \\
							%????

					-2 &
				% TODO try size/length gt 0; take over for other passages
					\multicolumn{1}{X}{ -2   } &


					%65 &
					  \num{65} &
					%--
					  \num[round-mode=places,round-precision=2]{1.78} &
					    \num[round-mode=places,round-precision=2]{0.23} \\
							%????

					-1 &
				% TODO try size/length gt 0; take over for other passages
					\multicolumn{1}{X}{ -1   } &


					%134 &
					  \num{134} &
					%--
					  \num[round-mode=places,round-precision=2]{3.67} &
					    \num[round-mode=places,round-precision=2]{0.48} \\
							%????

					0 &
				% TODO try size/length gt 0; take over for other passages
					\multicolumn{1}{X}{ gleich geblieben   } &


					%1331 &
					  \num{1331} &
					%--
					  \num[round-mode=places,round-precision=2]{36.46} &
					    \num[round-mode=places,round-precision=2]{4.72} \\
							%????

					1 &
				% TODO try size/length gt 0; take over for other passages
					\multicolumn{1}{X}{ 1   } &


					%674 &
					  \num{674} &
					%--
					  \num[round-mode=places,round-precision=2]{18.46} &
					    \num[round-mode=places,round-precision=2]{2.39} \\
							%????

					2 &
				% TODO try size/length gt 0; take over for other passages
					\multicolumn{1}{X}{ 2   } &


					%600 &
					  \num{600} &
					%--
					  \num[round-mode=places,round-precision=2]{16.43} &
					    \num[round-mode=places,round-precision=2]{2.13} \\
							%????

					3 &
				% TODO try size/length gt 0; take over for other passages
					\multicolumn{1}{X}{ 3   } &


					%453 &
					  \num{453} &
					%--
					  \num[round-mode=places,round-precision=2]{12.41} &
					    \num[round-mode=places,round-precision=2]{1.61} \\
							%????

					4 &
				% TODO try size/length gt 0; take over for other passages
					\multicolumn{1}{X}{ verbessert   } &


					%367 &
					  \num{367} &
					%--
					  \num[round-mode=places,round-precision=2]{10.05} &
					    \num[round-mode=places,round-precision=2]{1.3} \\
							%????
						%DIFFERENT OBSERVATIONS >20
					\midrule
					\multicolumn{2}{l}{Summe (gültig)} &
					  \textbf{\num{3651}} &
					\textbf{\num{100}} &
					  \textbf{\num[round-mode=places,round-precision=2]{12.96}} \\
					%--
					\multicolumn{5}{l}{\textbf{Fehlende Werte}}\\
							-998 &
							keine Angabe &
							  \num{20} &
							 - &
							  \num[round-mode=places,round-precision=2]{0.07} \\
							-995 &
							keine Teilnahme (Panel) &
							  \num{24511} &
							 - &
							  \num[round-mode=places,round-precision=2]{86.97} \\
					\midrule
					\multicolumn{2}{l}{\textbf{Summe (gesamt)}} &
				      \textbf{\num{28182}} &
				    \textbf{-} &
				    \textbf{\num{100}} \\
					\bottomrule
					\end{longtable}
					\end{filecontents}
					\LTXtable{\textwidth}{\jobname-cski02g}
				\label{tableValues:cski02g}
				\vspace*{-\baselineskip}
                    \begin{noten}
                	    \note{} Deskriptive Maßzahlen:
                	    Anzahl unterschiedlicher Beobachtungen: 9%
                	    ; 
                	      Minimum ($min$): -4; 
                	      Maximum ($max$): 4; 
                	      Median ($\tilde{x}$): 1; 
                	      Modus ($h$): 0
                     \end{noten}

