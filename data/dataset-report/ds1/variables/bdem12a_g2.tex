%EVERY VARIABLE HAS IT'S OWN PAGE

    \setcounter{footnote}{0}

    %omit vertical space
    \vspace*{-1.8cm}
	\section{bdem12a\_g2 (Vater: Beruf (KldB-1992-Berufsordnung) (3-Steller))}
	\label{section:bdem12a_g2}



	% TABLE FOR VARIABLE DETAILS
  % '#' has to be escaped
    \vspace*{0.5cm}
    \noindent\textbf{Eigenschaften\footnote{Detailliertere Informationen zur Variable finden sich unter
		\url{https://metadata.fdz.dzhw.eu/\#!/de/variables/var-gsl2008-ds1-bdem12a_g2$}}}\\
	\begin{tabularx}{\hsize}{@{}lX}
	Datentyp: & numerisch \\
	Skalenniveau: & nominal \\
	Zugangswege: &
	  remote-desktop-suf, 
	  onsite-suf
 \\
    \end{tabularx}



    %TABLE FOR QUESTION DETAILS
    %This has to be tested and has to be improved
    %rausfinden, ob einer Variable mehrere Fragen zugeordnet werden
    %dann evtl. nur die erste verwenden oder etwas anderes tun (Hinweis mehrere Fragen, auflisten mit Link)
				%TABLE FOR QUESTION DETAILS
				\vspace*{0.5cm}
                \noindent\textbf{Frage\footnote{Detailliertere Informationen zur Frage finden sich unter
		              \url{https://metadata.fdz.dzhw.eu/\#!/de/questions/que-gsl2008-ins2-44$}}}\\
				\begin{tabularx}{\hsize}{@{}lX}
					Fragenummer: &
					  Fragebogen des DZHW-Studienberechtigtenpanels 2008 - zweite Welle:
					  44
 \\
					%--
					Fragetext: & Welchen Beruf üben/übten Ihre Eltern aktuell bzw. zuletzt hauptberuflich aus? \\
				\end{tabularx}





				%TABLE FOR THE NOMINAL / ORDINAL VALUES
        		\vspace*{0.5cm}
                \noindent\textbf{Häufigkeiten}

                \vspace*{-\baselineskip}
					%NUMERIC ELEMENTS NEED A HUGH SECOND COLOUMN AND A SMALL FIRST ONE
					\begin{filecontents}{\jobname-bdem12a_g2}
					\begin{longtable}{lXrrr}
					\toprule
					\textbf{Wert} & \textbf{Label} & \textbf{Häufigkeit} & \textbf{Prozent(gültig)} & \textbf{Prozent} \\
					\endhead
					\midrule
					\multicolumn{5}{l}{\textbf{Gültige Werte}}\\
						%DIFFERENT OBSERVATIONS <=20
								11 & \multicolumn{1}{X}{Landwirte/Landwirtinnen, Pflanzenschützer/Pflanzenschützerinnen} & %81 &
								  \num{81} &
								%--
								  \num[round-mode=places,round-precision=2]{1.47} &
								  \num[round-mode=places,round-precision=2]{0.29} \\
								12 & \multicolumn{1}{X}{Winzer/Winzerinnen} & %2 &
								  \num{2} &
								%--
								  \num[round-mode=places,round-precision=2]{0.04} &
								  \num[round-mode=places,round-precision=2]{0.01} \\
								13 & \multicolumn{1}{X}{Landarbeitskräfte} & %3 &
								  \num{3} &
								%--
								  \num[round-mode=places,round-precision=2]{0.05} &
								  \num[round-mode=places,round-precision=2]{0.01} \\
								23 & \multicolumn{1}{X}{Tier-, Pferde-, Fischwirte und -wirtinnen} & %3 &
								  \num{3} &
								%--
								  \num[round-mode=places,round-precision=2]{0.05} &
								  \num[round-mode=places,round-precision=2]{0.01} \\
								24 & \multicolumn{1}{X}{Tierpfleger/Tierpflegerinnen und verwandte Berufe, a.n.g.} & %2 &
								  \num{2} &
								%--
								  \num[round-mode=places,round-precision=2]{0.04} &
								  \num[round-mode=places,round-precision=2]{0.01} \\
								31 & \multicolumn{1}{X}{Verwalter/Verwalterinnen in der Land- und Tierwirtschaft} & %1 &
								  \num{1} &
								%--
								  \num[round-mode=places,round-precision=2]{0.02} &
								  \num[round-mode=places,round-precision=2]{0} \\
								32 & \multicolumn{1}{X}{Land-, Tierwirtschaftsberater und -beraterinnen, Agraringenieur/Agraringenieurinnen, Agrartechniker/Agrartechnikerinnen} & %10 &
								  \num{10} &
								%--
								  \num[round-mode=places,round-precision=2]{0.18} &
								  \num[round-mode=places,round-precision=2]{0.04} \\
								51 & \multicolumn{1}{X}{Gärtner/Gärtnerinnen, Gartenarbeiter/Gartenarbeiterinnen} & %36 &
								  \num{36} &
								%--
								  \num[round-mode=places,round-precision=2]{0.65} &
								  \num[round-mode=places,round-precision=2]{0.13} \\
								52 & \multicolumn{1}{X}{Ingenieure/Ingenieurinnen, Techniker/Technikerinnen in Gartenbau und Landespflege} & %3 &
								  \num{3} &
								%--
								  \num[round-mode=places,round-precision=2]{0.05} &
								  \num[round-mode=places,round-precision=2]{0.01} \\
								61 & \multicolumn{1}{X}{Forstverwalter/Forstverwalterinnen, Förster/Försterinnen, Jäger/Jägerinnen} & %18 &
								  \num{18} &
								%--
								  \num[round-mode=places,round-precision=2]{0.33} &
								  \num[round-mode=places,round-precision=2]{0.06} \\
							... & ... & ... & ... & ... \\
								993 & \multicolumn{1}{X}{Vorarbeiter/Vorarbeiterinnen, Gruppenleiter/Gruppenleiterinnen ohne nähere Tätigkeitsangabe} & %7 &
								  \num{7} &
								%--
								  \num[round-mode=places,round-precision=2]{0.13} &
								  \num[round-mode=places,round-precision=2]{0.02} \\

								995 & \multicolumn{1}{X}{Selbständige ohne nähere Tätigkeitsangabe} & %18 &
								  \num{18} &
								%--
								  \num[round-mode=places,round-precision=2]{0.33} &
								  \num[round-mode=places,round-precision=2]{0.06} \\

								996 & \multicolumn{1}{X}{Beratungs-, Planungsfachleute ohne nähere Tätigkeitsangabe} & %5 &
								  \num{5} &
								%--
								  \num[round-mode=places,round-precision=2]{0.09} &
								  \num[round-mode=places,round-precision=2]{0.02} \\

								997 & \multicolumn{1}{X}{Sonstige Arbeitskräfte ohne nähere Tätigkeitsangabe} & %11 &
								  \num{11} &
								%--
								  \num[round-mode=places,round-precision=2]{0.2} &
								  \num[round-mode=places,round-precision=2]{0.04} \\

								9912 & \multicolumn{1}{X}{Student} & %1 &
								  \num{1} &
								%--
								  \num[round-mode=places,round-precision=2]{0.02} &
								  \num[round-mode=places,round-precision=2]{0} \\

								9995 & \multicolumn{1}{X}{krank/arbeitsunfähig} & %4 &
								  \num{4} &
								%--
								  \num[round-mode=places,round-precision=2]{0.07} &
								  \num[round-mode=places,round-precision=2]{0.01} \\

								9996 & \multicolumn{1}{X}{verstorben} & %21 &
								  \num{21} &
								%--
								  \num[round-mode=places,round-precision=2]{0.38} &
								  \num[round-mode=places,round-precision=2]{0.07} \\

								9997 & \multicolumn{1}{X}{Arbeitslos} & %21 &
								  \num{21} &
								%--
								  \num[round-mode=places,round-precision=2]{0.38} &
								  \num[round-mode=places,round-precision=2]{0.07} \\

								9998 & \multicolumn{1}{X}{Rentner} & %23 &
								  \num{23} &
								%--
								  \num[round-mode=places,round-precision=2]{0.42} &
								  \num[round-mode=places,round-precision=2]{0.08} \\

								9999 & \multicolumn{1}{X}{Hausfrau/Hausmann} & %2 &
								  \num{2} &
								%--
								  \num[round-mode=places,round-precision=2]{0.04} &
								  \num[round-mode=places,round-precision=2]{0.01} \\

					\midrule
					\multicolumn{2}{l}{Summe (gültig)} &
					  \textbf{\num{5505}} &
					\textbf{\num{100}} &
					  \textbf{\num[round-mode=places,round-precision=2]{19.53}} \\
					%--
					\multicolumn{5}{l}{\textbf{Fehlende Werte}}\\
							-999 &
							weiß nicht &
							  \num{11} &
							 - &
							  \num[round-mode=places,round-precision=2]{0.04} \\
							-998 &
							keine Angabe &
							  \num{128} &
							 - &
							  \num[round-mode=places,round-precision=2]{0.45} \\
							-995 &
							keine Teilnahme (Panel) &
							  \num{22249} &
							 - &
							  \num[round-mode=places,round-precision=2]{78.95} \\
							-988 &
							trifft nicht zu &
							  \num{1} &
							 - &
							  \num[round-mode=places,round-precision=2]{0} \\
							-969 &
							unbekannter fehlender Wert &
							  \num{288} &
							 - &
							  \num[round-mode=places,round-precision=2]{1.02} \\
					\midrule
					\multicolumn{2}{l}{\textbf{Summe (gesamt)}} &
				      \textbf{\num{28182}} &
				    \textbf{-} &
				    \textbf{\num{100}} \\
					\bottomrule
					\end{longtable}
					\end{filecontents}
					\LTXtable{\textwidth}{\jobname-bdem12a_g2}
				\label{tableValues:bdem12a_g2}
				\vspace*{-\baselineskip}
                    \begin{noten}
                	    \note{} Deskriptive Maßzahlen:
                	    Anzahl unterschiedlicher Beobachtungen: 316%
                	    ; 
                	      Modus ($h$): 750
                     \end{noten}

