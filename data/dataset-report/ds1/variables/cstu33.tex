%EVERY VARIABLE HAS IT'S OWN PAGE

    \setcounter{footnote}{0}

    %omit vertical space
    \vspace*{-1.8cm}
	\section{cstu33 (weiteres Studium künftig beabsichtigt?)}
	\label{section:cstu33}



	% TABLE FOR VARIABLE DETAILS
  % '#' has to be escaped
    \vspace*{0.5cm}
    \noindent\textbf{Eigenschaften\footnote{Detailliertere Informationen zur Variable finden sich unter
		\url{https://metadata.fdz.dzhw.eu/\#!/de/variables/var-gsl2008-ds1-cstu33$}}}\\
	\begin{tabularx}{\hsize}{@{}lX}
	Datentyp: & numerisch \\
	Skalenniveau: & ordinal \\
	Zugangswege: &
	  remote-desktop-suf, 
	  onsite-suf
 \\
    \end{tabularx}



    %TABLE FOR QUESTION DETAILS
    %This has to be tested and has to be improved
    %rausfinden, ob einer Variable mehrere Fragen zugeordnet werden
    %dann evtl. nur die erste verwenden oder etwas anderes tun (Hinweis mehrere Fragen, auflisten mit Link)
				%TABLE FOR QUESTION DETAILS
				\vspace*{0.5cm}
                \noindent\textbf{Frage\footnote{Detailliertere Informationen zur Frage finden sich unter
		              \url{https://metadata.fdz.dzhw.eu/\#!/de/questions/que-gsl2008-ins3-4.16$}}}\\
				\begin{tabularx}{\hsize}{@{}lX}
					Fragenummer: &
					  Fragebogen des DZHW-Studienberechtigtenpanels 2008 - dritte Welle:
					  4.16
 \\
					%--
					Fragetext: & Beabsichtigen Sie, künftig ein weiteres Studium aufzunehmen \\
				\end{tabularx}





				%TABLE FOR THE NOMINAL / ORDINAL VALUES
        		\vspace*{0.5cm}
                \noindent\textbf{Häufigkeiten}

                \vspace*{-\baselineskip}
					%NUMERIC ELEMENTS NEED A HUGH SECOND COLOUMN AND A SMALL FIRST ONE
					\begin{filecontents}{\jobname-cstu33}
					\begin{longtable}{lXrrr}
					\toprule
					\textbf{Wert} & \textbf{Label} & \textbf{Häufigkeit} & \textbf{Prozent(gültig)} & \textbf{Prozent} \\
					\endhead
					\midrule
					\multicolumn{5}{l}{\textbf{Gültige Werte}}\\
						%DIFFERENT OBSERVATIONS <=20

					-949 &
				% TODO try size/length gt 0; take over for other passages
					\multicolumn{1}{X}{ Zweitstudium bereits aufgenommen   } &


					%875 &
					  \num{875} &
					%--
					  \num[round-mode=places,round-precision=2]{28.44} &
					    \num[round-mode=places,round-precision=2]{3.1} \\
							%????

					-948 &
				% TODO try size/length gt 0; take over for other passages
					\multicolumn{1}{X}{ Doppelstudium   } &


					%14 &
					  \num{14} &
					%--
					  \num[round-mode=places,round-precision=2]{0.45} &
					    \num[round-mode=places,round-precision=2]{0.05} \\
							%????

					1 &
				% TODO try size/length gt 0; take over for other passages
					\multicolumn{1}{X}{ ja, auf jeden Fall   } &


					%375 &
					  \num{375} &
					%--
					  \num[round-mode=places,round-precision=2]{12.19} &
					    \num[round-mode=places,round-precision=2]{1.33} \\
							%????

					2 &
				% TODO try size/length gt 0; take over for other passages
					\multicolumn{1}{X}{ ja, wahrscheinlich   } &


					%339 &
					  \num{339} &
					%--
					  \num[round-mode=places,round-precision=2]{11.02} &
					    \num[round-mode=places,round-precision=2]{1.2} \\
							%????

					3 &
				% TODO try size/length gt 0; take over for other passages
					\multicolumn{1}{X}{ eventuell   } &


					%307 &
					  \num{307} &
					%--
					  \num[round-mode=places,round-precision=2]{9.98} &
					    \num[round-mode=places,round-precision=2]{1.09} \\
							%????

					4 &
				% TODO try size/length gt 0; take over for other passages
					\multicolumn{1}{X}{ nein, wahrscheinlich nicht   } &


					%645 &
					  \num{645} &
					%--
					  \num[round-mode=places,round-precision=2]{20.96} &
					    \num[round-mode=places,round-precision=2]{2.29} \\
							%????

					5 &
				% TODO try size/length gt 0; take over for other passages
					\multicolumn{1}{X}{ nein, auf keinen Fall   } &


					%522 &
					  \num{522} &
					%--
					  \num[round-mode=places,round-precision=2]{16.96} &
					    \num[round-mode=places,round-precision=2]{1.85} \\
							%????
						%DIFFERENT OBSERVATIONS >20
					\midrule
					\multicolumn{2}{l}{Summe (gültig)} &
					  \textbf{\num{3077}} &
					\textbf{100} &
					  \textbf{\num[round-mode=places,round-precision=2]{10.92}} \\
					%--
					\multicolumn{5}{l}{\textbf{Fehlende Werte}}\\
							-998 &
							keine Angabe &
							  \num{6} &
							 - &
							  \num[round-mode=places,round-precision=2]{0.02} \\
							-995 &
							keine Teilnahme (Panel) &
							  \num{24511} &
							 - &
							  \num[round-mode=places,round-precision=2]{86.97} \\
							-989 &
							filterbedingt fehlend &
							  \num{588} &
							 - &
							  \num[round-mode=places,round-precision=2]{2.09} \\
					\midrule
					\multicolumn{2}{l}{\textbf{Summe (gesamt)}} &
				      \textbf{\num{28182}} &
				    \textbf{-} &
				    \textbf{100} \\
					\bottomrule
					\end{longtable}
					\end{filecontents}
					\LTXtable{\textwidth}{\jobname-cstu33}
				\label{tableValues:cstu33}
				\vspace*{-\baselineskip}
                    \begin{noten}
                	    \note{} Deskriptive Maßzahlen:
                	    Anzahl unterschiedlicher Beobachtungen: 7%
                	    ; 
                	      Minimum ($min$): -949; 
                	      Maximum ($max$): 5; 
                	      Median ($\tilde{x}$): 2; 
                	      Modus ($h$): -949
                     \end{noten}


