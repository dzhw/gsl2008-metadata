%EVERY VARIABLE HAS IT'S OWN PAGE

    \setcounter{footnote}{0}

    %omit vertical space
    \vspace*{-1.8cm}
	\section{ainf06a (Informationsquelle Zukunft: Eignungs- und Persönlichkeitstests)}
	\label{section:ainf06a}



	% TABLE FOR VARIABLE DETAILS
  % '#' has to be escaped
    \vspace*{0.5cm}
    \noindent\textbf{Eigenschaften\footnote{Detailliertere Informationen zur Variable finden sich unter
		\url{https://metadata.fdz.dzhw.eu/\#!/de/variables/var-gsl2008-ds1-ainf06a$}}}\\
	\begin{tabularx}{\hsize}{@{}lX}
	Datentyp: & numerisch \\
	Skalenniveau: & ordinal \\
	Zugangswege: &
	  remote-desktop-suf, 
	  onsite-suf
 \\
    \end{tabularx}



    %TABLE FOR QUESTION DETAILS
    %This has to be tested and has to be improved
    %rausfinden, ob einer Variable mehrere Fragen zugeordnet werden
    %dann evtl. nur die erste verwenden oder etwas anderes tun (Hinweis mehrere Fragen, auflisten mit Link)
				%TABLE FOR QUESTION DETAILS
				\vspace*{0.5cm}
                \noindent\textbf{Frage\footnote{Detailliertere Informationen zur Frage finden sich unter
		              \url{https://metadata.fdz.dzhw.eu/\#!/de/questions/que-gsl2008-ins1-25$}}}\\
				\begin{tabularx}{\hsize}{@{}lX}
					Fragenummer: &
					  Fragebogen des DZHW-Studienberechtigtenpanels 2008 - erste Welle:
					  25
 \\
					%--
					Fragetext: & Schätzen Sie bitte ein, in welchem Maße Sie aus den folgenden Quellen wichtige Informationen für Ihre Studien-/Ausbildungsplanung erhalten haben und wie hilfreich diese waren.\par  Informationsangebote\par  Eignungstests, Persönlichkeitstests \\
				\end{tabularx}





				%TABLE FOR THE NOMINAL / ORDINAL VALUES
        		\vspace*{0.5cm}
                \noindent\textbf{Häufigkeiten}

                \vspace*{-\baselineskip}
					%NUMERIC ELEMENTS NEED A HUGH SECOND COLOUMN AND A SMALL FIRST ONE
					\begin{filecontents}{\jobname-ainf06a}
					\begin{longtable}{lXrrr}
					\toprule
					\textbf{Wert} & \textbf{Label} & \textbf{Häufigkeit} & \textbf{Prozent(gültig)} & \textbf{Prozent} \\
					\endhead
					\midrule
					\multicolumn{5}{l}{\textbf{Gültige Werte}}\\
						%DIFFERENT OBSERVATIONS <=20

					1 &
				% TODO try size/length gt 0; take over for other passages
					\multicolumn{1}{X}{ sehr hilfreich   } &


					%2002 &
					  \num{2002} &
					%--
					  \num[round-mode=places,round-precision=2]{13.34} &
					    \num[round-mode=places,round-precision=2]{7.1} \\
							%????

					2 &
				% TODO try size/length gt 0; take over for other passages
					\multicolumn{1}{X}{ 2   } &


					%4495 &
					  \num{4495} &
					%--
					  \num[round-mode=places,round-precision=2]{29.95} &
					    \num[round-mode=places,round-precision=2]{15.95} \\
							%????

					3 &
				% TODO try size/length gt 0; take over for other passages
					\multicolumn{1}{X}{ 3   } &


					%4692 &
					  \num{4692} &
					%--
					  \num[round-mode=places,round-precision=2]{31.26} &
					    \num[round-mode=places,round-precision=2]{16.65} \\
							%????

					4 &
				% TODO try size/length gt 0; take over for other passages
					\multicolumn{1}{X}{ 4   } &


					%2287 &
					  \num{2287} &
					%--
					  \num[round-mode=places,round-precision=2]{15.24} &
					    \num[round-mode=places,round-precision=2]{8.12} \\
							%????

					5 &
				% TODO try size/length gt 0; take over for other passages
					\multicolumn{1}{X}{ nicht hilfreich   } &


					%1534 &
					  \num{1534} &
					%--
					  \num[round-mode=places,round-precision=2]{10.22} &
					    \num[round-mode=places,round-precision=2]{5.44} \\
							%????
						%DIFFERENT OBSERVATIONS >20
					\midrule
					\multicolumn{2}{l}{Summe (gültig)} &
					  \textbf{\num{15010}} &
					\textbf{100} &
					  \textbf{\num[round-mode=places,round-precision=2]{53.26}} \\
					%--
					\multicolumn{5}{l}{\textbf{Fehlende Werte}}\\
							-998 &
							keine Angabe &
							  \num{1252} &
							 - &
							  \num[round-mode=places,round-precision=2]{4.44} \\
							-989 &
							filterbedingt fehlend &
							  \num{2966} &
							 - &
							  \num[round-mode=places,round-precision=2]{10.52} \\
							-988 &
							trifft nicht zu &
							  \num{8954} &
							 - &
							  \num[round-mode=places,round-precision=2]{31.77} \\
					\midrule
					\multicolumn{2}{l}{\textbf{Summe (gesamt)}} &
				      \textbf{\num{28182}} &
				    \textbf{-} &
				    \textbf{100} \\
					\bottomrule
					\end{longtable}
					\end{filecontents}
					\LTXtable{\textwidth}{\jobname-ainf06a}
				\label{tableValues:ainf06a}
				\vspace*{-\baselineskip}
                    \begin{noten}
                	    \note{} Deskriptive Maßzahlen:
                	    Anzahl unterschiedlicher Beobachtungen: 5%
                	    ; 
                	      Minimum ($min$): 1; 
                	      Maximum ($max$): 5; 
                	      Median ($\tilde{x}$): 3; 
                	      Modus ($h$): 3
                     \end{noten}


