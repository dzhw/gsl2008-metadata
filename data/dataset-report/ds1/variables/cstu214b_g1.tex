%EVERY VARIABLE HAS IT'S OWN PAGE

    \setcounter{footnote}{0}

    %omit vertical space
    \vspace*{-1.8cm}
	\section{cstu214b\_g1 (4. Tätigkeit: 2. Studienfach (Studienbereich))}
	\label{section:cstu214b_g1}



	% TABLE FOR VARIABLE DETAILS
  % '#' has to be escaped
    \vspace*{0.5cm}
    \noindent\textbf{Eigenschaften\footnote{Detailliertere Informationen zur Variable finden sich unter
		\url{https://metadata.fdz.dzhw.eu/\#!/de/variables/var-gsl2008-ds1-cstu214b_g1$}}}\\
	\begin{tabularx}{\hsize}{@{}lX}
	Datentyp: & numerisch \\
	Skalenniveau: & nominal \\
	Zugangswege: &
	  remote-desktop-suf, 
	  onsite-suf
 \\
    \end{tabularx}



    %TABLE FOR QUESTION DETAILS
    %This has to be tested and has to be improved
    %rausfinden, ob einer Variable mehrere Fragen zugeordnet werden
    %dann evtl. nur die erste verwenden oder etwas anderes tun (Hinweis mehrere Fragen, auflisten mit Link)
				%TABLE FOR QUESTION DETAILS
				\vspace*{0.5cm}
                \noindent\textbf{Frage\footnote{Detailliertere Informationen zur Frage finden sich unter
		              \url{https://metadata.fdz.dzhw.eu/\#!/de/questions/que-gsl2008-ins3-2.1$}}}\\
				\begin{tabularx}{\hsize}{@{}lX}
					Fragenummer: &
					  Fragebogen des DZHW-Studienberechtigtenpanels 2008 - dritte Welle:
					  2.1
 \\
					%--
					Fragetext: & Wir bitten Sie nun, uns in dem folgenden Schema einen Überblick Ihres Werdegangs von Juli 2008 bis Dezember 2012 zu geben. \\
				\end{tabularx}





				%TABLE FOR THE NOMINAL / ORDINAL VALUES
        		\vspace*{0.5cm}
                \noindent\textbf{Häufigkeiten}

                \vspace*{-\baselineskip}
					%NUMERIC ELEMENTS NEED A HUGH SECOND COLOUMN AND A SMALL FIRST ONE
					\begin{filecontents}{\jobname-cstu214b_g1}
					\begin{longtable}{lXrrr}
					\toprule
					\textbf{Wert} & \textbf{Label} & \textbf{Häufigkeit} & \textbf{Prozent(gültig)} & \textbf{Prozent} \\
					\endhead
					\midrule
					\multicolumn{5}{l}{\textbf{Gültige Werte}}\\
						%DIFFERENT OBSERVATIONS <=20
								1 & \multicolumn{1}{X}{Sprach- und Kulturwissenschaften allgemein} & %5 &
								  \num{5} &
								%--
								  \num[round-mode=places,round-precision=2]{1.28} &
								  \num[round-mode=places,round-precision=2]{0.02} \\
								2 & \multicolumn{1}{X}{Evang. Theologie, -Religionslehre} & %10 &
								  \num{10} &
								%--
								  \num[round-mode=places,round-precision=2]{2.56} &
								  \num[round-mode=places,round-precision=2]{0.04} \\
								3 & \multicolumn{1}{X}{Kath. Theologie, Religionslehre} & %3 &
								  \num{3} &
								%--
								  \num[round-mode=places,round-precision=2]{0.77} &
								  \num[round-mode=places,round-precision=2]{0.01} \\
								4 & \multicolumn{1}{X}{Philosophie} & %6 &
								  \num{6} &
								%--
								  \num[round-mode=places,round-precision=2]{1.54} &
								  \num[round-mode=places,round-precision=2]{0.02} \\
								5 & \multicolumn{1}{X}{Geschichte} & %27 &
								  \num{27} &
								%--
								  \num[round-mode=places,round-precision=2]{6.92} &
								  \num[round-mode=places,round-precision=2]{0.1} \\
								6 & \multicolumn{1}{X}{Bibliothekswissenschaft, Dokumentation Publizistik} & %17 &
								  \num{17} &
								%--
								  \num[round-mode=places,round-precision=2]{4.36} &
								  \num[round-mode=places,round-precision=2]{0.06} \\
								7 & \multicolumn{1}{X}{Allgemeine und vergleichende Literatur- und Sprachwissenschaft} & %8 &
								  \num{8} &
								%--
								  \num[round-mode=places,round-precision=2]{2.05} &
								  \num[round-mode=places,round-precision=2]{0.03} \\
								8 & \multicolumn{1}{X}{Altphilologie (klass. Phililologie), Neugriechisch} & %7 &
								  \num{7} &
								%--
								  \num[round-mode=places,round-precision=2]{1.79} &
								  \num[round-mode=places,round-precision=2]{0.02} \\
								9 & \multicolumn{1}{X}{Germanistik (Deutsch, germanische Sprachen ohne Anglistik)} & %26 &
								  \num{26} &
								%--
								  \num[round-mode=places,round-precision=2]{6.67} &
								  \num[round-mode=places,round-precision=2]{0.09} \\
								10 & \multicolumn{1}{X}{Anglistik, Amerikanistik} & %28 &
								  \num{28} &
								%--
								  \num[round-mode=places,round-precision=2]{7.18} &
								  \num[round-mode=places,round-precision=2]{0.1} \\
							... & ... & ... & ... & ... \\
								58 & \multicolumn{1}{X}{Agrarwissenschaften, Lebensmittel- und Getränketechnologie} & %1 &
								  \num{1} &
								%--
								  \num[round-mode=places,round-precision=2]{0.26} &
								  \num[round-mode=places,round-precision=2]{0} \\

								63 & \multicolumn{1}{X}{Maschinenbau/Verfahrenstechnik} & %13 &
								  \num{13} &
								%--
								  \num[round-mode=places,round-precision=2]{3.33} &
								  \num[round-mode=places,round-precision=2]{0.05} \\

								64 & \multicolumn{1}{X}{Elektrotechnik} & %3 &
								  \num{3} &
								%--
								  \num[round-mode=places,round-precision=2]{0.77} &
								  \num[round-mode=places,round-precision=2]{0.01} \\

								66 & \multicolumn{1}{X}{Architektur, Innenarchitektur} & %1 &
								  \num{1} &
								%--
								  \num[round-mode=places,round-precision=2]{0.26} &
								  \num[round-mode=places,round-precision=2]{0} \\

								67 & \multicolumn{1}{X}{Raumplanung} & %1 &
								  \num{1} &
								%--
								  \num[round-mode=places,round-precision=2]{0.26} &
								  \num[round-mode=places,round-precision=2]{0} \\

								68 & \multicolumn{1}{X}{Bauningenieurwesen} & %1 &
								  \num{1} &
								%--
								  \num[round-mode=places,round-precision=2]{0.26} &
								  \num[round-mode=places,round-precision=2]{0} \\

								74 & \multicolumn{1}{X}{Kunst, Kunstwissenschaft allgemein} & %7 &
								  \num{7} &
								%--
								  \num[round-mode=places,round-precision=2]{1.79} &
								  \num[round-mode=places,round-precision=2]{0.02} \\

								76 & \multicolumn{1}{X}{Gestaltung} & %1 &
								  \num{1} &
								%--
								  \num[round-mode=places,round-precision=2]{0.26} &
								  \num[round-mode=places,round-precision=2]{0} \\

								77 & \multicolumn{1}{X}{Darstellende Kunst, Film und Fernsehen, Theaterwissenschaft} & %3 &
								  \num{3} &
								%--
								  \num[round-mode=places,round-precision=2]{0.77} &
								  \num[round-mode=places,round-precision=2]{0.01} \\

								78 & \multicolumn{1}{X}{Musik, Musikwissenschaft} & %2 &
								  \num{2} &
								%--
								  \num[round-mode=places,round-precision=2]{0.51} &
								  \num[round-mode=places,round-precision=2]{0.01} \\

					\midrule
					\multicolumn{2}{l}{Summe (gültig)} &
					  \textbf{\num{390}} &
					\textbf{100} &
					  \textbf{\num[round-mode=places,round-precision=2]{1.38}} \\
					%--
					\multicolumn{5}{l}{\textbf{Fehlende Werte}}\\
							-998 &
							keine Angabe &
							  \num{3281} &
							 - &
							  \num[round-mode=places,round-precision=2]{11.64} \\
							-995 &
							keine Teilnahme (Panel) &
							  \num{24511} &
							 - &
							  \num[round-mode=places,round-precision=2]{86.97} \\
					\midrule
					\multicolumn{2}{l}{\textbf{Summe (gesamt)}} &
				      \textbf{\num{28182}} &
				    \textbf{-} &
				    \textbf{100} \\
					\bottomrule
					\end{longtable}
					\end{filecontents}
					\LTXtable{\textwidth}{\jobname-cstu214b_g1}
				\label{tableValues:cstu214b_g1}
				\vspace*{-\baselineskip}
                    \begin{noten}
                	    \note{} Deskriptive Maßzahlen:
                	    Anzahl unterschiedlicher Beobachtungen: 47%
                	    ; 
                	      Modus ($h$): 30
                     \end{noten}


