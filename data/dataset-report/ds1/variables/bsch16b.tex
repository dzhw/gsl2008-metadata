%EVERY VARIABLE HAS IT'S OWN PAGE

    \setcounter{footnote}{0}

    %omit vertical space
    \vspace*{-1.8cm}
	\section{bsch16b (3. Prüfungsfach: wöchentliche Stundenzahl)}
	\label{section:bsch16b}



	% TABLE FOR VARIABLE DETAILS
  % '#' has to be escaped
    \vspace*{0.5cm}
    \noindent\textbf{Eigenschaften\footnote{Detailliertere Informationen zur Variable finden sich unter
		\url{https://metadata.fdz.dzhw.eu/\#!/de/variables/var-gsl2008-ds1-bsch16b$}}}\\
	\begin{tabularx}{\hsize}{@{}lX}
	Datentyp: & numerisch \\
	Skalenniveau: & verhältnis \\
	Zugangswege: &
	  remote-desktop-suf, 
	  onsite-suf
 \\
    \end{tabularx}



    %TABLE FOR QUESTION DETAILS
    %This has to be tested and has to be improved
    %rausfinden, ob einer Variable mehrere Fragen zugeordnet werden
    %dann evtl. nur die erste verwenden oder etwas anderes tun (Hinweis mehrere Fragen, auflisten mit Link)
				%TABLE FOR QUESTION DETAILS
				\vspace*{0.5cm}
                \noindent\textbf{Frage\footnote{Detailliertere Informationen zur Frage finden sich unter
		              \url{https://metadata.fdz.dzhw.eu/\#!/de/questions/que-gsl2008-ins2-04$}}}\\
				\begin{tabularx}{\hsize}{@{}lX}
					Fragenummer: &
					  Fragebogen des DZHW-Studienberechtigtenpanels 2008 - zweite Welle:
					  04
 \\
					%--
					Fragetext: & Bitte nennen Sie Ihre Prüfungsfächer und geben Sie zusätzlich an, mit welcher wöchentlichen Stundenzahl diese in Ihrem Abschlussjahr unterrichtet wurden.\par  Stunden je Woche \\
				\end{tabularx}





				%TABLE FOR THE NOMINAL / ORDINAL VALUES
        		\vspace*{0.5cm}
                \noindent\textbf{Häufigkeiten}

                \vspace*{-\baselineskip}
					%NUMERIC ELEMENTS NEED A HUGH SECOND COLOUMN AND A SMALL FIRST ONE
					\begin{filecontents}{\jobname-bsch16b}
					\begin{longtable}{lXrrr}
					\toprule
					\textbf{Wert} & \textbf{Label} & \textbf{Häufigkeit} & \textbf{Prozent(gültig)} & \textbf{Prozent} \\
					\endhead
					\midrule
					\multicolumn{5}{l}{\textbf{Gültige Werte}}\\
						%DIFFERENT OBSERVATIONS <=20

					1 &
				% TODO try size/length gt 0; take over for other passages
					\multicolumn{1}{X}{ -  } &


					%8 &
					  \num{8} &
					%--
					  \num[round-mode=places,round-precision=2]{0.15} &
					    \num[round-mode=places,round-precision=2]{0.03} \\
							%????

					2 &
				% TODO try size/length gt 0; take over for other passages
					\multicolumn{1}{X}{ -  } &


					%222 &
					  \num{222} &
					%--
					  \num[round-mode=places,round-precision=2]{4.07} &
					    \num[round-mode=places,round-precision=2]{0.79} \\
							%????

					3 &
				% TODO try size/length gt 0; take over for other passages
					\multicolumn{1}{X}{ -  } &


					%1926 &
					  \num{1926} &
					%--
					  \num[round-mode=places,round-precision=2]{35.34} &
					    \num[round-mode=places,round-precision=2]{6.83} \\
							%????

					4 &
				% TODO try size/length gt 0; take over for other passages
					\multicolumn{1}{X}{ -  } &


					%2622 &
					  \num{2622} &
					%--
					  \num[round-mode=places,round-precision=2]{48.11} &
					    \num[round-mode=places,round-precision=2]{9.3} \\
							%????

					5 &
				% TODO try size/length gt 0; take over for other passages
					\multicolumn{1}{X}{ -  } &


					%486 &
					  \num{486} &
					%--
					  \num[round-mode=places,round-precision=2]{8.92} &
					    \num[round-mode=places,round-precision=2]{1.72} \\
							%????

					6 &
				% TODO try size/length gt 0; take over for other passages
					\multicolumn{1}{X}{ -  } &


					%156 &
					  \num{156} &
					%--
					  \num[round-mode=places,round-precision=2]{2.86} &
					    \num[round-mode=places,round-precision=2]{0.55} \\
							%????

					7 &
				% TODO try size/length gt 0; take over for other passages
					\multicolumn{1}{X}{ -  } &


					%6 &
					  \num{6} &
					%--
					  \num[round-mode=places,round-precision=2]{0.11} &
					    \num[round-mode=places,round-precision=2]{0.02} \\
							%????

					8 &
				% TODO try size/length gt 0; take over for other passages
					\multicolumn{1}{X}{ -  } &


					%20 &
					  \num{20} &
					%--
					  \num[round-mode=places,round-precision=2]{0.37} &
					    \num[round-mode=places,round-precision=2]{0.07} \\
							%????

					9 &
				% TODO try size/length gt 0; take over for other passages
					\multicolumn{1}{X}{ -  } &


					%4 &
					  \num{4} &
					%--
					  \num[round-mode=places,round-precision=2]{0.07} &
					    \num[round-mode=places,round-precision=2]{0.01} \\
							%????
						%DIFFERENT OBSERVATIONS >20
					\midrule
					\multicolumn{2}{l}{Summe (gültig)} &
					  \textbf{\num{5450}} &
					\textbf{\num{100}} &
					  \textbf{\num[round-mode=places,round-precision=2]{19.34}} \\
					%--
					\multicolumn{5}{l}{\textbf{Fehlende Werte}}\\
							-998 &
							keine Angabe &
							  \num{483} &
							 - &
							  \num[round-mode=places,round-precision=2]{1.71} \\
							-995 &
							keine Teilnahme (Panel) &
							  \num{22249} &
							 - &
							  \num[round-mode=places,round-precision=2]{78.95} \\
					\midrule
					\multicolumn{2}{l}{\textbf{Summe (gesamt)}} &
				      \textbf{\num{28182}} &
				    \textbf{-} &
				    \textbf{\num{100}} \\
					\bottomrule
					\end{longtable}
					\end{filecontents}
					\LTXtable{\textwidth}{\jobname-bsch16b}
				\label{tableValues:bsch16b}
				\vspace*{-\baselineskip}
                    \begin{noten}
                	    \note{} Deskriptive Maßzahlen:
                	    Anzahl unterschiedlicher Beobachtungen: 9%
                	    ; 
                	      Minimum ($min$): 1; 
                	      Maximum ($max$): 9; 
                	      arithmetisches Mittel ($\bar{x}$): \num[round-mode=places,round-precision=2]{3.7288}; 
                	      Median ($\tilde{x}$): 4; 
                	      Modus ($h$): 4; 
                	      Standardabweichung ($s$): \num[round-mode=places,round-precision=2]{0.8642}; 
                	      Schiefe ($v$): \num[round-mode=places,round-precision=2]{0.9036}; 
                	      Wölbung ($w$): \num[round-mode=places,round-precision=2]{6.1996}
                     \end{noten}

