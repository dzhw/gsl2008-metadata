%EVERY VARIABLE HAS IT'S OWN PAGE

    \setcounter{footnote}{0}

    %omit vertical space
    \vspace*{-1.8cm}
	\section{asch01\_g1o (Schulart (erfragt und aggregiert))}
	\label{section:asch01_g1o}



	% TABLE FOR VARIABLE DETAILS
  % '#' has to be escaped
    \vspace*{0.5cm}
    \noindent\textbf{Eigenschaften\footnote{Detailliertere Informationen zur Variable finden sich unter
		\url{https://metadata.fdz.dzhw.eu/\#!/de/variables/var-gsl2008-ds1-asch01_g1o$}}}\\
	\begin{tabularx}{\hsize}{@{}lX}
	Datentyp: & numerisch \\
	Skalenniveau: & nominal \\
	Zugangswege: &
	  onsite-suf
 \\
    \end{tabularx}



    %TABLE FOR QUESTION DETAILS
    %This has to be tested and has to be improved
    %rausfinden, ob einer Variable mehrere Fragen zugeordnet werden
    %dann evtl. nur die erste verwenden oder etwas anderes tun (Hinweis mehrere Fragen, auflisten mit Link)
				%TABLE FOR QUESTION DETAILS
				\vspace*{0.5cm}
                \noindent\textbf{Frage\footnote{Detailliertere Informationen zur Frage finden sich unter
		              \url{https://metadata.fdz.dzhw.eu/\#!/de/questions/que-gsl2008-ins1-01$}}}\\
				\begin{tabularx}{\hsize}{@{}lX}
					Fragenummer: &
					  Fragebogen des DZHW-Studienberechtigtenpanels 2008 - erste Welle:
					  01
 \\
					%--
					Fragetext: & Welchen Schultyp bzw. Schulzweig besuchen Sie gegenwärtig? \\
				\end{tabularx}





				%TABLE FOR THE NOMINAL / ORDINAL VALUES
        		\vspace*{0.5cm}
                \noindent\textbf{Häufigkeiten}

                \vspace*{-\baselineskip}
					%NUMERIC ELEMENTS NEED A HUGH SECOND COLOUMN AND A SMALL FIRST ONE
					\begin{filecontents}{\jobname-asch01_g1o}
					\begin{longtable}{lXrrr}
					\toprule
					\textbf{Wert} & \textbf{Label} & \textbf{Häufigkeit} & \textbf{Prozent(gültig)} & \textbf{Prozent} \\
					\endhead
					\midrule
					\multicolumn{5}{l}{\textbf{Gültige Werte}}\\
						%DIFFERENT OBSERVATIONS <=20

					1 &
				% TODO try size/length gt 0; take over for other passages
					\multicolumn{1}{X}{ Gymnasium   } &


					%14143 &
					  \num{14143} &
					%--
					  \num[round-mode=places,round-precision=2]{50.18} &
					    \num[round-mode=places,round-precision=2]{50.18} \\
							%????

					7 &
				% TODO try size/length gt 0; take over for other passages
					\multicolumn{1}{X}{ gymnasiale Oberstufe einer Berufsfachschule   } &


					%276 &
					  \num{276} &
					%--
					  \num[round-mode=places,round-precision=2]{0.98} &
					    \num[round-mode=places,round-precision=2]{0.98} \\
							%????

					9 &
				% TODO try size/length gt 0; take over for other passages
					\multicolumn{1}{X}{ Berufsoberschule   } &


					%989 &
					  \num{989} &
					%--
					  \num[round-mode=places,round-precision=2]{3.51} &
					    \num[round-mode=places,round-precision=2]{3.51} \\
							%????

					10 &
				% TODO try size/length gt 0; take over for other passages
					\multicolumn{1}{X}{ Fachoberschule   } &


					%4727 &
					  \num{4727} &
					%--
					  \num[round-mode=places,round-precision=2]{16.77} &
					    \num[round-mode=places,round-precision=2]{16.77} \\
							%????

					11 &
				% TODO try size/length gt 0; take over for other passages
					\multicolumn{1}{X}{ (höhere) Berufsfachschule   } &


					%1629 &
					  \num{1629} &
					%--
					  \num[round-mode=places,round-precision=2]{5.78} &
					    \num[round-mode=places,round-precision=2]{5.78} \\
							%????

					14 &
				% TODO try size/length gt 0; take over for other passages
					\multicolumn{1}{X}{ Abendgymnasium/Kolleg   } &


					%740 &
					  \num{740} &
					%--
					  \num[round-mode=places,round-precision=2]{2.63} &
					    \num[round-mode=places,round-precision=2]{2.63} \\
							%????

					15 &
				% TODO try size/length gt 0; take over for other passages
					\multicolumn{1}{X}{ Fachschule/-akademie   } &


					%1435 &
					  \num{1435} &
					%--
					  \num[round-mode=places,round-precision=2]{5.09} &
					    \num[round-mode=places,round-precision=2]{5.09} \\
							%????

					16 &
				% TODO try size/length gt 0; take over for other passages
					\multicolumn{1}{X}{ Gesamtschule/Waldorfschule   } &


					%1439 &
					  \num{1439} &
					%--
					  \num[round-mode=places,round-precision=2]{5.11} &
					    \num[round-mode=places,round-precision=2]{5.11} \\
							%????

					17 &
				% TODO try size/length gt 0; take over for other passages
					\multicolumn{1}{X}{ Fachgymnasium/gymnasiale Oberstufe im Oberstufenzentrum   } &


					%2804 &
					  \num{2804} &
					%--
					  \num[round-mode=places,round-precision=2]{9.95} &
					    \num[round-mode=places,round-precision=2]{9.95} \\
							%????
						%DIFFERENT OBSERVATIONS >20
					\midrule
					\multicolumn{2}{l}{Summe (gültig)} &
					  \textbf{\num{28182}} &
					\textbf{\num{100}} &
					  \textbf{\num[round-mode=places,round-precision=2]{100}} \\
					%--
					\multicolumn{5}{l}{\textbf{Fehlende Werte}}\\
						& & 0 & 0 & 0 \\
					\midrule
					\multicolumn{2}{l}{\textbf{Summe (gesamt)}} &
				      \textbf{\num{28182}} &
				    \textbf{-} &
				    \textbf{\num{100}} \\
					\bottomrule
					\end{longtable}
					\end{filecontents}
					\LTXtable{\textwidth}{\jobname-asch01_g1o}
				\label{tableValues:asch01_g1o}
				\vspace*{-\baselineskip}
                    \begin{noten}
                	    \note{} Deskriptive Maßzahlen:
                	    Anzahl unterschiedlicher Beobachtungen: 9%
                	    ; 
                	      Modus ($h$): 1
                     \end{noten}

