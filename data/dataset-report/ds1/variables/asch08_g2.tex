%EVERY VARIABLE HAS IT'S OWN PAGE

    \setcounter{footnote}{0}

    %omit vertical space
    \vspace*{-1.8cm}
	\section{asch08\_g2 (Schwerpunktfach (Berufsschüler) (aggregiert))}
	\label{section:asch08_g2}



	% TABLE FOR VARIABLE DETAILS
  % '#' has to be escaped
    \vspace*{0.5cm}
    \noindent\textbf{Eigenschaften\footnote{Detailliertere Informationen zur Variable finden sich unter
		\url{https://metadata.fdz.dzhw.eu/\#!/de/variables/var-gsl2008-ds1-asch08_g2$}}}\\
	\begin{tabularx}{\hsize}{@{}lX}
	Datentyp: & numerisch \\
	Skalenniveau: & nominal \\
	Zugangswege: &
	  remote-desktop-suf, 
	  onsite-suf
 \\
    \end{tabularx}



    %TABLE FOR QUESTION DETAILS
    %This has to be tested and has to be improved
    %rausfinden, ob einer Variable mehrere Fragen zugeordnet werden
    %dann evtl. nur die erste verwenden oder etwas anderes tun (Hinweis mehrere Fragen, auflisten mit Link)
				%TABLE FOR QUESTION DETAILS
				\vspace*{0.5cm}
                \noindent\textbf{Frage\footnote{Detailliertere Informationen zur Frage finden sich unter
		              \url{https://metadata.fdz.dzhw.eu/\#!/de/questions/que-gsl2008-ins1-05$}}}\\
				\begin{tabularx}{\hsize}{@{}lX}
					Fragenummer: &
					  Fragebogen des DZHW-Studienberechtigtenpanels 2008 - erste Welle:
					  05
 \\
					%--
					Fragetext: & Der Unterricht in meinem Schwerpunktfach, und zwar ...\par  war... \\
				\end{tabularx}





				%TABLE FOR THE NOMINAL / ORDINAL VALUES
        		\vspace*{0.5cm}
                \noindent\textbf{Häufigkeiten}

                \vspace*{-\baselineskip}
					%NUMERIC ELEMENTS NEED A HUGH SECOND COLOUMN AND A SMALL FIRST ONE
					\begin{filecontents}{\jobname-asch08_g2}
					\begin{longtable}{lXrrr}
					\toprule
					\textbf{Wert} & \textbf{Label} & \textbf{Häufigkeit} & \textbf{Prozent(gültig)} & \textbf{Prozent} \\
					\endhead
					\midrule
					\multicolumn{5}{l}{\textbf{Gültige Werte}}\\
						%DIFFERENT OBSERVATIONS <=20

					6 &
				% TODO try size/length gt 0; take over for other passages
					\multicolumn{1}{X}{ Technik (berufsbildende Schule)   } &


					%1793 &
					  \num{1793} &
					%--
					  \num[round-mode=places,round-precision=2]{20.35} &
					    \num[round-mode=places,round-precision=2]{6.36} \\
							%????

					7 &
				% TODO try size/length gt 0; take over for other passages
					\multicolumn{1}{X}{ Bauwesen (berufsbildende Schule)   } &


					%240 &
					  \num{240} &
					%--
					  \num[round-mode=places,round-precision=2]{2.72} &
					    \num[round-mode=places,round-precision=2]{0.85} \\
							%????

					8 &
				% TODO try size/length gt 0; take over for other passages
					\multicolumn{1}{X}{ Informatik (berufsbildende Schule)   } &


					%609 &
					  \num{609} &
					%--
					  \num[round-mode=places,round-precision=2]{6.91} &
					    \num[round-mode=places,round-precision=2]{2.16} \\
							%????

					9 &
				% TODO try size/length gt 0; take over for other passages
					\multicolumn{1}{X}{ Wirtschaft (und Verwaltung) (berufsbildende Schule)   } &


					%3123 &
					  \num{3123} &
					%--
					  \num[round-mode=places,round-precision=2]{35.45} &
					    \num[round-mode=places,round-precision=2]{11.08} \\
							%????

					10 &
				% TODO try size/length gt 0; take over for other passages
					\multicolumn{1}{X}{ Landwirtschaft/ Naturwissenschaften (berufsbildende Schule)   } &


					%192 &
					  \num{192} &
					%--
					  \num[round-mode=places,round-precision=2]{2.18} &
					    \num[round-mode=places,round-precision=2]{0.68} \\
							%????

					11 &
				% TODO try size/length gt 0; take over for other passages
					\multicolumn{1}{X}{ Gestaltung (berufsbildende Schule)   } &


					%589 &
					  \num{589} &
					%--
					  \num[round-mode=places,round-precision=2]{6.69} &
					    \num[round-mode=places,round-precision=2]{2.09} \\
							%????

					12 &
				% TODO try size/length gt 0; take over for other passages
					\multicolumn{1}{X}{ Ernährung/Hauswirtschaft/Gesundheit (berufsbildende Schule)   } &


					%514 &
					  \num{514} &
					%--
					  \num[round-mode=places,round-precision=2]{5.83} &
					    \num[round-mode=places,round-precision=2]{1.82} \\
							%????

					13 &
				% TODO try size/length gt 0; take over for other passages
					\multicolumn{1}{X}{ Sozialwesen (berufsbildende Schule)   } &


					%1641 &
					  \num{1641} &
					%--
					  \num[round-mode=places,round-precision=2]{18.63} &
					    \num[round-mode=places,round-precision=2]{5.82} \\
							%????

					14 &
				% TODO try size/length gt 0; take over for other passages
					\multicolumn{1}{X}{ Sprachen (berufsbildende Schule)   } &


					%88 &
					  \num{88} &
					%--
					  \num[round-mode=places,round-precision=2]{1} &
					    \num[round-mode=places,round-precision=2]{0.31} \\
							%????

					15 &
				% TODO try size/length gt 0; take over for other passages
					\multicolumn{1}{X}{ Medien (und Kommunikation) (berufsbildende Schule)   } &


					%21 &
					  \num{21} &
					%--
					  \num[round-mode=places,round-precision=2]{0.24} &
					    \num[round-mode=places,round-precision=2]{0.07} \\
							%????
						%DIFFERENT OBSERVATIONS >20
					\midrule
					\multicolumn{2}{l}{Summe (gültig)} &
					  \textbf{\num{8810}} &
					\textbf{\num{100}} &
					  \textbf{\num[round-mode=places,round-precision=2]{31.26}} \\
					%--
					\multicolumn{5}{l}{\textbf{Fehlende Werte}}\\
							-998 &
							keine Angabe &
							  \num{3051} &
							 - &
							  \num[round-mode=places,round-precision=2]{10.83} \\
							-988 &
							trifft nicht zu &
							  \num{16321} &
							 - &
							  \num[round-mode=places,round-precision=2]{57.91} \\
					\midrule
					\multicolumn{2}{l}{\textbf{Summe (gesamt)}} &
				      \textbf{\num{28182}} &
				    \textbf{-} &
				    \textbf{\num{100}} \\
					\bottomrule
					\end{longtable}
					\end{filecontents}
					\LTXtable{\textwidth}{\jobname-asch08_g2}
				\label{tableValues:asch08_g2}
				\vspace*{-\baselineskip}
                    \begin{noten}
                	    \note{} Deskriptive Maßzahlen:
                	    Anzahl unterschiedlicher Beobachtungen: 10%
                	    ; 
                	      Modus ($h$): 9
                     \end{noten}

