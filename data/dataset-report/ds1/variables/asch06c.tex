%EVERY VARIABLE HAS IT'S OWN PAGE

    \setcounter{footnote}{0}

    %omit vertical space
    \vspace*{-1.8cm}
	\section{asch06c (Mathe (Fach): praxisnah)}
	\label{section:asch06c}



	% TABLE FOR VARIABLE DETAILS
  % '#' has to be escaped
    \vspace*{0.5cm}
    \noindent\textbf{Eigenschaften\footnote{Detailliertere Informationen zur Variable finden sich unter
		\url{https://metadata.fdz.dzhw.eu/\#!/de/variables/var-gsl2008-ds1-asch06c$}}}\\
	\begin{tabularx}{\hsize}{@{}lX}
	Datentyp: & numerisch \\
	Skalenniveau: & ordinal \\
	Zugangswege: &
	  remote-desktop-suf, 
	  onsite-suf
 \\
    \end{tabularx}



    %TABLE FOR QUESTION DETAILS
    %This has to be tested and has to be improved
    %rausfinden, ob einer Variable mehrere Fragen zugeordnet werden
    %dann evtl. nur die erste verwenden oder etwas anderes tun (Hinweis mehrere Fragen, auflisten mit Link)
				%TABLE FOR QUESTION DETAILS
				\vspace*{0.5cm}
                \noindent\textbf{Frage\footnote{Detailliertere Informationen zur Frage finden sich unter
		              \url{https://metadata.fdz.dzhw.eu/\#!/de/questions/que-gsl2008-ins1-04$}}}\\
				\begin{tabularx}{\hsize}{@{}lX}
					Fragenummer: &
					  Fragebogen des DZHW-Studienberechtigtenpanels 2008 - erste Welle:
					  04
 \\
					%--
					Fragetext: & Der Unterricht der letzten beiden Schuljahre kann unter verschiedenen Gesichtspunkten bewertet werden:\par  Der Unterricht im Fach Mathematik war…\par  praxisnah \\
				\end{tabularx}





				%TABLE FOR THE NOMINAL / ORDINAL VALUES
        		\vspace*{0.5cm}
                \noindent\textbf{Häufigkeiten}

                \vspace*{-\baselineskip}
					%NUMERIC ELEMENTS NEED A HUGH SECOND COLOUMN AND A SMALL FIRST ONE
					\begin{filecontents}{\jobname-asch06c}
					\begin{longtable}{lXrrr}
					\toprule
					\textbf{Wert} & \textbf{Label} & \textbf{Häufigkeit} & \textbf{Prozent(gültig)} & \textbf{Prozent} \\
					\endhead
					\midrule
					\multicolumn{5}{l}{\textbf{Gültige Werte}}\\
						%DIFFERENT OBSERVATIONS <=20

					1 &
				% TODO try size/length gt 0; take over for other passages
					\multicolumn{1}{X}{ trifft voll zu   } &


					%2767 &
					  \num{2767} &
					%--
					  \num[round-mode=places,round-precision=2]{10.27} &
					    \num[round-mode=places,round-precision=2]{9.82} \\
							%????

					2 &
				% TODO try size/length gt 0; take over for other passages
					\multicolumn{1}{X}{ 2   } &


					%6712 &
					  \num{6712} &
					%--
					  \num[round-mode=places,round-precision=2]{24.92} &
					    \num[round-mode=places,round-precision=2]{23.82} \\
							%????

					3 &
				% TODO try size/length gt 0; take over for other passages
					\multicolumn{1}{X}{ 3   } &


					%8529 &
					  \num{8529} &
					%--
					  \num[round-mode=places,round-precision=2]{31.67} &
					    \num[round-mode=places,round-precision=2]{30.26} \\
							%????

					4 &
				% TODO try size/length gt 0; take over for other passages
					\multicolumn{1}{X}{ 4   } &


					%5794 &
					  \num{5794} &
					%--
					  \num[round-mode=places,round-precision=2]{21.51} &
					    \num[round-mode=places,round-precision=2]{20.56} \\
							%????

					5 &
				% TODO try size/length gt 0; take over for other passages
					\multicolumn{1}{X}{ trifft gar nicht zu   } &


					%3129 &
					  \num{3129} &
					%--
					  \num[round-mode=places,round-precision=2]{11.62} &
					    \num[round-mode=places,round-precision=2]{11.1} \\
							%????
						%DIFFERENT OBSERVATIONS >20
					\midrule
					\multicolumn{2}{l}{Summe (gültig)} &
					  \textbf{\num{26931}} &
					\textbf{\num{100}} &
					  \textbf{\num[round-mode=places,round-precision=2]{95.56}} \\
					%--
					\multicolumn{5}{l}{\textbf{Fehlende Werte}}\\
							-999 &
							weiß nicht &
							  \num{857} &
							 - &
							  \num[round-mode=places,round-precision=2]{3.04} \\
							-998 &
							keine Angabe &
							  \num{394} &
							 - &
							  \num[round-mode=places,round-precision=2]{1.4} \\
					\midrule
					\multicolumn{2}{l}{\textbf{Summe (gesamt)}} &
				      \textbf{\num{28182}} &
				    \textbf{-} &
				    \textbf{\num{100}} \\
					\bottomrule
					\end{longtable}
					\end{filecontents}
					\LTXtable{\textwidth}{\jobname-asch06c}
				\label{tableValues:asch06c}
				\vspace*{-\baselineskip}
                    \begin{noten}
                	    \note{} Deskriptive Maßzahlen:
                	    Anzahl unterschiedlicher Beobachtungen: 5%
                	    ; 
                	      Minimum ($min$): 1; 
                	      Maximum ($max$): 5; 
                	      Median ($\tilde{x}$): 3; 
                	      Modus ($h$): 3
                     \end{noten}

